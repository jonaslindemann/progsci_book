%---------------------------------------------------------------------
%---------------------------------------------------------------------
\chapter{Fortran and Python}
%---------------------------------------------------------------------
%---------------------------------------------------------------------

When developing numerical codes today, it is more and more important to be able to combine benefits form many programming languages. One language that have gained popularity in numerical computing i Python. Python is a powerful dynamic scripting language, which is easy to use and combined with the Scipy toolkit it can provide a environment very similar to MATLAB. Python also supports the development of a multitude of different kinds of applications ranging from graphical user interfaces to web interfaces and web services. 

By combining Fortran and Python, the performance of Fortran can be combined with the easy of use and flexibility of Python. This chapter describes a method for combining these languages using a special tool, f2py.

%---------------------------------------------------------------------
\section{Python extension modules}
%---------------------------------------------------------------------

In addition to implement Python modules in Python, modules can also be implemented in C using a special API, which usually is found in the libpython library. To illustrate how an extension module is developed, a simple sum function will be implemented in an extension module, calcualtions, using the Python extension API.

A typical Python extension module requires 3 parts:

\begin{itemize}
\item Exported functions defined using the Python extension API.
\item Module function table declaring all exported functions in the module.
\item Module initialisation function for initialising the module and function table.
\end{itemize}

An exported function must be declared in a format that can be understood by Python. Our exported sum function is declared as shown in the following code:

\cmode

\begin{lstlisting}
static PyObject* 
sum(PyObject *self, PyObject *args)
\end{lstlisting}

The left side declares the return values from the function. This declaration must always be there even if the function does not return anything. Next, the sum function is declared. All exported functions have the same arguments. \cname{PyObject* self} is a pointer to the module instance that the function belongs to. The second argument is a special Python object containing the arguments that the function is called with.

Next, the input arguments must be parsed. This is done using the \cname{PyArg\_ParseTuple}-function in the Python API. This function parses the input arguments, \cvar{args}, for the required parameters. If no match is found the function returns NULL, which will trigger an exception in the Python interpreter. If a match is found the function will assign values to provided C-variables. Argument parsing for our sum function is shown in the following code:

\begin{lstlisting}
// C variables that will contain input values

double a;
double b;

// Parse input arguments

if (!PyArg_ParseTuple(args, "dd", &a, &b))
    return NULL;
\end{lstlisting}

First, variables, \cvar{a} and \cvar{b}, are declared for storing the actual input arguments. The \cname{PyArg\_ParseTuple}-function takes the input parameter, args, and processes this according to a signature string describing the required Python arguments. Our function sum takes two double values as input arguments, the signature string for this is ''dd''. 

Now we have all input data, so now we do our actual computation in C:

\begin{lstlisting}
double c = a + b;
\end{lstlisting}

To be able to use the computed value in Python it has to be converted to a PyObject. This can be done using the function \cname{Py\_BuildValue}. This function is similar to the \cname{PyArg\_ParseTuple}-function as it also uses the signature string to define what Python-datatypes to create. This is used in the last part of the \cname{sum}-function to return a Python-datatype.

\begin{lstlisting}
return Py_BuildValue("d", c);
\end{lstlisting}

The complete sum function then becomes:

\begin{lstlisting}
static PyObject* 
sum(PyObject *self, PyObject *args)
{
    double a;
    double b;

    // Parse input arguments

    if (!PyArg_ParseTuple(args, "dd", &a, &b))
        return NULL;

    // Do our computation

    double c = a + b;

    // Return the results

    return Py_BuildValue("d", c);
}
\end{lstlisting}

To be able to compile this function as an extension module, a function table and module initialisation have to be added. The additional code required is shown below:

\begin{lstlisting}
// Module function table.

static PyMethodDef
module_functions[] = {
    { "sum", sum, METH_VARARGS, "Calculate sum." },
    { NULL }
};

// Module initialisation

void
initcext(void)
{
    Py_InitModule3("cext", module_functions, "A minimal module.");
}
\end{lstlisting}

To build the extension module, the \fname{setuptools} module in NumPy is used. The following \fname{setup.py} is used to build the extension module:

\pymode

\begin{lstlisting}
from numpy.distutils.core import setup, Extension

setup(
	ext_modules = [
		Extension("cext",
			sources=["calculations_c.c"]),
	]
)\end{lstlisting}

Building the module from the command line is then done using the following command:

\cmdmode

\begin{lstlisting}
> python setup.py build
running build
running build_ext
building 'calculations' extension
gcc -fno-strict-aliasing -I/Users/lindemann/anaconda/include -arch x86_64 -DNDEBUG -g -fwrapv -O3 -Wall -Wstrict-prototypes -I/Users/lindemann/anaconda/include/python2.7 -c calculations.c -o build/temp.macosx-10.5-x86_64-2.7/calculations.o
gcc -bundle -undefined dynamic_lookup -L/Users/lindemann/anaconda/lib -arch x86_64 -arch x86_64 build/temp.macosx-10.5-x86_64-2.7/calculations.o -L/Users/lindemann/anaconda/lib -o build/lib.macosx-10.5-x86_64-2.7/calculations.so
\end{lstlisting}

The following example shows how the module can be used like any other module in Python:

\pymode

\begin{lstlisting}
>>> import cext
>>> dir(cext)
['__doc__', '__file__', '__name__', '__package__', 'sum']
>>> s = cext.sum(2.0, 3.0)
>>> print s
5.0
>>>
\end{lstlisting}

%---------------------------------------------------------------------
\section{Integrating Fortran in extension modules}
%---------------------------------------------------------------------

To integrate Fortran in a Python extension module, requires us to compile and link Fortran code into the extension module. To illustrate this, the example in the previous section will be modified to call a fortran subroutine to perform the computation. To link a Fortran routine with a C, the calling convention in Fortran must be adapted to C. In the following example the \fname{iso\_c\_binindg}, \fname{value} and \fname{bind} is used to define a Fortran routine that uses the C calling convention and C datatypes to make the linking easier:

\fmode

\begin{lstlisting}
subroutine forsum(a, b, c) bind(C, name='forsum')

	use iso_c_binding

	real(c_double), value :: a, b
	real(c_double)        :: c

	c = a + b

end subroutine forsum
\end{lstlisting}

The code in the Python extension module is now updated to call the Fortran routine as shown below:

\cmode

\begin{lstlisting}
static PyObject* 
sum(PyObject *self, PyObject *args)
{
    double a;
    double b;
    double c;

    // Parse input arguments

    if (!PyArg_ParseTuple(args, "dd", &a, &b))
        return NULL;

    // Do our computation

    forsum(a, b, &c);

    // Return the results

    return Py_BuildValue("d", c);
}
\end{lstlisting}

The reason for the \& operator is to pass the \cname{c}-variable as a reference to the Fortran routine.

To build the modified extension module, the Fortran routine must be compiled separately and then provided as a \fname{.o}-file to the \fname{setup.py} script:

\pymode

\begin{lstlisting}
from numpy.distutils.core import setup, Extension

setup(
	ext_modules = [
		Extension("fext",
			sources=["fext.c"],
			extra_objects=["forsum.o"])
	]
)
\end{lstlisting}

It is also possible to transfer matrices between Fortran and Python. However, it requires even more complicated binding code. Instead of doing this by hand, special tools can be used to automatically generate the binding code for us as well as enabling us to use NumPy arrays to transfer matrices between Fortran and Python in an efficient way.

%---------------------------------------------------------------------
\section{F2PY}
%---------------------------------------------------------------------

F2PY is a tool developed by Pearu Peterson that parses Fortran code, generates Python wrapper code and compiles it as a Python extension module. F2PY automatically create wrapper code for Fortran arrays, so that NumPy arrays can be passed directly to the generated functions. 

To illustrate the process of generating an extension module with F2PY the following simple Fortran routine will be wrapped as a module:

\fmode

\begin{lstlisting}
subroutine simple(a,b,c)

	real, intent(in) :: a, b
	real, intent(out) :: c

	c = a + b

end subroutine simple
\end{lstlisting}

To be able to use F2PY effectively it is important that the \fname{intent}-attribute is used on the subroutine arguments. If not specified, F2PY, will treat all subroutine parameters as input-variables and no output parameters can be passed back to the the calling Python routine.

To create a Python module from the source code we execute the \cli{f2py}-command on the command line as show below:

\cmmode

\begin{lstlisting}
> f2py -m fortmod -c simple.f90
...
3n535b8krwsz88vl8bm0000gn/T/tmp5STblc/src.macosx-10.5-x86_64-2.7/fortranobject.o /var/folders/w6/1zqjp3n535b8krwsz88vl8bm0000gn/T/tmp5STblc/simple.o -L/opt/local/lib/gcc49/gcc/x86_64-apple-darwin14/4.9.1 -L/Users/lindemann/anaconda/lib -lgfortran -o ./fortmod.so
Removing build directory /var/folders/w6/1zq...
\end{lstlisting}

In the build directory there should now be a \fname{fortmod.so} or a \fname{fortmod.pyd} depending on the platform used.

The new module is loaded and used as shown in the following example:

\pymode

\begin{lstlisting}
>>> import fortmod
>>> print fortmod.simple(2.0, 3.0)
5.0
\end{lstlisting}

F2PY will automatically generate built-in documentation in the module. To display this documentation the \pyvar{\_\_doc\_\_} property is used, as shown in the following example:

\begin{lstlisting}
>>> print fortmod.__doc__
This module 'fortmod' is auto-generated with f2py (version:2).
Functions:
  c = simple(a,b)
.
>>> print fortmod.simple.__doc__
c = simple(a,b)

Wrapper for ``simple``.

Parameters
----------
a : input float
b : input float

Returns
-------
c : float
\end{lstlisting}

As show above, F2PY generates documentation both for the generated module as well as for the individual functions.

Already now it is clear that using F2PY is significantly easier that hand-coding Python wrappers for Fortran. F2PY takes care of all the steps. 

%---------------------------------------------------------------------
\subsection{Passing arrays}
%---------------------------------------------------------------------

F2PY will automatically handle conversion of NumPy arrays when calling a Fortran extension module. However, it is important to note that NumPy by default uses C ordered arrays. These will be automatically converted to Fortran ordered arrays. For smaller arrays the overhead is not so large, but for large arrays the overhead can be significant. To avoid the automatic conversion, NumPy arrays should be created with the \pyvar{order='F'} option in the array constructor, as shown in the following example:

\pymode

\begin{lstlisting}
A = ones((10,10), 'f', order='F')
\end{lstlisting}

Using this option will pass the allocated memory for the NumPy array directly to the Fortran routine without conversion.

%---------------------------------------------------------------------
\subsection{A more complete example - Matrix multiplication}
%---------------------------------------------------------------------

To illustrate the use of arrays in a Fortran extension module we create a Fortran subroutine that takes two input arrays and returns the matrix multiplication of these two arrays, The first version of the function is shown below:

\fmode

\begin{lstlisting}
! A[r,s] * B[s,t] = C[r,t]
subroutine matrix_multiply(A,r,s,B,t,C)
	integer :: r, s, t
	real, intent(in) :: A(r,s)
	real, intent(in) :: B(s,t)
	real, intent(out) :: C(r,t)

	C = matmul(A,B)
end subroutine matrix_multiply
\end{lstlisting}

Input variables \fvar{r, s, t} define the sizes of the incoming matrices. We use the Fortran attributes \fvar{intent(in)} and \fvar{intent(out)} to tell F2PY what should be treated as an input variable or an output variable. Creating a Fortran extension module with F2PY on the above routine produces the following corresponding Python routine (from the generated documentation):

\cmdmode

\begin{lstlisting}
c = matrix_multiply(a,b,[r,s,t])

Wrapper for ``matrix_multiply``.

Parameters
----------
a : input rank-2 array('f') with bounds (r,s)
b : input rank-2 array('f') with bounds (s,t)

Other Parameters
----------------
r : input int, optional
    Default: shape(a,0)
s : input int, optional
    Default: shape(a,1)
t : input int, optional
    Default: shape(b,1)

Returns
-------
c : rank-2 array('f') with bounds (r,t)
\end{lstlisting}

\pymode

We can see in the documentation that the syntax of the Python routine is:

\begin{lstlisting}
c = matrix_multiply(a,b,[r,s,t])
\end{lstlisting}

The Fortran output argument, \pyvar{c} is returned on the left side and the input arguments, \pyvar{a, b} are input parameters to the Fortran routine. Please note that the size input parameters will be provided by the generated function and are not required when calling the routine from Python.

The created extension module can be uses from Python as shown in the following code:

\begin{lstlisting}
from numpy import *
from fortmod import *

A = ones((6,6), 'f', order='F') * 10.0
B = ones((6,6), 'f', order='F') * 20.0

C = matrix_multiply(A, B)

print C
\end{lstlisting}

Output from the Python code is:

\cmdmode

\begin{lstlisting}
[[ 1200.  1200.  1200.  1200.  1200.  1200.]
 [ 1200.  1200.  1200.  1200.  1200.  1200.]
 [ 1200.  1200.  1200.  1200.  1200.  1200.]
 [ 1200.  1200.  1200.  1200.  1200.  1200.]
 [ 1200.  1200.  1200.  1200.  1200.  1200.]
 [ 1200.  1200.  1200.  1200.  1200.  1200.]]
\end{lstlisting}

Output variables, \pyvar{c}, from Fortran will be automatically created. It is not possible to reference data in an already existing \pyvar{c} array as shown in the following example:

\pymode

\begin{lstlisting}
A = ones((6,6), 'f', order='F') * 10.0
B = ones((6,6), 'f', order='F') * 20.0
C = zeros((6,6), 'f', order='F')

print "id of C before multiply =",id(C)

C = matrix_multiply(A, B)

print "id of C after multiply =",id(C)
\end{lstlisting}

In this example, an array \pyvar{C} is created before the call to our Fortran routine. The id or memory location is queried using the \pymethod{id()} and displayed before and after the call. The output is:

\cmdmode

\begin{lstlisting}
id of C before multiply = 4299985824
id of C after multiply = 4340070160
\end{lstlisting}

The \pyvar{C} array is apparently overwritten. This is due to how the Python language is designed. An euqality operator will replace the reference to the first \pyvar{C} instance with a new instance. The next section covers how to pass variables that can be modified by Fortran.

%---------------------------------------------------------------------
\subsection{Matrix mulitplication with modifiable output variables}
%---------------------------------------------------------------------

If the Fortran extension module should be able to modify the contents of the incoming arrays, the \fvar{intent(inout)} attribute must be used. This tells F2PY to generate code that handles this. Our modified matrix multiplication subroutine then becomes:

\fmode

\begin{lstlisting}
! A[r,s] * B[s,t] = C[r,t]
subroutine matrix_multiply2(A,r,s,B,t,C)
	integer :: r, s, t
	real, intent(in) :: A(r,s)
	real, intent(in) :: B(s,t)
	real, intent(inout) :: C(r,t)

	C = matmul(A,B)
end subroutine matrix_multiply2
\end{lstlisting}

The only difference is the \fvar{intent(inout)} attribute on the \fvar{C} array declaration. However, the generated Python routine is quite different:

\cmdmode

\begin{lstlisting}
matrix_multiply2(a,b,c,[r,s,t])

Wrapper for ``matrix_multiply2``.

Parameters
----------
a : input rank-2 array('f') with bounds (r,s)
b : input rank-2 array('f') with bounds (s,t)
c : in/output rank-2 array('f') with bounds (r,t)

Other Parameters
----------------
r : input int, optional
    Default: shape(a,0)
s : input int, optional
    Default: shape(a,1)
t : input int, optional
    Default: shape(b,1)
\end{lstlisting}

Now all input parameters are given on the right side. Now it is possible to directly modify the \pyvar{c} variable in the Fortran code and pass any changes back to Python, without copying the data. The memory address of the array is the same as used by the NumPy array in the Python code. The following code shows how to use the modified Fortran extension:

\fmode

\begin{lstlisting}
A = ones((6,6), 'f', order='F') * 10.0
B = ones((6,6), 'f', order='F') * 20.0
C = zeros((6,6), 'f', order='F')

print "id of C before multiply =",id(C)

matrix_multiply2(A, B, C)

print "id of C after multiply =",id(C)

print C
\end{lstlisting}

For this code to work it is now required to create the array, \pyvar{C}, before calling the Fortran extension. This is due to the fact that the memory area for the array needs to exist before the call as the pointer to the array is passed directly to the Fortran code. The output of the Python code is shown below:

\cmdmode

\begin{lstlisting}
id of C before multiply = 4302082976
id of C after multiply = 4302082976
[[ 1200.  1200.  1200.  1200.  1200.  1200.]
 [ 1200.  1200.  1200.  1200.  1200.  1200.]
 [ 1200.  1200.  1200.  1200.  1200.  1200.]
 [ 1200.  1200.  1200.  1200.  1200.  1200.]
 [ 1200.  1200.  1200.  1200.  1200.  1200.]
 [ 1200.  1200.  1200.  1200.  1200.  1200.]]
\end{lstlisting}

From the output, we can see that the memory of the array is the same before and after the call to the Fortran extension module.


























