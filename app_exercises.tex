%---------------------------------------------------------------------
%---------------------------------------------------------------------
\chapter{Exercises} \label{app:exercises}
%---------------------------------------------------------------------
%---------------------------------------------------------------------

\newcommand{\excn}[1]{\textsf{\textbf{#1}}}

%---------------------------------------------------------------------
\section{Fortran}
%---------------------------------------------------------------------

\fmode

%---------------------------------------------------------------------

\begin{tabular}{lp{0.8\textwidth}}
\excn{1-1} & Which of the following names can be used as Fortran variable names? \\
& \\
& a) \fvar{number\_of\_stars} \\
& b) \fvar{fortran\_is\_a\_nice\_language\_to\_use}\\
& c) \fvar{2001\_a\_space\_odyssey} \\
& d) \fvar{more\$\_money}\\
& \\
\excn{1-2} & Declare the following variables in Fortran: 2 scalar integers
\fvar{a}, and \fvar{b}, 3 floating point scalars \fvar{c}, \fvar{d}
and \fvar{e}, 2 character strings \fvar{infile} and \fvar{outfile}
and a logical variable \fvar{f}.\\
& \\
\excn{1-3} & Declare a floating point variable \textit{a} that can
represent values between 10$^{-150}$ and 10$^{150}$ with 14 significatn numbers. \\
& \\
\end{tabular}

%---------------------------------------------------------------------

\begin{tabular}{lp{0.8\textwidth}}
\excn{1-4} & What is printed by the following program?\\
&
\begin{minipage}{0.8\textwidth}
\begin{lstlisting}
program precision

      implicit none

      integer, parameter :: ap = &
        selected_real_kind(15,300)

      real(ap) :: a, b

      a = 1.234567890123456
      b = 1.234567890123456_ap

      if (a==b) then
           write(*,*) 'Values are equal.'
      else
           write(*,*) 'Values are different.'
      endif

      stop

end program precision
\end{lstlisting}
\end{minipage} \\
& \\
\excn{1-5} & Declare a $[3 \times 3]$ floating point array
,\fvar{Ke}, and an 3 element integer array, \fvar{f}\\
& \\
\excn{1-6} & Declare an integer array, \fvar{idx}, with the following indices \\
&\fvar{[0, 1, 2, 3, 4, 5, 6, 7]}.\\
& \\
\excn{1-7} & Give the following assignments:\\
& Floating point array, \fvar{A}, is assigned the value 5.0 at
(2,3).\\
& Integer matrix, \fvar{C}, is assigned the value 0 at row 2.\\
& \\
\excn{1-8} & Give the following if-statements:\\
& \\
& If the value of the variable,\fvar{i}, is greater than 100
print 'i is greater than 100!'\\
& \\
& If the value of the logical variable, \fvar{extra\_filling},
is true print 'Extra filling is ordered.', otherwise print
'No extra filling.'.\\
& \\
\excn{1-9} & Give a case-statment for the variable, \textit{a},
printing 'a is 1' when a is 1, 'a is between 1 and 20' for
values between 1 and 20 and prints 'a is not between 1 and 20'
for all other values. \\
\end{tabular}

%---------------------------------------------------------------------

\begin{tabular}{lp{0.8\textwidth}}
\excn{1-10} & Write a program consisting of a do-statement 1 to 20 with the control variable, \fvar{i}.
For values, \fvar{i}, between 1 till 5, the value of \fvar{i} is printed, otherwise 'i>5' is printed.
The loop is to be terminated when \fvar{i} equals 15.\\
& \\
\excn{1-11} & Write a program declaring a floating point matrix, \fvar{I}, with the dimensions
$[10 \times 10]$ and initialises it with the identity matrix. \\
\excn{1-12} & Give the following expressions in Fortran: \\
& \\
& a) $\frac{1}{\sqrt{2}}$ \\
& b) $e^{x} \sin ^{2} x$ \\
& c) $\sqrt{a^{2} +b^{2}}$ \\
& d) $\left| x-y\right|$ \\
& \\
\excn{1-13} & Give the following matrix and vector expressions in Fortran.
Also give appropriate array declarations: \\
& \\
& a) $\mathbf{AB}$ \\
& b) $\mathbf{A^{T} A}$ \\
& c) $\mathbf{ABC}$ \\
& d) $\mathbf{a\cdot b}$ \\
& \\
\excn{1-14} & Show expressions in Fortran calculating
maximum, mininmum, sum and product of the elements of an array.\\
& \\
\excn{1-15} & Declare an allocatable 2-dimensional floating point array and
a 1-dimensional floating point vector. Also show program-statements how memory
for these variables are allocated and deallocated.\\
& \\
\excn{1-16} & Create a subroutine, \fname{identity}, initialising a {\underline {arbitrary}}
two-dimensionl to the identity matrix. Write a program illustrating the use of the subroutine.\\
& \\
\excn{1-17} & Implement a function returning the value of the the following expression: \\
& \\
& $e^{x} \sin ^{2} x$ \\
\end{tabular}

%---------------------------------------------------------------------

\begin{tabular}{lp{0.8\textwidth}}
\excn{1-18} & Write a program listing \\
& $f(x)=\sin x$ from $-1.0$ to $1.0$ in intervals of $0.1$. The output from the program
should have the following format:\\
&
\begin{minipage}{0.8\textwidth}
\begin{lstlisting}
         111111111122222222223
123456789012345678901234567890
 x      f(x)
-1.000 -0.841
-0.900 -0.783
-0.800 -0.717
-0.700 -0.644
-0.600 -0.565
-0.500 -0.479
-0.400 -0.389
-0.300 -0.296
-0.200 -0.199
-0.100 -0.100
 0.000  0.000
 0.100  0.100
 0.200  0.199
 0.300  0.296
 0.400  0.389
 0.500  0.479
 0.600  0.565
 0.700  0.644
 0.800  0.717
 0.900  0.783
 1.000  0.841
\end{lstlisting}
\end{minipage} \\
& \\
\excn{1-19} & Write a program calculating the total length of a piecewise linear curve. The curve is defined
in a textfile \ffname{line.dat}. \\
& \\
& The file has the following structure:\\
& \\
&\{number of points n in the file\}\\
&\{x-coordinate point 1\} \{y-coordinate point 1\}\\
&\{x-coordinate point 2\} \{y-coordinate point 2\}\\
&.\\
&.\\
&\{x-coordinate point n\} \{y-coordinate point n\}\\
& \\
&The program must not contain any limitations regarding the number of points in
the number of points in the curve read from the file.\\
\end{tabular}

%---------------------------------------------------------------------

\begin{tabular}{lp{0.8\textwidth}}
\excn{1-20} & Declare 3 strings, \fvar{c1}, \fvar{c2} and
\fvar{c3} containing the words 'Fortran', 'is' och 'fun'. Merge these into a
new string, \fvar{c4}, making a complete sentence.\\
& \\
\excn{1-21} & Write a function converting a string into a floating point value.
Write a program illustrating the use of the function.\\
& \\
\excn{1-22} & Create a module, \fmodule{conversions}, containing the function in 1-21
and a function for converting a string to an integer value. Change the program in 1-21 to
use this module. The module is placed in a separate file, \ffname{conversions.f90} and
the main program in \ffname{main.f90}. \\
\end{tabular}

%%---------------------------------------------------------------------
%\section{Object Pascal}
%%---------------------------------------------------------------------
%
%\pmode
%
%\begin{tabular}{lp{0.8\textwidth}}
%\excn{2-1} & Vilka av följande beteckningar kan användas som variabelnamn
%i Object Pascal? \\
%& \\
%& a) \fvar{number\_of\_stars} \\
%& b) \fvar{pascal\_is\_a\_nice\_language\_to\_use}\\
%& c) \fvar{2001\_a\_space\_odyssey} \\
%& d) \fvar{more\$\_money}\\
%& \\
%\excn{2-2} & Deklarera följande variabler i Object Pascal: två
%heltalsskalärer \pvar{a} och \pvar{b}, tre flyttalsskalärer
%\pvar{c}, \pvar{d} och \pvar{e}, två
%teckensträngar \pvar{infile} och \pvar{outfile} och en logisk variabel \pvar{f}.\\
%& \\
%\excn{2-3} & Ange följande if-satser:\\
%& \\
%& Om värdet på variabeln i är större än 100 skriv ut
%'i större än hundra!'\\
%& \\
%& Om värdet på den logiska variabeln extra\_fyllning är sann
%skrivs 'Extra fyllning beställd.', annars skrivs 'Ingen extra
%fyllning.'.\\
%& \\
%\excn{2-4} & Ange en case-sats för variabeln \pvar{a} som skriver
%ut 'a är 1' för värdet 1, 'a är mellan 2 och 20' för värden mellan
%2 och 20, och skriver ut 'a är inte mellan 1 och 20' för alla
%andra tal.\\
%& \\
%\excn{2-5} & Ange en slinga 1 till 20 med styrvariabeln \pvar{i}.
%För \pvar{i} mellan 1 till 5 skrivs värdet på \pvar{i} ut på
%skärmen, annars skrivs
%'i>5' ut.\\
%& \\
%\excn{2-6} & Skriv ett program som deklarerar en flyttalsmatris
%\pvar{I} med storleken $[10 \times 10]$ och sedan initialiserar denna med enhetsmatrisen.\\
%& \\
%\excn{2-7} & Ange följande uttryck i Object Pascal: \\
%& \\
%& a) $\frac{1}{\sqrt{2}}$ \\
%& b) $e^{x} \sin ^{2} x$ \\
%& c) $\sqrt{a^{2} +b^{2}}$ \\
%& d) $\left| x-y\right|$ \\
%& \\
%\end{tabular}
%
%%---------------------------------------------------------------------
%
%\begin{tabular}{lp{0.8\textwidth}}
%\excn{2-8} & Deklarera en dynamisk flyttalsmatris och en dynamisk
%flyttalsvektor. Visa med programsatser hur minne allokeras och
%frigörs för dessa variabler\\
%& \\
%\excn{2-9} & Skapa en subrutin identity som initialiserar en
%godtycklig matris till enhetsmatrisen. Visa genom att skriva ett
%huvudprogram
%hur subrutinen anropas.\\
%& \\
%\excn{2-10} & Skapa en funktion som returnerar värdet av uttrycket \\
%& \\
%& $e^{x} \sin ^{2} x$ \\
%& \\
%& Skriv också ett program som skriver ut funktionsvärdet när $x =
%1.0$.\\
%& \\
%\excn{2-11} & Skriv ett program som listar \\
%& $f(x)=\sin x$ från --1.0 till 1.0 i steg om 0.1. Utskriften från programmet skall ha följande utseende:\\
%&
%\begin{minipage}{0.8\textwidth}
%\begin{lstlisting}
%         111111111122222222223
%123456789012345678901234567890
% x      f(x)
%-1.000 -0.841
%-0.900 -0.783
%-0.800 -0.717
%-0.700 -0.644
%-0.600 -0.565
%-0.500 -0.479
%-0.400 -0.389
%-0.300 -0.296
%-0.200 -0.199
%-0.100 -0.100
% 0.000  0.000
% 0.100  0.100
% 0.200  0.199
% 0.300  0.296
% 0.400  0.389
% 0.500  0.479
% 0.600  0.565
% 0.700  0.644
% 0.800  0.717
% 0.900  0.783
% 1.000  0.841
%\end{lstlisting}
%\end{minipage} \\
%\end{tabular}
%
%%---------------------------------------------------------------------
%
%\begin{tabular}{lp{0.8\textwidth}}
%\excn{2-12} & Deklarera tre strängar \pvar{c1}, \pvar{c2} och \pvar{c3} som innhåller orden
%'Object Pascal', 'är' och 'kul'. Slå ihop dessa till en ny sträng
%\pvar{c4}, så att de bildar en komplett mening.\\
%& \\
%\excn{2-13} & Skriv en enhet/unit med funktionerna: \pmethod{identity}
%för att initialisera en matris till enhetsmatrisen, \pmethod{zero}
%för att nollställa en matris och \pmethod{transpose} för att
%transponera en matris. Funktionerna skall använda
%\ptype{variant}-matriser och kunna hantera godtyckligt stora
%\ptype{variant}-matriser. Skriv
%också ett huvudprogram som använder rutinerna i enheten.\\
%\end{tabular}
