%---------------------------------------------------------------------
%---------------------------------------------------------------------
\chapter{L�sningar till �vningsuppgifter} \label{app:solutions}
%---------------------------------------------------------------------
%---------------------------------------------------------------------

\renewcommand{\excn}[1]{\textsf{\textbf{#1}}}

%---------------------------------------------------------------------
\section{Fortran}
%---------------------------------------------------------------------

\fmode

%---------------------------------------------------------------------

\begin{tabular}[t]{lp{0.8\textwidth}}
\excn{1-1} & Endast a) �r en korrekt Fortran variabel. b) innhe�ller
mer �n 31 tecken. c) b�rjar ej med en bokstav. d) inneh�ller ett
icke-alfanumeriskt tecken. \\
& \\
\excn{1-2} & \vspace{-2em}
\begin{lstlisting}
integer :: a, b
real(8) :: c, d, e
character(255) :: infile
character(255) :: outfile
logical :: f
\end{lstlisting}\\
& \\
\excn{1-3} & \vspace{-2em}
\begin{lstlisting}
integer, parameter :: ap = &
    selected_real_kind(14,150)

real(ap) :: a
\end{lstlisting} \\
& \\
\excn{1-4} & ''Talen �r olika.'' kommer att skrivas ut. Om inget
annat anges representeras skal�rkonstanter med 7 signifikanta
siffror och kan ta v�rden i omr�det $10^{-38}$ till $10^{38}$.
Skal�rkonstanten som tilldelas a i exemplet kommer ej att kunna
representeras med mer �n 7 signifikanta siffror. Skal�ren som
tilldelas b �r deklarerad med 15 signifikanta siffror med suffixet
\ftype{ap} och kommer att kunna representera hela det angivna
talet. Vid j�mf�relsen kommer variablerna \fvar{a} och \fvar{b}
att inneh�lla olika v�rden, trots att \fvar{a} och \fvar{b} �r deklarerade p� samma s�tt. \\
& \\
\end{tabular}

%---------------------------------------------------------------------

\begin{tabular}[t]{lp{0.8\textwidth}}
\excn{1-5} & \vspace{-2em}
\begin{lstlisting}
integer, parameter :: ap = &
    selected_real_kind(15,300)

real(ap) :: Ke(3,3)
real(ap) :: f(3)
\end{lstlisting} \\
& \\
\excn{1-6} & \vspace{-2em}
\begin{lstlisting}
integer :: idx(0:7)
\end{lstlisting} \\
& \\
\excn{1-7} & \vspace{-2em}
\begin{lstlisting}
1-7 A(2,3) = 5.0
B = 0.0
C(2,:) = 0
\end{lstlisting} \\
& \\
\excn{1-8} & \vspace{-2em}
\begin{lstlisting}
if (i>100) then
    write(*,*) 'i storre �n hundra!'
end if

if (extra_fyllning) then
    write(*,*) 'Extra fyllning best�lld.'
else
    write(*,*) 'Ingen extra fyllning.'
end if
\end{lstlisting} \\
& \\
\excn{1-9} & \vspace{-2em}
\begin{lstlisting}
select case (a)
    case 1
        write(*,*) 'a �r 1.'
    case 2:20
        write(*,*) 'a �r mellan 2 och 20.'
    case default
        write(*,*) 'a �r inte mellan 1 och 20.'
end select
\end{lstlisting} \\
& \\
\end{tabular}

%---------------------------------------------------------------------

\begin{tabular}[t]{lp{0.8\textwidth}}
\excn{1-10} & \vspace{-2em}
\begin{lstlisting}
do i=1,20
    if (i<6) then
        write(*,*) i
    else
        write(*,*) 'i>5'
    end if

    if (i==15) then
        break
    end if
end do
\end{lstlisting} \\
& \\
\excn{1-11} & \vspace{-2em}
\begin{lstlisting}
integer, parameter :: ap = &
    selected_real_kind(15,300)

real(ap) :: I(10,10)
integer :: row, col

do row=1,10
    do col=1,10
        if (row==col) then
            I(row,col) = 1.0
        else
            I(row,col) = 0.0
        end if
    end do
end do
\end{lstlisting} \\
& \\
\excn{1-12} & a) \fsol{1/sqrt(2.0\_ap)} \\
& b) \fsol{exp(x)*sin(x)**2} \\
& c) \fsol{sqrt(a**2 + b**2)} \\
& d) \fsol{abs(x-y))} \\
& \\
\excn{1-13} & a) \fsol{matmul(A,B)  } \\
& b) \fsol{matmul(transpose(A),A)} \\
& c) \fsol{matmul(A,matmul(B,C))} \\
& d) \fsol{dot\_product(a,b)} \\
& \\
\excn{1-14} & \vspace{-2em}
\begin{lstlisting}
b = maxval(A)
c = minval(A)
d = product(A)
e = sum(A)
\end{lstlisting} \\
\end{tabular}

%---------------------------------------------------------------------

\begin{tabular}[t]{lp{0.8\textwidth}}
\excn{1-15} & \vspace{-2em}
\begin{lstlisting}
integer, parameter :: ap = &
    selected_real_kind(15,300)

real(ap), allocatable :: A(:,:)
real(ap), allocatable :: b(:)

allocate(A(20,20))
allocate(b(30))

A = 0.0_ap
b = 0.0_ap

deallocate(A)
deallocate(b)
\end{lstlisting} \\
& \\
\excn{1-16} & \vspace{-2em}
\begin{lstlisting}
program subtest

    integer, parameter :: ap = ...

    real(ap) :: M(20,20)
    call identity(M,20)
    stop

contains

subroutine identity(A,n)

    implicit none
    integer, parameter :: ap = ...

    real(ap) :: A(n,*)
    integer :: row, col

    do row=1,n
        do col=1,n
            if (row==col) then
                A(row,col) = 1.0
            else
                A(row,col) = 0.0
            end if
        end do
    end do

    return

end subroutine identity
\end{lstlisting}\\
\end{tabular}

%---------------------------------------------------------------------

\begin{tabular}[t]{lp{0.8\textwidth}}
\excn{1-17} & \vspace{-2em}
\begin{lstlisting}
real function f(x)

    implicit none

    real(8) :: x

    f = exp(x)*sin(x)**2

    return

end function f
\end{lstlisting} \\
& \\
\excn{1-18} & \vspace{-2em}
\begin{lstlisting}
program exercise1_18

    implicit none
    integer, parameter :: ap = &
        selected_real_kind(15,300)

    real(ap) :: x
    real(ap) :: f

    write(*,'(T5,A,T21,A)') 'x','f(x)'

    x = -1.0_ap
    do while (x<1.05)
        f = sin(x)
        write(*,'(T4,F6.3,T20,F6.3)') x, f
        x = x + 0.1_ap
    end do

    stop

end program exercise1_18
\end{lstlisting} \\
\end{tabular}

%---------------------------------------------------------------------

\begin{tabular}[t]{lp{0.8\textwidth}}
\excn{1-19} & \vspace{-2em}
\begin{lstlisting}
program exercise1_19

    implicit none

    integer, parameter :: ap = &
        selected_real_kind(15,300)
    integer, parameter :: linjeFil = 15
    integer :: nPoints
    integer :: i
    real(ap) :: x, y, lastX, lastY
    real(ap) :: summedLength

    open(unit=linjeFil, file='linje.dat', &
        access='sequential', &
        action='read', status='old')

    read(linjeFil,*) nPoints

    summedLength = 0.0_ap

    do i=1,nPoints
        read(linjeFil,*) x, y

        if (i>1) then
            summedLength = summedLength + &
                   sqrt((x-lastX)**2+(y-lastY)**2)
        end if


        lastX = x
        lastY = y
    end do

    close(linjeFil)

    write(*,'(T1,A,G15.3)') 'Total langd = ', &
        summedLength

end program exercise1_19
\end{lstlisting}\\
\end{tabular}

%---------------------------------------------------------------------

\begin{tabular}[t]{lp{0.8\textwidth}}
\excn{1-20} & \vspace{-2em}
\begin{lstlisting}
program exercise1_20

    implicit none

    character(7) :: c1
    character(2) :: c2
    character(3) :: c3
    character(15) :: c4

    c1 = 'Fortran'
    c2 = '�r'
    c3 = 'kul'

    c4 = c1//' '//c2//' '//c3

    write(*,*) c4

    stop

end program exercise1_20
\end{lstlisting} \\
& \\
\excn{1-21} & \vspace{-2em}
\begin{lstlisting}
program excercise1_21

    implicit none
    character(20) :: c
    real(8) :: value

    c = '2.0'
    value = toReal(c)
    write(*,*) value

    stop

contains

real(8) function toReal(c)

    implicit none
    character(20), intent(in) :: c

    read(c, *) toReal
    return

end function toReal

end program excercise1_21
\end{lstlisting} \\
\end{tabular}

%---------------------------------------------------------------------

\begin{tabular}[t]{lp{0.8\textwidth}}
\excn{1-22} & \vspace{-2em}
\begin{lstlisting}
module conversions

    implicit none

contains

real(8) function toReal(c)

    implicit none
    character(20), intent(in) :: c
    read(c, *) toReal
    return

end function toReal

integer function toInteger(c)

    implicit none
    character(20), intent(in) :: c
    read(c, *) toInteger
    return

end function toInteger

end module conversions

! ------------------------------

program exercise2_22

    use conversions

    character(20) :: c
    character(20) :: d
    real(8) :: floatValue
    integer :: intValue

    c = '2.0'
    d = '42'

    floatValue = toReal(c)
    intValue = toInteger(d)

    write(*,*) floatValue
    write(*,*) intValue

    stop

end program exercise2_22
\end{lstlisting} \\
\end{tabular}

%%---------------------------------------------------------------------
%\section{Object Pascal}
%%---------------------------------------------------------------------
%
%\pmode
%
%\begin{tabular}[t]{lp{0.8\textwidth}}
%\excn{2-1} & 2-1 a) och b) �r korrekta namn p� variabler i Object
%Pascal. b) b�rjar med en siffra. c) inneh�ller ett icke
%alfanumeriskt tecken. \\
%& \\
%\excn{2-2} & \vspace{-2em}
%\begin{lstlisting}
%a, b : integer;
%c, d, e : double;
%infile, outfile : string;
%boolean : f;
%\end{lstlisting} \\
%& \\
%\excn{2-3} & \vspace{-2em}
%\begin{lstlisting}
%if i>100 then
%    writeln('i st�rre �n hundra!');
%
%if extra_fyllning then
%    writeln('Extra fyllning best�lld.')
%else
%    writeln('Ingen extra fyllning.');
%\end{lstlisting} \\
%& \\
%\excn{2-4} & \vspace{-2em}
%\begin{lstlisting}
%case a of
%    1     : writeln('a �r 1');
%    2..20 : writeln('a �r mellan 2-20.');
%else
%    writeln('a �r inte mellan 1 och 20.');
%end;
%\end{lstlisting} \\
%& \\
%\excn{2-5} & \vspace{-2em}
%\begin{lstlisting}
%for i:=1 to 20 do
%begin
%    if i>5 then
%        writeln('i>5')
%    else
%        writeln(i);
%end;
%\end{lstlisting} \\
%\end{tabular}
%
%%---------------------------------------------------------------------
%
%\begin{tabular}[t]{lp{0.8\textwidth}}
%\excn{2-6} & \vspace{-2em}
%\begin{lstlisting}
%program exercise2_6b;
%{$APPTYPE CONSOLE}
%uses
%    SysUtils;
%
%var
%    I : array [1..10, 1..10] of double;
%
%    row, col : integer;
%
%begin
%    for row:=1 to 10 do
%        for col:=1 to 10 do
%            if row=col then
%                I[row,col]:=1.0
%            else
%                I[row,col]:=0.0;
%end.
%\end{lstlisting} \\
%& \\
%\excn{2-6} & \vspace{-2em}
%\begin{lstlisting}
%program exercise2_6b;
%{$APPTYPE CONSOLE}
%uses
%    SysUtils;
%
%var
%    I : array [1..10, 1..10] of double;
%
%    row, col : integer;
%
%begin
%    for row:=1 to 10 do
%        for col:=1 to 10 do
%            if row=col then
%                I[row,col]:=1.0
%            else
%                I[row,col]:=0.0;
%end.
%\end{lstlisting} \\
%& \\
%\excn{2-7} & a) \psol{1/Sqrt(2)} \\
%& b) \psol{Exp(x)*Power(sin(x),2)} \\
%& c) \psol{Sqrt(Power(a,2)+Power(b,2))} \\
%& d) \psol{Abs(x-y)} \\
%\end{tabular}
%
%%---------------------------------------------------------------------
%
%\begin{tabular}[t]{lp{0.8\textwidth}}
%\excn{2-8} & \vspace{-2em}
%\begin{lstlisting}
%program exercise2_8;
%{$APPTYPE CONSOLE}
%uses
%    SysUtils;
%
%var
%    K : variant;
%    f : variant;
%
%begin
%    K:=VarArrayCreate([1,20,1,20], varDouble);
%    f:=VarArrayCreate([1,20], varDouble);
%
%    // Use matrices
%
%    VarClear(K);
%    VarClear(f);
%end.
%\end{lstlisting} \\
%& \\
%\excn{2-9} & \vspace{-2em}
%\begin{lstlisting}
%program exercise2_9;
%{$APPTYPE CONSOLE}
%uses
%  SysUtils;
%
%procedure Identity(var A : variant; n : integer);
%var
%    i, j : integer;
%begin
%    for i:=1 to n do
%        for j:=1 to n do
%            if i=j then
%                A[i,j]:=1.0
%            else
%                A[i,j]:=0.0;
%end;
%
%procedure Print(var A : variant; n : integer);
%var
%        i, j : integer;
%begin
%    for i:=1 to n do
%        for j:=1 to n do
%            write(A[i,j],', ');
%        writeln;
%    end;
%end;
%\end{lstlisting} \\
%\end{tabular}
%
%%---------------------------------------------------------------------
%
%\begin{tabular}[t]{lp{0.8\textwidth}}
%\excn{2-9} ... & \vspace{-2em}
%\begin{lstlisting}
%var
%    B : variant;
%
%begin
%    B:=VarArrayCreate([1,10,1,10],varDouble);
%    Identity(B,10);
%    Print(B,10);
%    VarClear(B);
%end.
%\end{lstlisting} \\
%& \\
%\excn{2-10} & \vspace{-2em}
%\begin{lstlisting}
%program exercise2_10;
%{$APPTYPE CONSOLE}
%uses
%    SysUtils, Math;
%
%function f(x : double) : double;
%begin
%  Result:=Exp(x)*Power(sin(x),2);
%end;
%
%begin
%    writeln(f(1.0));
%end.
%\end{lstlisting} \\
%& \\
%\excn{2-11} & \vspace{-2em}
%\begin{lstlisting}
%program exercise2_11;
%{$APPTYPE CONSOLE}
%uses
%  SysUtils,
%  Math;
%
%function f(x : double) : double;
%begin
%    Result:=sin(x);
%end;
%
%var
%    x : double;
%    dx : double;
%
%begin
%    writeln('    x              f(x)');
%\end{lstlisting} \\
%& \\
%\end{tabular}
%
%%---------------------------------------------------------------------
%
%\begin{tabular}[t]{lp{0.8\textwidth}}
%\excn{2-11} ... & \vspace{-2em}
%\begin{lstlisting}
%    DecimalSeparator:='.';
%    x:=-1.0;
%    dx:=0.1;
%    while x<1.05 do
%    begin
%        writeln(format('    %6.3f %6.3f',
%            [x, sin(x)]));
%        x:=x+dx;
%    end;
%    readln;
%end.
%\end{lstlisting} \\
%& \\
%\excn{2-12} & \vspace{-2em}
%\begin{lstlisting}[escapechar=\%]
%program exercise2_12;
%{$APPTYPE CONSOLE}
%uses
%    SysUtils, Math;
%var
%    f : TextFile;
%    nPoints, i : integer;
%    x, y, LastX, LastY : double;
%    SummedLength : double;
%begin
%
%    AssignFile(f, 'linje.dat');
%    Reset(f);
%    readln(f,nPoints);
%    SummedLength := 0.0;
%
%    for i:=1 to nPoints do
%    begin
%        readln(f, x, y);
%        if i>1 then
%        begin
%            SummedLength:=SummedLength+
%                Sqrt(Power(x-LastX,2)+Power(y-LastY,2));
%        end;
%
%        LastX:=x;
%        LastY:=y;
%    end;
%
%    CloseFile(f);
%
%    writeln('Total l%�%ngd', SummedLength);
%    readln;
%end.
%\end{lstlisting} \\
%\end{tabular}
%
%%---------------------------------------------------------------------
%
%\begin{tabular}[t]{lp{0.8\textwidth}}
%\excn{2-13} & \vspace{-2em}
%\begin{lstlisting}
%program exercise2_13;
%{$APPTYPE CONSOLE}
%uses
%  SysUtils;
%
%var
%    c1, c2, c3, c4 : string;
%
%begin
%    c1:='Object Pascal';
%    c2:='�r';
%    c3:='kul';
%    c4:=c1+' '+c2+' '+c3+'.';
%    writeln(c4);
%    readln;
%end.
%\end{lstlisting} \\
%& \\
%\excn{2-13} & \vspace{-2em}
%\begin{lstlisting}
%unit Matrix;
%
%interface
%
%procedure Identity(var A : variant; n : integer);
%procedure Zero(var A : variant; n : integer);
%procedure Transpose(var A : variant; n : integer);
%
%implementation
%
%procedure Identity(var A : variant; n : integer);
%var
%    i, j : integer;
%begin
%    for i:=1 to n do
%        for j:=1 to n do
%            if i=j then
%                A[i,j]:=1.0
%            else
%                A[i,j]:=0.0;
%end;
%
%procedure Zero(var A : variant; n : integer);
%var
%    i, j : integer;
%begin
%    for i:=1 to n do
%        for j:=1 to n do
%            A[i,j]:=0.0;
%end;
%\end{lstlisting} \\
%\end{tabular}
%
%%---------------------------------------------------------------------
%
%\begin{tabular}[t]{lp{0.8\textwidth}}
%\excn{2-13} ... & \vspace{-2em}
%\begin{lstlisting}
%procedure Transpose(var A : variant; n : integer);
%var
%    i, j : integer;
%    B : variant;
%begin
%
%    B:=VarArrayCreate([1,n,1,n], varDouble);
%
%    for i:=1 to n do
%        for j:=1 to n do
%            B[i,j]:=A[j,i];
%
%    for i:=1 to n do
%        for j:=1 to n do
%            A[i,j]:=B[i,j];
%end;
%
%end.
%\end{lstlisting} \\
%
%\excn{2-13} ... & \vspace{-2em}
%\begin{lstlisting}[escapechar=\%]
%%Huvudprogram%
%
%program exercise2_14;
%{$APPTYPE CONSOLE}
%uses
%    SysUtils,
%    Matrix in 'Matrix.pas';
%
%var
%    A : variant;
%
%begin
%    A:=VarArrayCreate([1,20,1,20], varDouble);
%    Matrix.Zero(A,20);
%    Matrix.Identity(A,20);
%    Matrix.Transpose(A,20);
%    VarClear(A);
%end.
%\end{lstlisting} \\
%& \\
%\end{tabular}
