\documentclass[b5paper,twoside,11pt]{book}
%\documentclass[a4paper,twoside,12pt]{book}
\usepackage{graphicx}
\usepackage{fancyhdr}
\usepackage{listings}
\usepackage[usenames,dvipsnames]{color}
\usepackage{ae}
\usepackage{hyperref}
\usepackage{palatino}
\usepackage{MnSymbol}
\usepackage{pifont}
%\usepackage{bookman}
%\usepackage{times}

\usepackage{tikz}
\usepackage{bbding}
\usepackage{ascii}

\hypersetup{
    bookmarks=true,         % show bookmarks bar?
    unicode=false,          % non-Latin characters in Acrobat’s bookmarks
    pdftoolbar=true,        % show Acrobat’s toolbar?
    pdfmenubar=true,        % show Acrobat’s menu?
    pdffitwindow=true,     % window fit to page when opened
    pdfstartview={FitH},    % fits the width of the page to the window
    pdftitle={Modern Fortran in Science and Technology},    % title
    pdfauthor={Jonas Lindemann and Ola Dahlblom},     % author
    pdfsubject={Programming},   % subject of the document
    pdfcreator={Jonas Lindemann},   % creator of the document
    pdfproducer={Jonas Lindemann}, % producer of the document
    pdfkeywords={Programming} {Fortran} {Multilanguage-programming}, % list of keywords
    pdfnewwindow=true,      % links in new window
    colorlinks=true,       % false: boxed links; true: colored links
    linkcolor=red,          % color of internal links
    citecolor=green,        % color of links to bibliography
    filecolor=magenta,      % color of file links
    urlcolor=cyan           % color of external links
}

% ---------------------------------------------------------
% ---- Layoutinställningar
% ---------------------------------------------------------

%\renewcommand*\rmdefault{ppl}
\renewcommand*\sfdefault{cmss}

\definecolor{light-gray}{gray}{0.95}

% B5 bredd = 182mm hjd = 257mm

\usepackage[inner=20mm,text={130mm,210mm}]{geometry}
%\usepackage[cam,a4,center]{crop}

\pagestyle{fancy}
\fancyhead[LE,RO]{\textsl{\rightmark}}
\fancyhead[LO,RE]{\textsl{\leftmark}}
\fancyfoot[C]{\textsf{\thepage}}
\renewcommand{\chaptermark}[1]{\markboth{#1}{}}
\renewcommand{\sectionmark}[1]{\markright{\thesection\ #1}}
\setlength{\headwidth}{130mm}
%\setlength{\headwidth}{160mm}

\addtolength{\parskip}{3pt}%

\usepackage[english]{babel}
\usepackage[T1]{fontenc}
\usepackage[latin1]{inputenc}
	
\definecolor{mycolor}{rgb}{0.9,0.9,0.9}
\newcommand{\greybox}{\textcolor{mycolor}{\rule{\linewidth}{\baselineskip}}\vspace{-1.25\baselineskip}}

% ---------------------------------------------------------
% ---- Globala inställningar kodstorlek
% ---------------------------------------------------------

\usepackage[scaled=.8]{beramono}
\newcommand{\spcodesize}{\scriptsize}
%\newcommand{\spcodesize}{\footnotesize}
%\newcommand{\spcodesize}{\small}

% ---------------------------------------------------------
% ---- Nyckelord fr Fortran
% ---------------------------------------------------------

\newcommand{\fkeyw}[1]{\textsf{#1}}
\newcommand{\foper}[1]{\textsf{#1}}
\newcommand{\fvar}[1]{\textsf{#1}}
\newcommand{\fvarname}[1]{\textsf{#1}}
\newcommand{\fmodule}[1]{\textsf{#1}}
\newcommand{\fname}[1]{\textsf{#1}}
\newcommand{\ftype}[1]{\textsf{#1}}
\newcommand{\fmethod}[1]{\textsf{#1}}
\newcommand{\fexpr}[1]{\textit{\textsf{#1}}}
\newcommand{\ftab}{\hspace{5mm}}
\newcommand{\ffname}[1]{\texttt{#1}}
\newcommand{\fsol}[1]{\textsf{#1}}

\newcommand\hlbox[2][]{\tikz[overlay]\node[fill=blue!20,inner sep=2pt, anchor=text, rectangle, rounded corners=1mm,#1] {#2};\phantom{#2}}
\newcommand{\hladded}{\scriptsize{\PencilLeftDown~\textsl{Added}}}

% ---------------------------------------------------------
% ---- Syntaxhantering i Fortran
% ---------------------------------------------------------

\newenvironment{fsyntax}
{\vspace{-6mm}\begin{quotation}\begin{sffamily}\begin{slshape}\noindent\flushleft}
{\end{slshape}\end{sffamily}\end{quotation}}

% prebreak=\raisebox{0ex}[0ex][0ex]{\ensuremath{\color{red}\rcurvearrowse\space}}

\newcommand{\fmode}{\lstset{language=[08]Fortran,basicstyle=\sffamily \spcodesize,
    xleftmargin=10mm,xrightmargin=7mm,aboveskip=5mm,belowskip=5mm,
    tabsize=4,showspaces=false,otherkeywords={forall},breaklines=true,
    extendedchars=true,framerule=0pt,frame=single,backgroundcolor=\color{light-gray},
    prebreak=\raisebox{0ex}[0ex][0ex]{\ensuremath{\lcurvearrowse\space}},
    postbreak=\raisebox{0ex}[0ex][0ex]{\ensuremath{\rcurvearrowse\space}},
    escapeinside={<*}{*>}
    }}

\newcommand{\fmodesrc}{\lstset{language=Fortran,basicstyle=\sffamily \small,
tabsize=4,showspaces=false,xleftmargin=0mm,breaklines=true,otherkeywords={forall},resetmargins=true,extendedchars=true}}


% ---------------------------------------------------------
% ---- Nyckelord fr Python
% ---------------------------------------------------------

\newcommand{\pykeyw}[1]{\textsf{#1}}
\newcommand{\pyoper}[1]{\textsf{#1}}
\newcommand{\pyvar}[1]{\textsf{#1}}
\newcommand{\pyvarname}[1]{\textsf{#1}}
\newcommand{\pymodule}[1]{\textsf{#1}}
\newcommand{\pyname}[1]{\textsf{#1}}
\newcommand{\pytype}[1]{\textsf{#1}}
\newcommand{\pymethod}[1]{\textsf{#1}}
\newcommand{\pyexpr}[1]{\textit{\textsf{#1}}}
\newcommand{\pytab}{\hspace{5mm}}
\newcommand{\pyfname}[1]{\texttt{#1}}
\newcommand{\pysol}[1]{\textsf{#1}}

\newenvironment{pysyntax}
{\vspace{-6mm}\begin{quotation}\begin{sffamily}\begin{slshape}\noindent\flushleft}
{\end{slshape}\end{sffamily}\end{quotation}}

\newcommand{\pymode}{\lstset{language=Python,basicstyle=\sffamily \spcodesize,
    xleftmargin=10mm,xrightmargin=7mm,aboveskip=5mm,belowskip=5mm,
    tabsize=4,showspaces=false,breaklines=true,
    extendedchars=true}}
    
% ---------------------------------------------------------
% ---- Nyckelord fr C
% ---------------------------------------------------------

\newcommand{\ckeyw}[1]{\textsf{#1}}
\newcommand{\coper}[1]{\textsf{#1}}
\newcommand{\cvar}[1]{\textsf{#1}}
\newcommand{\cvarname}[1]{\textsf{#1}}
\newcommand{\cmodule}[1]{\textsf{#1}}
\newcommand{\cname}[1]{\textsf{#1}}
\newcommand{\ctype}[1]{\textsf{#1}}
\newcommand{\cmethod}[1]{\textsf{#1}}
\newcommand{\cexpr}[1]{\textit{\textsf{#1}}}
\newcommand{\ctab}{\hspace{5mm}}
\newcommand{\cfname}[1]{\texttt{#1}}
\newcommand{\csol}[1]{\textsf{#1}}

\newcommand{\cmode}{\lstset{language=C,basicstyle=\sffamily \spcodesize,
    xleftmargin=10mm,xrightmargin=7mm,aboveskip=5mm,belowskip=5mm,
    tabsize=4,showspaces=false,breaklines=true,
    extendedchars=true}}
    

% ---------------------------------------------------------
% ---- Nyckelord fr Object Pascal
% ---------------------------------------------------------

\newcommand{\pkeyw}[1]{\textsf{#1}}
\newcommand{\poper}[1]{\textsf{#1}}
\newcommand{\pvar}[1]{\textsf{#1}}
\newcommand{\pvarname}[1]{\textsf{#1}}
\newcommand{\pname}[1]{\textsf{#1}}
\newcommand{\punit}[1]{\textsf{#1}}
\newcommand{\ptype}[1]{\textsf{#1}}
\newcommand{\pprop}[1]{\textsf{#1}}
\newcommand{\pexpr}[1]{\textit{\textsf{#1}}}
\newcommand{\ptab}{\hspace{5mm}}
\newcommand{\pfname}[1]{\texttt{#1}}
\newcommand{\pmethod}[1]{\textsf{#1}}
\newcommand{\pobject}[1]{\textsf{#1}}
\newcommand{\psol}[1]{\textsf{#1}}

% ---------------------------------------------------------
% ---- Syntaxhantering i Object Pascal
% ---------------------------------------------------------

\newenvironment{psyntax}
{\vspace{-6mm}\begin{quotation}\begin{sffamily}\begin{slshape}\noindent\flushleft}
{\end{slshape}\end{sffamily}\end{quotation}}

\newcommand{\pmode}{\lstset{language=Delphi,basicstyle=\sffamily \spcodesize,
    xleftmargin=10mm,xrightmargin=7mm,aboveskip=5mm,belowskip=5mm,
    tabsize=4,showspaces=false,breaklines=true,extendedchars=true}}

\newcommand{\pmodesrc}{\lstset{language=Delphi,basicstyle=\sffamily \footnotesize,
    tabsize=4,showspaces=false,xleftmargin=0mm,breaklines=true,extendedchars=true}}

% ---------------------------------------------------------
% ---- Speciella definitioner fr GUI
% ---------------------------------------------------------

\newcommand{\menuitem}[1]{\textsf{\textbf{#1}}}
\newcommand{\keyb}[1]{\textsf{\textbf{#1}}}
\newcommand{\button}[1]{\textsf{\textbf{#1}}}
\newcommand{\guitab}[1]{\textsf{\textbf{#1}}}
\newcommand{\guicontrol}[1]{\textsf{\textbf{#1}}}

\newcommand{\pbox}{\rule{1.5mm}{1.5mm}}

% ---------------------------------------------------------
% ---- Makefile syntax
% ---------------------------------------------------------

\newenvironment{msyntax}
{\vspace{-6mm}\begin{quotation}\begin{sffamily}\begin{slshape}\noindent\flushleft}
{\end{slshape}\end{sffamily}\end{quotation}}

\newcommand{\mmode}{\lstset{language=make,basicstyle=\sffamily \spcodesize,xleftmargin=10mm,xrightmargin=7mm,aboveskip=5mm,belowskip=5mm,tabsize=4,showspaces=false,showtabs=true,breaklines=true,extendedchars=true,frame=none}}

\newcommand{\mvar}[1]{\textsf{#1}}
\newcommand{\mfname}[1]{\textsf{#1}}
\newcommand{\mfunc}[1]{\textsf{#1}}

% ---------------------------------------------------------
% ---- CMake syntax
% ---------------------------------------------------------

\newenvironment{cmsyntax}
{\vspace{-6mm}\begin{quotation}\begin{sffamily}\begin{slshape}\noindent\flushleft}
{\end{slshape}\end{sffamily}\end{quotation}}

\newcommand{\cmmode}{\lstset{language=Clean,basicstyle=\sffamily \spcodesize,xleftmargin=10mm,xrightmargin=7mm,aboveskip=5mm,belowskip=5mm,tabsize=4,showspaces=false,showtabs=false,breaklines=true,extendedchars=true,frame=none}}

\newcommand{\cmvar}[1]{\textsf{#1}}
\newcommand{\cmfname}[1]{\textsf{#1}}
\newcommand{\cmfunc}[1]{\textsf{#1}}

% ---------------------------------------------------------
% ---- Command line
% ---------------------------------------------------------

\newcommand{\cmdmode}{\lstset{language=Clean,basicstyle=\footnotesize\ttfamily,xleftmargin=10mm,xrightmargin=7mm,aboveskip=5mm,belowskip=5mm,tabsize=4,showspaces=false,showtabs=false,breaklines=true,extendedchars=true,frame=tblr}}

\newcommand{\cli}[1]{\texttt{#1}}

% ---------------------------------------------------------
% ---- Figurdefinitioner
% ---------------------------------------------------------

\newcommand{\fig}[4]{
%\vspace{-2em}
\begin{figure}[!htb]%
\begin{center}%
\includegraphics[width=#1]{./images/#2}%
\end{center}%
\caption{#3} \label{#4}
\end{figure}
}

\newcommand{\figsimple}[1]{\includegraphics{./images/#1}}
\newcommand{\figsimplesize}[2]{\includegraphics[width=#1]{./images/#2}}
\newcommand{\figinline}[1]{\begin{center}\includegraphics[width=0.7\textwidth]{./images/#1}\end{center}}
\newcommand{\figinlinesize}[2]{\begin{center}\includegraphics[width=#1]{./images/#2}\end{center}}
\newcommand{\figbutton}[1]{\includegraphics[width=5mm, height=5mm]{./images/#1}}

\newcommand{\fignc}[3]{
%\begin{figure}[!ht]%
\begin{center}%
\includegraphics[width=#1]{./images/#2}%
\end{center}%
%\end{figure}
}

\newcommand{\fignormal}[3]{
\fig{0.7\textwidth}{#1}{#2}{#3}
}

\newcommand{\fignormalnc}[1]{
\fignc{0.7\textwidth}{#1}{}
}

\newcommand{\figmedium}[3]{
\fig{0.5\textwidth}{#1}{#2}{#3} }

\newcommand{\figmediumnc}[1]{
\fignc{0.5\textwidth}{#1}{}
}

\newcommand{\figsmall}[3]{
\fig{0.4\textwidth}{#1}{#2}{#3} }

\newcommand{\figsmallnc}[1]{
\fignc{0.4\textwidth}{#1}{}
}

\newcommand{\figtiny}[3]{
\fig{0.2\textwidth}{#1}{#2}{#3} }

\newcommand{\figtinync}[1]{
\fignc{0.2\textwidth}{#1}{} }

\newcommand{\figfull}[3]{
\fig{1.0\textwidth}{#1}{#2}{#3}
}

\newcommand{\figfullnc}[1]{
\fignc{1.0\textwidth}{#1}{} }

% ---------------------------------------------------------
% ---- Speciella listor
% ---------------------------------------------------------

\newenvironment{xlist}{\begin{list}{-}{\itemsep -2mm}}{\end{list}}
\newenvironment{xlist2}{\begin{list}{}{\itemsep -2mm}}{\end{list}}

\newcounter{Mycount}

\title{\textbf{Modern Fortran in Science and Technology}}
\author{Jonas Lindemann och Ola Dahlblom}

% ---------------------------------------------------------
% ---------------------------------------------------------
% ---- Huvuddokument
% ---------------------------------------------------------
% ---------------------------------------------------------


\begin{document}

\maketitle

\frontmatter

\tableofcontents%
\listoffigures%
%\listoftables%

\mainmatter

%---------------------------------------------------------------------
%---------------------------------------------------------------------
\chapter{Introduction}
%---------------------------------------------------------------------
%---------------------------------------------------------------------

This book is an introduction in programming with Fortran
95/2003/2008 in science and technology. The book also covers
methods for integrating Fortran code with other programming
languages both dynamic (Python) and compiled languages (C++).
An introduction in using modern development enrvironments such
as QtCreator/Eclipse/Photran, for debugging and development is also
given.


\fmode

%---------------------------------------------------------------------
%---------------------------------------------------------------------
\chapter{The Fortran Language}
%---------------------------------------------------------------------
%---------------------------------------------------------------------

Fortran was the first high-level language and was developed in the fifties. The languages has since the developed through a number of standards Fortran IV (1966), Fortran 77, Fortran 90, Fortran 95 and the latest Fortran 2003. The advantages with standardised languages is that the code can be run on different computer architectures without modification. In every new standard the language has been extended with more modern language elements. To be compatible with previous standards older language elements are not removed. However, language elements that are considered bad or outdated can be removed after 2 standard revisions. As an example Fortran 90 is fully backwards compatible with Fortran 77, but in Fortran 95 some
older language constructs where removed. 

The following sections gives a short introduction to the Fortran 90 language and some of the extensions in Fortran 95. The description is centered on the most important language features. A more thorough description of the language can be found in the book Fortran 95/2003 Explained \cite{metcalf00}

%---------------------------------------------------------------------
\section{Program structure}
%---------------------------------------------------------------------

Every Fortran-program must have a main program routine. From the main routine, other subroutines that make up the program are called. The syntax for a main program is:

\begin{fsyntax}
[\textbf{program} program-name]\newline%
\indent[specification statements]\newline%
\indent[executable statements]\newline%
\indent[contains]\newline%
\indent[subroutines]\newline%
\textbf{end} [\textbf{program} [program-name]]\newline
\end{fsyntax}

From the syntax it can be seen that the only identifier that must be included in main program definition is \fkeyw{end}.

The syntax for a subroutine and functions are defined in the same way, but the \fkeyw{program} identifier is replaced with \fkeyw{subroutine} or \fkeyw{function}. A proper way of organizing subroutines is to place these in separat files or place the in modules (covered in upcoming sections). Subroutines can also be placed in the main program \fkeyw{contains}-section, which is the preferred method if all subroutines are placed in the same source file. The code below shows a simple example of a main program with a subroutine in Fortran.

\lstinputlisting{source/sample1/sample1.f90}

The program source code can contain upper and lower case letters, numbers and special characters. However, it should be noted that Fortran does not differentiate between upper and lower case letters. The program source is written starting from the first position with one statement on each line. If a row is terminated with the charachter \foper{\&}, this indicates that the statement is continued on the the next line. All text placed after the character \foper{!} is a comment and wont affect the function of the program. Even if the comments don't have any function in the program they are important for source code readability. This is especially important for future modification of the program. In addition to the source code
form described above there is also the possibility of writing code in fixed form, as in Fortran 77 and earlier versions. In previous version of the Fortran standard this was the only source code form available.

%---------------------------------------------------------------------
\section{Variables}
%---------------------------------------------------------------------

%---------------------------------------------------------------------
\subsection{Naming of variables}
%---------------------------------------------------------------------

Variables in Fortran 95 consists of 1 to 31 alphanumeric characters (letters except   and , underscore and numbers). The first character of a variable name must be a letter. Allowable variable names can be:

\begin{lstlisting}
a
a_thing
x1
mass
q123
time_of_flight
\end{lstlisting}

Variable names can consist of both upper case and lower case letters. It should be noted that \fvar{a} and \fvar{A} references the same variable. Invalid variables names can be:

\begin{lstlisting}[texcl]
1a      ! First character not a letter
a thing ! Contains a space character
\$sign  ! Contains a non-alphanumeric character
\end{lstlisting}

%---------------------------------------------------------------------
\subsection{Variable types and declarations}
%---------------------------------------------------------------------

There are 5 built-in data types in Fortran:

\begin{xlist}
    \item \ftype{integer}, Integers
    \item \ftype{real}, Floating point numbers
    \item \ftype{complex}, Complex numbers
    \item \ftype{logical}, Boolean values
    \item \ftype{character}, Strings and characters
\end{xlist}

The syntax for a variable declaration is:

\begin{fsyntax}
type [[,attribute]... ::] entity-list
\end{fsyntax}

\fexpr{type} defines the variable type and can be\ftype{integer}, \ftype{real}, \ftype{complex}, \ftype{logical}, \ftype{character}, or \ftype{type}( type-name ). \ftype{attribute} defines additional special attributes or how the variable is to be used. The following examples shows
some typical Fortran variable declarations.

\begin{lstlisting}[texcl]
integer :: a     ! Scalar integer variable
real :: b        ! Scalar floating point variable
logical :: flag  ! boolean variable

real :: D(10)    ! Floating point array consisting of 10 elements
real :: K(20,20) ! Floating point array of 20x20 elements

integer, dimension(10) :: C     ! Integer array of 10 elements

character :: ch                 ! Character
character, dimension(60) :: chv ! Array of characters
character(len=80) :: line       ! Character string
character(len=80) :: lines(60)  ! Array of strings
\end{lstlisting}

Constants are declared by specifying an additional keyword, \fkeyw{parameter}. A declared constant can be used in following variable declarations. An example of use is shown in the following example.

\begin{lstlisting}[texcl]
integer, parameter :: A = 5 ! Integer constant
real :: C(A)                ! Floating point array where
                            ! the number of elements is
                            ! specified by A
\end{lstlisting}

The precision and size of the variable type can be specified by adding a parenthesis directly after the type declaration. The variables \fvar{A} and \fvar{B} in the following example are declared as floating point scalars with different precisions. The number in the parenthesis denotes for many architectures, how many bytes a floating point variable is represented with.

\begin{lstlisting}[texcl]
real(8) :: A
real(4) :: B
integer(4) :: I
\end{lstlisting}

To be able to choose the correct precision for a floating point variable, Fortran has a built in function \fkeyw{selected\_real\_kind} that returns the value to be used in the declaration with a given precision. This is illustrated in the following example.

\begin{lstlisting}
integer, parameter :: ap = selected_real_kind(15,300)
real(kind=ap) :: X,Y
\end{lstlisting}

In this example the floating point variable should have at least 15 significant decimals and could represent numbers from 10$^{-300}$ to 10$^{300}$. For several common architectures \fkeyw{selected\_real\_kind} will return the value 8. The advantage of using the above approach is that the precision of the floating point values can be specified in a architectural independent way. The precision constant can also be used when
specifying numbers in variable assignments as the following example illustrate.

\begin{lstlisting}
X = 6.0_ap
\end{lstlisting}

The importance of specifying the precision for assigning scalar values to variables is illustrated in the following example.

\begin{lstlisting}
program constants

    implicit none

    integer, parameter :: ap = selected_real_kind(15,300)

    real(ap) :: pi1, pi2
    pi1 = 3.141592653589793
    pi2 = 3.141592653589793_ap

    write(*,*) 'pi1 = ', pi1
    write(*,*) 'pi2 = ', pi2

    stop

end program constants
\end{lstlisting}

The program gives the following results:

\cmdmode

\begin{lstlisting}
pi1 = 3.14159274101257
pi2 = 3.14159265358979
\end{lstlisting}

\fmode

The scalar number assigned to the variable \fvar{pi1} is chosen by the compiler to be represented by the least number of bytes floating point precision, in this case \ftype{real(4)}, which is shown in the output from the above program. 

Variable declarations in Fortran always precedes the executable statements in the main program or in a subroutine. Declarations can also be placed directly after the \fkeyw{module} identifier in modules. 

%---------------------------------------------------------------------
\subsection{General type rule}
%---------------------------------------------------------------------

Variable do not have to be declared in Fortran. The default is that variables starting I, J,..., N are defined as \ftype{integer} and variables starting with A, B,... ,H or O, P,... , Z are defined as \ftype{real}. This kind of implicit variable declaration is not recommended as it can lead to programming errors when variables are misspelled. To avoid implicit variable declarations the following declaration can be placed first in a program or module:

\fmode

\begin{lstlisting}
implicit none
\end{lstlisting}

This statement forces the compiler to make sure that all variables are declared. If a variable is not declared the compilation is stopped with an error message. This is default for many other strongly typed languages such as, C, C++ and Java.

%---------------------------------------------------------------------
\subsection{Assignment of variables}
%---------------------------------------------------------------------

The syntax for scalar variable assignment is, 

\begin{fsyntax}
variable = expr
\end{fsyntax}

where \fvar{variable} denotes the variable to be assigned and \fexpr{expr} the expression to be assigned. The following example assign the \fvar{a} variable the value 5.0 with the precision defined in the constant \fkeyw{ap}.

\begin{lstlisting}
a = 5.0_ap
\end{lstlisting}

Assignment of boolean variables are done in the same way using the keywords, \fkeyw{.false.} and \fkeyw{.true.} indicating a true or false value. A boolean expression can also be used int the assignment. In the following example the variable, \fvar{flag}, is assigned the value \fkeyw{.false.}.

\begin{lstlisting}
flag =.false.
\end{lstlisting}

Assignment of strings are illustrated in the following example. 

\begin{lstlisting}
character(40) :: first_name
character(40) :: last_name
character(20) :: company_name1
character(20) :: company_name2

...

first_name = 'Jan'
last_name = "Johansson"
company_name1 = "McDonald's"
company_name2 = 'McDonald''s'
\end{lstlisting}

The first variable, \fvar{first\_name}, is assigned the text ''Jan'', remaining characters in the string will be padded with spaces. A string is assigned using citation marks, '' or apostrophes, '. This can be of help when apostrophes or citation marks is used in strings as shown in the assignemnt of the variables, \fvar{company\_name1} och \fvar{company\_name2}.


%---------------------------------------------------------------------
\subsection{Defined and undefined variables}
%---------------------------------------------------------------------

%---------------------------------------------------------------------
\subsection{Derived datatypes}
%---------------------------------------------------------------------

%---------------------------------------------------------------------
\subsection{Determining appropriate types}
%---------------------------------------------------------------------

%---------------------------------------------------------------------
\section{Operators}
%---------------------------------------------------------------------

The following arithmetic operators are defined in Fortran:

\vspace{5mm}
\begin{tabular}{ll}
  \foper{**} & power to \\
  \foper{*} & multiplication \\
  \foper{/} & division \\
  \foper{+} & addition \\
  \foper{-} & subtraction \\
\end{tabular}
\vspace{5mm}

Parenthesis are used to specify the order of different operators. If no parenthesis are given in an expression operators are evaluated in the following order:

\begin{enumerate}
\item Operations with \foper{**}%
\item Operations with \foper{*} or \foper{/}%
\item Operations with \foper{+} or \foper{--}
\end{enumerate}

\noindent The following code illustrates operator precedence.

\begin{lstlisting}[texcl]
c = a+b/2 ! is equivalent to $a+(b/2)$
c = (a+b)/2 ! in this case $(a+b)$ is evaluated and then $/$ 2
\end{lstlisting}

\noindent Relational operators:

\vspace{5mm}
\begin{tabular}{lp{0.6\textwidth}}
  % after \\: \hline or \cline{col1-col2} \cline{col3-col4} ...
  \foper{<} or \foper{.lt.} & less than ({\underline {l}}ess {\underline {t}}han) \\
  \foper{<=} or \foper{.le.} & less than or equal to ({\underline {l}}ess than or {\underline {e}}qual) \\
  \foper{>} or \foper{.gt.} & greater than ({\underline{g}}reater {\underline {t}}han)  \\
  \foper{>=} or \foper{.ge.} & greater than or equal to ({\underline {g}}reater than or {\underline {e}}qual) \\
  \foper{==} or \foper{.eq.} & equal to ({\underline {eq}}ual) \\
  \foper{/=} or \foper{.ne.} & not equal to ({\underline {n}}ot {\underline {e}}qual) \\
\end{tabular}
\vspace{5mm}

\noindent Logical operators:

\vspace{5mm}
\begin{tabular}{ll}
\foper{.and.} & and \\
\foper{.or.} & or \\
\foper{.not.} & not \\
\end{tabular}
\vspace{5mm}

%---------------------------------------------------------------------
\subsection{Expressions}
%---------------------------------------------------------------------

Integer expressions.

Mixed mode expressions

Array expressions?


%---------------------------------------------------------------------
\section{Arrays and matrices}
%---------------------------------------------------------------------

In scientific and technical applications matrices and arrays are important concepts. As Fortran is a language mainly for
technical computing, arrays and matrices play a vital role in the language.

Declaring arrays and matrices can be done in two ways. In the first method the dimensions are specified using the special attribute, \fkeyw{dimension}, after the data type declaration. The second method, the dimensions are specified by adding the dimensions directly after the variable name. The following code illustrate these methods of declaring arrays.

\begin{lstlisting}
integer, parameter :: ap = selected_real_kind(15,300)
real(ap),dimension(20,20) :: K ! Matrix 20x20 elements
real(ap) :: fe(6) ! Array with 6 elements
\end{lstlisting}

The default starting index in arrays is 1. It is however possible to define custom indices in the declaration, as the following example shows.

\begin{lstlisting}
real(ap) :: idx(-3:3)
\end{lstlisting}

This declares an array, \fvar{idx} with the indices [-3, -2,
-1, 0, 1, 2, 3], which contains 7 elements.

%---------------------------------------------------------------------
\subsection{Array assignment}
%---------------------------------------------------------------------

Arrays are assigned values either by explicit indices or the entire array in a single statement. The following code assigned the variable, \fvar{K}, the value 5.0 at position row 5 and column 6.

\begin{lstlisting}
K(5,6) = 5.0
\end{lstlisting}

If the assignment had been written as

\begin{lstlisting}
K = 5.0
\end{lstlisting}

the entire array, \fvar{K}, would have been assigned the value 5.0. This is an efficient way of assigning entire arrays
initial values.

Explicit values can be assigned to arrays in a single statement using the following assignment.

\begin{lstlisting}
real(ap) :: v(5) ! Array with 5 elements
v = (/ 1.0, 2.0, 3.0, 4.0, 5.0 /)
\end{lstlisting}

This is equivalent to an assignment using the following statements.

\begin{lstlisting}
v(1) = 1.0
v(2) = 2.0
v(3) = 3.0
v(4) = 4.0
v(5) = 5.0
\end{lstlisting}

The number of elements in the list must be the same as the number of elements in the array variable.

Assignments to specific parts of arrays can be achieved using index-notation. The following example illustrates this concept.

\begin{lstlisting}[texcl]
program index_notation

    implicit none
    real :: A(4,4)
    real :: B(4)
    real :: C(4)

    B = A(2,:) ! Assigns B the values of row 2 in A
    C = A(:,1) ! Assigns C the values of column 1 in A

    stop

end program index_notation
\end{lstlisting}

Using index-notation rows or columns can be assigned in single statements as shown in the following code:

\begin{lstlisting}[texcl]
! Assign row 5 in matrix K the values 1, 2, 3, 4, 5

K(5,:) = (/ 1.0, 2.0, 3.0, 4.0, 5.0 /)

! Assign the array v the values 5, 4, 3, 2, 1

v = (/ 5.0, 4.0, 3.0, 2.0, 1.0 /)
\end{lstlisting}


%---------------------------------------------------------------------
\subsection{Array storage}
%---------------------------------------------------------------------

%---------------------------------------------------------------------
\subsection{Allocatable arrays}
%---------------------------------------------------------------------

In Fortran 77 and earlier versions of the standard it was not possible to dynamically allocate memory during program execution. This capability is now available in Fortran 90 and later versions. To declare an array as dynamically allocatable, the attribute \fkeyw{allocatable} must be added to the array declaration. The dimensions are also replaced with a colon, :, indicating the number of dimensions in the declared variable. A typical allocatable array declaration is shown in the following example.

\begin{lstlisting}
real, dimension(:,:), allocatable :: K
\end{lstlisting}

In this example the two-dimensional array, K, is defined as allocatable. To indicate that the array is two-dimensional is done by specifying \fkeyw{dimension(:,:)} in the variable attribute. To declare a one-dimensional array the code becomes:

\begin{lstlisting}
real, dimension(:), allocatable :: f
\end{lstlisting}

Variables with the \fkeyw{allocatable} attribute can't be used until memory is allocated. Memory allocation is done using the \fkeyw{allocate} method. To allocate the variables, \fvar{K,f}, in the previous examples the following code is used.

\begin{lstlisting}
allocate(K(20,20))
allocate(f(20))
\end{lstlisting}

When the allocated memory is no longer needed it can be deallocated using the command, \fkeyw{deallocate}, as the following code illustrates.

\begin{lstlisting}
deallocate(K)
deallocate(f)
\end{lstlisting}

An important issue when using dynamically allocatable variable is to make sure the application does not ''leak''. ''Leaking'' is term used by applications that allocate memory during the execution and never deallocate used memory. If unchecked the application will use more and more resources and will eventually make the operating system start swapping and perhaps become also become unstable. A rule of thumb is that an
\fkeyw{allocate} statement should always have corresponding \fkeyw{deallocate}. An example of using dynamically allocated arrays is shown in section XXX.

%---------------------------------------------------------------------
\subsection{Array subobjects}
%---------------------------------------------------------------------

%---------------------------------------------------------------------
\section{Conditional statements}
%---------------------------------------------------------------------

The simplest form of if-statements in Fortran have the following syntax

\begin{fsyntax}
\textbf{if} (scalar-logical-expr) \textbf{then}\\
\ftab  block\\
\textbf{end if}\\
\end{fsyntax}

where \fexpr{scalar-logical-expr} is a boolean expression, that has to be evaluated as true, (\fkeyw{.true.}), for the statements in, \fexpr{block}, to be executed. An extended version of the if-statement adds a \fkeyw{else}-block with the following syntax

\begin{fsyntax}
\textbf{if} (scalar-logical-expr) \textbf{then}\\
\ftab block1\\
\textbf{else}\\
\ftab block2\\
\textbf{end if}\\
\end{fsyntax}

In this form the \fexpr{block1} will be executed if \fexpr{scalar-logical-expr} is evaluated as true, otherwise \fexpr{block2} will be executed. A third form of if-statement contains one or more \fkeyw{else if}-statements with the following syntax:

\begin{fsyntax}
\textbf{if} (scalar-logical-expr1) \textbf{then}\\
\ftab block1\\
\textbf{else if} (scalar-logical-expr2) \textbf{then}\\
\ftab block2\\
\textbf{else}\\
\ftab block3\\
\textbf{end if}
\end{fsyntax}

In this form the \fexpr{scalar-logical-expr1} is evaluated first. If this expression is true \fexpr{block1} is executed, otherwise if \fexpr{scalar-logical-expr2} evaluates as true \fexpr{block2} is executed. If no other expressions are evaluated to true, \fexpr{block3} is executed. An if-statement can contain several \fkeyw{else if}-blocks. The use of if-statements is illustrated in the following example:

\lstinputlisting[texcl,escapechar=\%]{source/logic/logic.f90}

Another conditional constructi is the case-statement.

\begin{fsyntax}
\textbf{select case} (expr)\\
\ftab\textbf{case} selector \\
\ftab\ftab block  \\
\textbf{end select}
\end{fsyntax}

In this statement the expression, \fexpr{expr} is evaluated and the \fkeyw{case}-block with the corresponding \fexpr{selector} is executed. To handle the case when no \fkeyw{case}-block corresponds to the \fexpr{expr}, a \fkeyw{case}-block with the \fkeyw{default} keyword can be added. The syntax then becomes:

\begin{fsyntax}
\textbf{select case}(expr)\\
\ftab\textbf{case} selector\\
\ftab\ftab block \\
\ftab\textbf{case default}\\
\ftab\ftab block \\
\textbf{end select}
\end{fsyntax}

Example of case-statement use is shown in the following example:

\begin{lstlisting}
select case (display_mode)
    case (displacements)
        ...
    case (geometry)
        ...
end select
\end{lstlisting}

To handle the case when \fvar{display\_mode} does not correspone to any of the alternatives the above code is modified to the following code.

\begin{lstlisting}
select case (display_mode) case (displacements)
        ...
    case (geometry)
        ...
    case default
        ...
end select
\end{lstlisting}

The following program example illustrate how case-statements can be used.

\lstinputlisting[texcl]{source/case/case.f90}

%---------------------------------------------------------------------
\section{Repetitive statements}
%---------------------------------------------------------------------

The most common repetitive statement in Fortran is the \fkeyw{do}-statement. The syntax is:

\begin{fsyntax}
\textbf{do} variable = expr1, expr2 [,expr3]\\
\ftab block\\
\textbf{end do}
\end{fsyntax}

\fexpr{variable} is the control-variable of the loop. \fexpr{expr1} is the starting value, \fexpr{expr2} is the end value and \fexpr{expr3} is the step interval. If the step interval is not given it is assumed to be 1. There are two ways of controlling the execution flow in a \fkeyw{do}-statement. The \fkeyw{exit} command terminates the loop and program execution is continued after the \fkeyw{do}-statement. The \fkeyw{cycle} command terminates the execution of the current block and continues execution with the next value of the control variable. The example below illustrates the use of a \fkeyw{do}-statement.

\begin{lstlisting}[texcl]
program loop_sample

    implicit none

    integer :: i

    do i=1,20
        if (i>10) then
            write(*,*) 'Terminates do-statement.'
            exit
        else if (i<5) then
            write(*,*) 'Cycling to next value.'
            cycle
        end if
        write(*,*) i
    end do

    stop

end program loop_sample
\end{lstlisting}

The above program gives the following output:

\cmdmode

\begin{lstlisting}[texcl]
Cycling to next value.
Cycling to next value.
Cycling to next value.
Cycling to next value.
5
6
7
8
9
10
Terminates do-statement.
\end{lstlisting}

\fmode

Another repetitive statement available is the \fkeyw{do while}-statement. With this statement, the code block can execute until a certain condition is fulfilled. The syntax is:

\begin{fsyntax}
\textbf{do while} (scalar-logical-expr)\\
\ftab block\\
\textbf{end do}
\end{fsyntax}

The following code shows a simple \fkeyw{do while}-statement printing the function $f(x)=sin(x)$.

\begin{lstlisting}
x = 0.0
do while x<1.05
    f = sin(x)
    x = x + 0.1
    write(*,*) x, f
end do
\end{lstlisting}

There are other repetitive statements such as \fkeyw{forall} and \fkeyw{where} covered int the array features sections.

%---------------------------------------------------------------------
\section{Built-in functions}
%---------------------------------------------------------------------

Fortran has a number of built-in functions covering a number of different areas. The following tables list a selection of these. For a more thorough description of the built-in function please see, Metcalf and Reid \cite{metcalf00}.

\begin{table}[!hb]
\begin{center}
\begin{tabular}{|l|l|}
\hline Function & Description \\ \hline
\lstinline!acos(x)! & Returns $\arccos(x)$ \\
\lstinline!asin(x)! & Returns $\arcsin(x)$ \\
\lstinline!atan(x)! & Returns $\arctan(x)$ \\
\lstinline!atan2(y,x)! & Returns $\arctan(\frac{y}{x})$ from $-\pi$ till $-\pi$ \\
\lstinline!cos(x)!  & Returns $\cos(x)$ \\
\lstinline!cosh(x)! & Returns $\cosh(x)$ \\
\lstinline!exp(x)!  & Returns $e^{x}$ \\
\lstinline!log(x)!  & Returns $\ln(x)$ \\
\lstinline!log10(x)! & Returns $\lg(x)$ \\
\lstinline!sin(x)! & Returns $\sin(x)$ \\
\lstinline!sinh(x)! & Returns $\sinh(x)$ \\
\lstinline!sqrt(x)! & Returns $\sqrt{x}$\\
\lstinline!tan(x)! & Returns $\tan(x)$ \\
\lstinline!tanh(x)! & Returns $\tanh(x)$ \\ \hline
\end{tabular}
\end{center}
\caption{Mathematical functions}
\end{table}

\begin{table}[!hb]
\begin{center}
\begin{tabular}{|l|l|}
\hline Function & Description \\ \hline
\lstinline!abs(a)!  & Returns absolute value of a \\
\lstinline!aint(a)! & Truncates a floating point value \\
\lstinline!int(a)!  & Converts a floating point value to an integer \\
\lstinline!nint(a)! & Rounds a floating point value to the nearest integer \\
\lstinline!real(a)! & Converts an integer to a floating point value \\
\lstinline!max(a1,a2[,a3,...])! & Returns the maximum value of two or more values \\
\lstinline!min(a1,a2[,a3,...])! & Returns the minimum value of two or more values \\
\hline
\end{tabular}
\end{center}
\caption{Miscellaneous conversion functions}
\end{table}

\begin{table}[!hb]
\begin{center}
\begin{tabular}{|l|p{70mm}|}
\hline
Function & Description \\
\hline
\lstinline!dot_product(u, v)! & Returns the scalar product of $u\cdot v$  \\
 & \\
\lstinline!matmul(A, B)! &  Matrix multiplication. The result must have the same for as $\mathbf{AB}$ \\
 & \\
\lstinline!transpose(C)! & Returns the transpose $\mathbf{C}^{T}$. Elementet $C^{T}_{ij}$ motsvarar $C_{ji}$ \\
\hline
\end{tabular}
\end{center}
\caption{Vector and matrix functions}
\end{table}

\begin{table}[!hb]
\begin{center}
\begin{tabular}{|l|p{75mm}|}
\hline
Function & Description \\
\hline
\lstinline!all(mask)! &  Returns true of all elements in the logical array \fexpr{mask} are true. For example \lstinline!all(A>0)! returns true if all elements
in $\mathbf{A}$ are greater than 0. \\
 & \\
\lstinline!any(mask)! &  Returns true if any of the elements in \fexpr{mask} are true. \\
 & \\
\lstinline!count(mask)! & Returns the number of elements in \fexpr{mask} that are true. \\
 & \\
\lstinline!maxval(array)! &   Returns the maximum value of the elements in the array \fexpr{array}. \\
 & \\
\lstinline!minval(array)! &   Returns the minimum value of the elements in the array \fexpr{array}. \\
 & \\
\lstinline!product(array)! &  Returns the product of the elements in the array \fexpr{array}. \\
 & \\
\lstinline!sum(array)! &  Returns the sum of elements in the array \fexpr{array}. \\
\hline
\end{tabular}
\end{center}
\caption{Array functions}
\end{table}

Most built-in functions and operators in Fortran support arrays. The following example shows how functions and operators support operations on arrays.

\begin{lstlisting}[texcl]
real, dimension(20,20) :: A, B, C

C = A/B ! Division $C_{ij}=A_{ij}/B_{ij}$

C = sqrt(A) ! Square root $C_{ij}=\sqrt{A_{ij}}$
\end{lstlisting}

The following example shows how a stiffness matrix for a bar element easily can be created using these functions and operators. The Matrix $\mathbf{K}_{e}$ is defined as follows

\begin{equation}
\mathbf{K}_{e} =(\mathbf{G}^{T} \mathbf{K}_{el} )\mathbf{G}
\end{equation}

The $G^T$ is returned by using the Fortran function \fkeyw{transpose} and the matrix multiplications are performed with \fkeyw{matmul}. The matrices $\mathbf{K}_{el}$ and $\mathbf{G}$ are defined as

\begin{equation}
\mathbf{K}_{el} =\frac{EA}{L} \left[
\begin{array}{cc}
1 & -1 \\
-1 & 1 \\
\end{array}
\right]
\end{equation}

and

\begin{equation}
\mathbf{G}=\left[
\begin{array}{cccccc}
n_{x}  & n_{y}  & n_{z}  & 0 & 0 & 0 \\
0 & 0 & 0 & n_{x} & n_{y} & n_{z} \\
\end{array}
\right]
\end{equation}

Length and directional cosines are defined as

\begin{equation}
L=\sqrt{(x_{2} -x_{1} )^{2} +(y_{2} -y_{1} )^{2} +(z_{2} -z_{1}
)^{2}}
\end{equation}

\begin{equation}
n_{x} =\frac{x_{2} -x_{1} }{L} ,\,\,\,n_{y} =\frac{y_{2} -y_{1}
}{L} ,\,\,\,n_{z} =\frac{z_{2} -z_{1} }{L}
\end{equation}

In the example the input parameters are assigned the following values:

\begin{eqnarray}
x_{1} =0,\,y_{1} =0,\,z_{1} =0 \\
x_{2} =1,\,y_{2} =1,\,z_{2} =1 \\
E\/=\/1,\,\,A=1
\end{eqnarray}

\lstinputlisting[texcl,escapechar=\%]{source/functions/functions.f90}

The program produces the following output

\cmdmode

\begin{lstlisting}[numbers=none,breaklines=false]
0.1925    0.1925    0.1925    -.1925    -.1925    -.1925
0.1925    0.1925    0.1925    -.1925    -.1925    -.1925
0.1925    0.1925    0.1925    -.1925    -.1925    -.1925
-.1925    -.1925    -.1925    0.1925    0.1925    0.1925
-.1925    -.1925    -.1925    0.1925    0.1925    0.1925
-.1925    -.1925    -.1925    0.1925    0.1925    0.1925
\end{lstlisting}

\fmode

For a more thorough description of matrix handling in Fortran 90/95, see Metcalf and Reid \cite{metcalf00}

\subsection{Elemental procedures}
\subsection{Vector and matrix functions}
\subsection{Reduction routines}
\subsection{Information functions}

%---------------------------------------------------------------------
\section{Program units and subroutines}
%---------------------------------------------------------------------

%---------------------------------------------------------------------
\subsection{Subprograms}
%---------------------------------------------------------------------

A subroutine in Fortran 90/95 has the following syntax

\begin{fsyntax}
\textbf{subroutine} subroutine-name[([dummy-argument-list])]\\
\ftab [argument-declaration]\\
\ftab ...\\
\ftab \textbf{return}\\
\textbf{end subroutine} [subroutine-name]
\end{fsyntax}

All variables in Fortran program are passed to subroutines as references to the actual variables. Modifying a parameter in a subroutine will modify the values of variables in the calling subroutine or program. To be able to use the variables in the argument list they must be declared in the subroutine. This is done right after the subroutine declaration. When a subroutine is finished control is returned to the calling routine or program using the \fkeyw{return}-command. Several return statements can exist in subroutine to return control to the calling routine or program. This is illustrated in the following example.

\begin{lstlisting}
subroutine myproc(a,B,C)
    implicit none
    integer :: a
    real, dimension(a,*) :: B
    real, dimension(a) :: C
    .
    .
    .
    return
end subroutine
\end{lstlisting}

A subroutine is called using the \fkeyw{call} statement. The above subroutine is called with the following code.

\begin{lstlisting}
call myproc(a,B,C)
\end{lstlisting}

It should be noted that the names used for variables are local to each respective subroutine. Names of variables passed as arguments does not need to have the same name in the calling and called subroutines. It is the order of the arguments that determines how the variables are referenced from the calling subroutine.

In the previous example illustrates how to make the subroutines independent of problem size. The dimensions of the arrays are passed using the \fvar{a} parameter instead of using constant values. The last index of an array does not have to specified, indicated with a *, as it is not needed to determine the address to array element.

%---------------------------------------------------------------------
\subsection{Functions}
%---------------------------------------------------------------------

Functions are subroutines with a return value, and can be used in different kinds of expressions. The syntax is

\begin{fsyntax}
type \textbf{function} function-name([dummy-argument-list])\\
\ftab [argument-declaration]\\
\ftab ...\\
\ftab function-name = return-value\\
\ftab ...\\
\ftab \textbf{return}\\
\textbf{end function} function-name
\end{fsyntax}

The following code shows a simple function definition returning the value of $sin(x)$

\begin{lstlisting}
real function f(x)
    real :: x
    f=sin(x)
    return
end function f
\end{lstlisting}

The return value defined by assigning the name of the function a value. As seen in the previous example. The function is called by giving the name of the function and the associated function arguments.

\begin{lstlisting}
a = f(y)
\end{lstlisting}

The following example illustrates how to use subroutines to assign an element matrix for a three-dimensional bar element. The example also shows how dynamic memory allocation can be used to allocate matrices. See also the example in section XX

\lstinputlisting[texcl]{source/subroutines/subroutines.f90}

The program gives the following output.

\begin{lstlisting}[numbers=none,breaklines=false]
0.1925    0.1925    0.1925    -.1925    -.1925    -.1925
0.1925    0.1925    0.1925    -.1925    -.1925    -.1925
0.1925    0.1925    0.1925    -.1925    -.1925    -.1925
-.1925    -.1925    -.1925    0.1925    0.1925    0.1925
-.1925    -.1925    -.1925    0.1925    0.1925    0.1925
-.1925    -.1925    -.1925    0.1925    0.1925    0.1925
\end{lstlisting}

%---------------------------------------------------------------------
\subsection{Keyword and optional arguments}
%---------------------------------------------------------------------

Sometimes when implementing subroutines the number of arguments can grow, making the usage of the unnecessary complicated. To solve this wrapper subroutines could be written providing default parameters for main subroutines. This has the drawback of additional maintenance of the wrapper subroutines when the main subroutine is changed. Fortran 2003 provides a solution to this using keyword and optional arguments. An additional parameter attribute, \fkeyw{optional}, can be specified when declaring subroutine parameters. In the following example the, \fvar{c} is declared optinal and does not need to be given when the routine is called.

\begin{lstlisting}
subroutine dostuff(A, b, c)

	real(8) :: A(10,10)
	integer :: b
	integer, optional :: c
	...
	
\end{lstlisting}

The \fkeyw{dostuff} routine can be called in 2 ways:

\begin{lstlisting}
call dostuff(A, b)    ! c is omitted as it is optional
call dostuff(A, b, c) 
\end{lstlisting}

If a routine is called without optional parameters the routine has to be able to determine if a parameter used this can be done using a special function, \fkeyw{present(...)}. This functions returns \fkeyw{.true.} if given parameter is present in the call to the subroutine.

In addition of having optional parameters, subroutine parameters can also be specified by parameter name or keyword. In the following example all these techniques are employed when implementing the \fkeyw{order\_icecream} subroutine. This routine only has one required argument, \fkeyw{number}. The other parameters are optional as indicated by the, \fkeyw{optional} in the parameter declaration.

\lstinputlisting{source/optional_arguments/main.f90}

In the last call the \fkeyw{topping} keyword is used to specify the last optional argument, but leaving the \fkeyw{flavor} parameter undefined.

%---------------------------------------------------------------------
\subsection{Procedure arguments}
%---------------------------------------------------------------------

An efficient feature that exists in many other languages is the ability to pass subroutines as arguments to subroutines. This can provide efficient ways to provide algorithms with user provided functions to be used within the algorithm. As an example, a function can be input to a numeric differentiation algorithm as a subroutine parameter. It is now possible to do this in Fortran 2003. 

To implement a subroutine that takes a function as an input parameter, the function definition has to be declared in the subroutine parameter declaration using an \fkeyw{interface} block.

\begin{lstlisting}
real(8) function integrate(a, b, func)

	real(8) :: a, b
	
	interface 
		real(8) function func(x)
			real(8), intent(in) :: x
		end function func
	end interface
	
	...
\end{lstlisting}

The routine can then be called by providing a function with the same interface as input to the function:

\begin{lstlisting}
real(8) function myfunc(x)
	
	real(8) :: x
	
	myfunc = sin(x)**2
	
end function myfunc
\end{lstlisting}

Calling the \fkeyw{integrate} function then becomes:

\begin{lstlisting}
area = integrate(0.0, 1.0, myfunc)
\end{lstlisting}

\lstinputlisting{source/procedures_as_arguments/utils.f90}
\lstinputlisting{source/procedures_as_arguments/main.f90}


%---------------------------------------------------------------------
\subsection{Modules}
%---------------------------------------------------------------------

When programs become larger, they often need to be split into more manageable parts. In other languages this is often achieved using include files or packages. In Fortran 77, no such functionality exists. Source files can be grouped in files, but no standard way of including specific libraries of subroutines or function exists in the language. The C preprocessor is often used to include code from libraries in Fortran, but is not standardised in the language itself. 

In Fortran 90 the concept of modules was introduced. A Fortran 90 module can contain both variables, parameters and subroutines. This makes it possible to divide programs into well defined modules which are more easily maintained. The syntax for a module is similar to that of how a main program in Fortran is defined.

\begin{fsyntax}
\textbf{module} module-name \\
\ftab [specification-stmts] \\
\protect{[}\textbf{contains} \\
\ftab module-subprograms] \\
\textbf{end module} [module-name]]
\end{fsyntax}

The block \fexpr{specification-stmts} defines the variables that are available for programs or subroutines using the module. In the block, \fexpr{module\--sub\-programs}, subroutines in the module are declared. A module can contain only variables or only subroutines or both. One use of this, is to declare variables common to several modules i a separate module. Modules are also a good way to divide a program into logical and coherent parts. Variables and functions in a module can be made private to a module, hiding them for routines using the module. The keywords \fkeyw{public} and \fkeyw{private} can be used to control the access to a variable or a function. In the following code the variable, \fvar{a}, is hidden from subroutines or programs using this module. The variable, \fvar{b}, is however visible. When nothing is specified in the variable declaration, the variable is assumed to be public.

\begin{lstlisting}
module mymodule

    integer, private :: a
    integer :: b
    ...

\end{lstlisting}

The ability to hide variables in modules enables the developer to hide the implementation details of a module, reducing the risk of accidental modification variables and use of subroutines used in the implementation.

To access the routines and variables in a module the \fkeyw{use} statement is used. This makes all the public variables and subroutines available in programs and other modules. In the following example illustrate how the subroutines use in the previous examples are placed in a module, \fname{truss}, and used from a main program. 

\lstinputlisting[texcl]{source/modules/module_truss.f90}

Main program using the \fname{truss} module.

\lstinputlisting[texcl]{source/modules/module_main.f90}

Please note that the declaration of \fvar{ap} in the \fname{truss} module is used to define the precision of the variables in the main program.

%---------------------------------------------------------------------
\subsection{Public and private attributes}
%---------------------------------------------------------------------

When implementing modules, some of the routines and variables are only used the implementation of the module. That is, some of the variables and subroutines should not be accessible for the user of the module. To control access to variables and subroutines the attributes \fkeyw{private} and \fkeyw{public} can be used in the declaration of variables and subroutines. A variable can be declared private by adding the keyword \fkeyw{private} to the attribute list in the declaration as shown in the following example:

\begin{lstlisting}
real, private :: a
\end{lstlisting}

If no \fkeyw{private} attribute is given the variable is by default declared as \fkeyw{public}. If a private variable is access from another module a main program will generate a compiler error. 

To declare a subroutine or function as private it has to be declared as such in the specification part of the module, that is before the \fkeyw{contains}-keyword. In the following example illustrates the concept.

\begin{lstlisting}
module mymodule

	private myprivatesub
	
contains

	subroutine myprivatesub
	
		print *, 'This subroutine can only be called from within the module.'
	
	end subroutine myprivatesub
	
	subroutine mypublicsub
	
		print *, 'This subroutine can be called from other modules.'
	
	end subroutine mypublicsub
	
end module mymodule
\end{lstlisting}

In this example, \fname{myprivatesub}, can only be called from within the module. Calling it from another module or main program will result in a compiler error. \fname{myprivatesub} is not declared as private in the specification part and hence can be called from all other modules. 

%---------------------------------------------------------------------
\subsection{Overloading}
%---------------------------------------------------------------------

As Fortran is a strongly typed language, supporting multiple data types in a single subroutine is not possible and requires separate unique subroutines declarations. To simplify module use and enable a module user to call a routine with different data types, Fortran 90 supports the concept of overloading. Using overloading the compiler can decide which routine to call depending on the datatype used. However, this requires a special declaration in the module specification. 

To illustrate this, a function, \fname{func}, is implemented that can take either a floating point parameter or an integer parameter. To implement this function, an interface declaration for \fname{func} is added in the module specification part:

\begin{lstlisting}
module overloaded

	interface func
		module procedure ifunc, rfunc
	end interface
	...
\end{lstlisting}

This tells the compiler to map the \fname{func}-function to the functions \fname{ifunc} or \fname{rfunc}, depending on the datatype used when the function is called. \fname{ifunc} or \fname{rfunc} are implemented as normal functions as shown below:

\begin{lstlisting}
	...
	
contains

integer function ifunc(x)
	
	integer, intent(in) :: x
	ifunc = x * 42
	
end function ifunc

real(8) function rfunc(x)
	
	real(8), intent(in) :: x
	rfunc = x / 42.0_8
	
end function rfunc

end module overloaded
\end{lstlisting}

The \fname{func}-function can now be called using either floating point values or integer values illustrated below in the following example:

\begin{lstlisting}
program overloading

	use special
	
	integer :: a = 42
	real(8) :: b = 42.0_8
	
	a = func(a)
	b = func(b)
	
	print *, a
	print *, b 
	
end program overloading
\end{lstlisting}

Running this program produce the following output:

\cmdmode

\begin{lstlisting}
$ ./overloading 
        1764
   1.0000000000000000
\end{lstlisting}%$

\fmode

This means that \fname{ifunc} is called in the first call to \fname{func} and \fname{rfunc} is called in the second call to \fname{func}.

%---------------------------------------------------------------------
\subsection{operator overloading}
%---------------------------------------------------------------------

In many modern languages such as C++ and Python, the operators can be overloaded to support expressions for user implemented data types. This is also possible in Fortran. To illustrate how this is achieved, a \fname{vector\_operations}-module is implemented, enabling addition of vectors using the + operator. 

First, a vector type is defined in our module \fname{vector\_operations}. This is the actual data type that will be used in the expressions to be evaluated.

\begin{lstlisting}
module vector_operations

	type vector
		real(8) :: components(3)
	end type vector
	...
\end{lstlisting}

Next, an interface for overloading the + operator is defined. The interface tells the compiler which function to call when it encounters an expression with our vector data type. In this example the \fname{vector\_plus\_vector}-function will be called for the + operator.

\begin{lstlisting}
	...
	interface operator(+)
		module procedure vector_plus_vector
	end interface
	...
\end{lstlisting}

In the final step the actual function for adding vectors is implemented. This functions needs to have two input parameters for the vectors to be added in the operation. It also needs to return a \fname{vector} data type.

\begin{lstlisting}
...
contains

type(vector) function vector_plus_vector(v1, v2)

	type(vector), intent(in) :: v1, v2
	vector_plus_vector%components = v1%components + v2%components
	
end function vector_plus_vector

end module vector_operations
\end{lstlisting}

The new data type together with the defined + operator can now be used to implement compact expressions for vector algebra as illustrated in the following code:

\begin{lstlisting}
program operator_overloading

	use vector_operations

	type(vector) :: v1
	type(vector) :: v2
	type(vector) :: v
	
	v1%components = (/1.0, 0.0, 0.0/)
	v2%components = (/0.0, 1.0, 0.0/)
	
	v = v1 + v2
	
	print *, v

end program operator_overloading
\end{lstlisting}

Running the code will produced the expected output:

\cmdmode

\begin{lstlisting}
$ ./opoverload 
   1.0000000000000000        1.0000000000000000        0.0000000000000000    
\end{lstlisting}%$

\fmode

Operators for -, * and / can be implemented using the same technique.

%---------------------------------------------------------------------
\subsection{Allocatable dummy arguments}
%---------------------------------------------------------------------

\lstinputlisting{source/allocatable_dummy/main.f90}

%---------------------------------------------------------------------
\subsection{Allocatable functions}
%---------------------------------------------------------------------

\lstinputlisting{source/allocatable_function/main.f90}

%---------------------------------------------------------------------
%\subsection{Submodules (2003)}
%---------------------------------------------------------------------

%\lstinputlisting{source/submodules/points.f90}
%\lstinputlisting{source/submodules/points_a.f90}
%\lstinputlisting{source/submodules/main.f90}

%---------------------------------------------------------------------
\section{Input and output}
%---------------------------------------------------------------------

Input and output to and from different devices, such as screen, keyboard and files are accomplished using the commands \fkeyw{read} and \fkeyw{write}. The syntax for these commands are:

\begin{fsyntax}
read(u, fmt) [list]\\
write(u, fmt) [list]
\end{fsyntax}

\fexpr{u} is the device that is used for reading or writing. If a star (*) is used as a device, standard output and standard
input are used (screen, keyboard or pipes).

\fexpr{fmt} is a string describing how variables should be read or written. This is often important when writing results to text files, to make it more easily readable. If a star (*) is used a so called free format is used, no special formatting is used. The format string consists of one or more format specifiers, which have the general form:

\begin{fsyntax}
[repeat-count] format-descriptor w[.m]
\end{fsyntax}

where \fexpr{repeat-count} is the number of variables that this format applies to. \fexpr{format-descriptor} defines the type of format specifier. \fexpr{w} defined the width of the output field and \textit{m} is the number of significant numbers or decimals in the output. The following example outputs some numbers using different format specifiers and table~\ref{table:formatkoder} show the most commonly used format specifiers.

\lstinputlisting[texcl]{source/formatting/formatting.f90}

The program produces the following output:

\begin{lstlisting}
123456789012345
      5.676
     0.6758E-01
     0.6758E+00
         0.6758
  0.675779000000000
            156
         156
\end{lstlisting}

\begin{table}
\begin{center}
\begin{tabular}{|l|l|}
\hline
Kod & Beskrivning \\
\hline
E   & Scientific notation. Values are converted to the format "-d.dddE+ddd". \\
F   & Decimal notation. Values are converted to the format "-d ddd.ddd...". \\
G   & Generic notation. Values are converted to the format -ddd.ddd or -d.dddE+ddd \\
I   & Integers. \\
A   & Strings \\
TRn & Move $n$ positions right \\
Tn  & Continue at position $n$ \\
\hline
\end{tabular}
\end{center}
\caption{Formatting codes in \fkeyw{read}/\fkeyw{write}}
\label{table:formatkoder}
\end{table}

During output a invisible cursor is moved from left to right. The format specifiers TR$n$ and T$n$ are used to move this cursor. TR$n$ moves the cursor $n$ positions to the right from the previous position. T$n$ places the cursor at position $n$. Figure~\ref{fig:format_positiong} shows how this can be used in a write-statement.

\fignormal{kompendiumv4Fig1} {Positioning of output in Fortran
90/95} {fig:format_positioning}

The output routines in Fortran was originally intended to be used on row printers where the first character was a control character. The effect of this is that the default behavior of these routines is that output always starts at the second position. On modern computers this is not an issue, and the first character can be used for printing. To print from the first character, the format specifier \fkeyw{T1} can be used to position the cursor at the first position. The following code writes ''Hej hopp!'' starting from the first position.

\begin{lstlisting}
write(*,'(T1,A)') 'Hej hopp!'
\end{lstlisting}

A more thorough description of the available format specifiers in Fortran is given in Metcalf and Reid~\cite{metcalf00}.

%---------------------------------------------------------------------
\subsection{Reading and Writing from files}
%---------------------------------------------------------------------

The input and output routines can also be used to write data to and from files. This is accomplished by associating a file in the file system with a file unit number, and using this number in the \fkeyw{read} and \fkeyw{write} statements to direct the input and output to the correct files. A file is associated, opened, with a unit number using an \fkeyw{open}-statement. When operations on the file is finished it is closed using the
\fkeyw{close}-statement. 

The file unit number is an integer usually between 1 and 99. On many systems the file unit number 5 is the keyboard and unit 6 the screen display. It is therefore recommended to avoid using these numbers in file operations.

In the \fkeyw{open}-statement the properties of the opened files are given, such as if the file already exists, how the file is accessed (reading or writing) and the filename used in the filesystem.

An example of reading and writing file is given in the following example.

\lstinputlisting[texcl]{source/sample2/sample2.f90}

In this example, 2 files are opened, \ffname{indata.dat} and \ffname{utdata.dat} with \fkeyw{open}-statements. Using the \fkeyw{read}-statement five rows with 3 numbers on each row are read from the file \ffname{indata.dat}. The sum of each row is calculated and is written using \fkeyw{write}-statements to the file \ffname{utdata.dat}. Finally the files are closed using the \fkeyw{close}-statements.

%---------------------------------------------------------------------
\subsection{Dynamic format codes}
%---------------------------------------------------------------------

One problem that arises when writing formatted output, is how to handle output of data in which the number of columns is unknown at compile time. To solve this, a special technique using strings as file units can be employed. To illustrate this technique we implement a subroutine \fname{writeArray}, which takes an array of any size as input and tries to print it nicely. First we declare the module subroutine and extract the size of the incoming array:

\begin{lstlisting}
subroutine writeArray(A)
		
	real(8), dimension(:,:) :: A
	integer :: rows, cols, i, j
	character(255) :: fmt
		
	rows = size(A,1)
	cols = size(A,2)

	...
\end{lstlisting}

Next, we use a \fname{write}-statement, that instead of a file unit number takes the string, \fname{fmt}, and uses it as an output file. In the \fname{write}-statement, we write out the needed format code for printing the incoming array, which is then stored in the string \fname{fmt}.

\begin{lstlisting}
	...
	write(fmt, '(A,I1,A)') '(',cols, 'G8.3)'  
	...
\end{lstlisting}

The generated format code can now be used when printing the incoming array \fvar{A}.

\begin{lstlisting}
	...
	do i=1,rows
		print fmt, (A(i,j), j=1,cols)
	end do
		
	return
	
end subroutine writeArray
\end{lstlisting}

In the following main program, the implemented \fname{writeArray} subroutine is used to print a 6 by 6 matrix.

\begin{lstlisting}
program dynamic_fcodes

	use array_utils

	real(8) :: A(6,6)
	
	A = 42.0_8
	
	call writeArray(A)
	
end program dynamic_fcodes
\end{lstlisting}

The resulting formatted output is shown below:

\cmdmode

\begin{lstlisting}
$ ./dynamic_fcodes 
42.0    42.0    42.0    42.0    42.0    42.0    
42.0    42.0    42.0    42.0    42.0    42.0    
42.0    42.0    42.0    42.0    42.0    42.0    
42.0    42.0    42.0    42.0    42.0    42.0    
42.0    42.0    42.0    42.0    42.0    42.0    
42.0    42.0    42.0    42.0    42.0    42.0    
\end{lstlisting}%$

\fmode

%---------------------------------------------------------------------
\subsection{Namelist I/O}
%---------------------------------------------------------------------

The standard way of writing or reading text files in Fortran is using list directed I/O. This means specifying a list of variables to be read or written using the \fkeyw{read}- and \fkeyw{write}-statements. Fortran will automatically handle the conversion of datatypes to and from a text based format. A more flexible way of handling text file I/O is using namelists. Namelists can be considered as named list of variables to be used for reading or writing. In this scheme, variables can be read and written to files using names. To write variables and data using this technique, variables must be listed using the special \fkeyw{namelist} statement as shown below:

\begin{lstlisting}
integer :: no_of_eggs, litres_of_milk, kilos_of_butter, list(5)
namelist /food/ no_of_eggs, litres_of_milk, kilos_of_butter, list
\end{lstlisting}

Here a namelist, \fkeyw{food}, is defined consisting of the specified variables. Variables in a namelist can be of any type. To write the variables to a file, the \fkeyw{nml}-keyword can be used in the \fkeyw{read}- and \fkeyw{write}-statements to specify which namelist that should be used. 

The namelist in the text file starts with the \& character followed by the namelist-name then the namelist variable pairs are listed separated by commas. The namelist is ended with a single /. The following example shows 2 namelist entries in a text file:

\begin{lstlisting}
&food litres_of_milk=5, no_of_eggs=12, kilos_of_butter=42, list=1,2,3,4,5 /
&food litres_of_milk=6, no_of_eggs=24, kilos_of_butter=84, list=2,3,4,5,6 /
\end{lstlisting}

Multiple namelist entries can be read from an opened file. The following code shows how 2 namelist entries of the type \fvar{food} are read from an opened file:

\begin{lstlisting}
open(unit=ir, file='food.txt', status='old')
read(ir, nml=food)
print *, no_of_eggs, litres_of_milk, kilos_of_butter
read(ir, nml=food)
print *, no_of_eggs, litres_of_milk, kilos_of_butter
close(unit=ir)
\end{lstlisting}

Running this code produces the following output:

\cmdmode

\begin{lstlisting}
          12           5          42
          24           6          84
\end{lstlisting}

\fmode

Writing using namelist I/O is done in the same way as reading. The following code shows how the same namelist variables are written to a namelist:

\begin{lstlisting}
open(unit=iw, file='food2.txt', status='new')
write(iw, nml=food)
close(unit=iw)
\end{lstlisting}

The contents of the written file, \fname{food2.txt}, is shown below:

\cmdmode

\begin{lstlisting}
&FOOD
 NO_OF_EGGS=         24,
 LITRES_OF_MILK=          6,
 KILOS_OF_BUTTER=         84,
 LIST=          2,          3,          4,          5,          6,
 
 /
\end{lstlisting}

Pleas notice that Fortran allways uses uppercase for variable names in the written file.

%\lstinputlisting{source/namelist_io/main.f90}

%---------------------------------------------------------------------
\subsection{Unformatted I/O}
%---------------------------------------------------------------------

In the previous sections data was read and written in human readable text format. For larger data structures this can be very inefficient. To solve this Fortran can also write data in its native binary format directly to disk. This can save space and can also be read and written much faster to disk. However, the binary format is not standardised and differs between different hardware platforms, preventing files to be used on different hardware. 

Reading and writing binary data is done using the same \fkeyw{read}- and wr\fkeyw{write}-ite statements as before, but without the formatting options. Writing an array to disk in binary form can be done with just one simple statement:

\fmode

\begin{lstlisting}
real :: A(100)
...
write(iw) A
\end{lstlisting}

Reading the same array back from disk is just as easy, using the \fkeyw{read}-statement.

\begin{lstlisting}
real :: A(100)
...
read(ir) A
\end{lstlisting}

It is also possible to write several variables to disk using multiple \fkeyw{write}- statements.

\begin{lstlisting}
real :: A(100), B(200)
...
write(iw) A
write(iw) B
\end{lstlisting}

However, it is important to note that data has to be read back in the same order it is was written. So the code for reading the data back becomes:

\begin{lstlisting}
real :: A(100), B(200)
...
read(ir) A
read(ir) B
\end{lstlisting}

To enable reading and writing unformatted I/O files (binary files) the keyword \fvar{form='unformatted'} must be added to the \fkeyw{open}-statement. 

\begin{lstlisting}
real :: A(100), B(200)
...
open(unit=ir, file='arrays.dat', form='unformatted')
read(ir) A
read(ir) B
close(ir)
\end{lstlisting}

The concept of unformatted I/O is illustrated in a larger example. In this example an array of the derived datatype \fvar{particle} is created, initialised and then saved to disk as unformatted I/O. After saving the data to disk it is read back using unformatted I/O and printed on standard output. The listing is shown below:

\lstinputlisting{source/unformatted_io/main2.f90}

The code produced the following output when run:

\cmdmode

\begin{lstlisting}
$ ./unformatted_io_2
   0.00000000       0.00000000       0.00000000       0.00000000       0.00000000       0.00000000       1.00000000       0.00000000       0.00000000       0.00000000       0.00000000       0.00000000       0.00000000       1.00000000       0.00000000       0.00000000       0.00000000       0.00000000       0.00000000       0.00000000       1.00000000       0.00000000       0.00000000       0.00000000       0.00000000       0.00000000       0.00000000       1.00000000       0.00000000       0.00000000       0.00000000       0.00000000 ....
\end{lstlisting}%$

Which means that the data was read correctly back from disk. 

%---------------------------------------------------------------------
\subsection{Direct access files}
%---------------------------------------------------------------------

A variant of unformatted I/O is direct access files. One problem with unformatted I/O is that files have to be read and written sequentially. This make it inefficient if you would like to access certain parts of the file randomly. To solve this problem Fortran provides direct access file format. In this format the file is divided in several equally spaced data records. These records can be read randomly back from a single file. It can be compared to a datbase file with data records. 

To create a direct access file consisting of records of the following derived data type,

\fmode

\begin{lstlisting}
type account
    character(len=40) :: account_holder
	real :: balance
end type account
\end{lstlisting}

the size of the data record has to be calculated. This can be done using the \fkeyw{inquire}-function. This assigns a variable the record size of the data type, which is shown in the following listing:

\begin{lstlisting}
type(account) :: account
integer :: recordSize
...	
inquire(iolength=recordSize) account
\end{lstlisting}

The \fvar{recordSize} variable can now be used when we create a direct access file using the \fkeyw{open}-statement:

\begin{lstlisting}
open(unit=iw, file='accounts.dat', access='direct', recl=recordSize, status='replace')
\end{lstlisting}

Writing the records is accomplished using the normal \fkeyw{write}-statement with an added \fkeyw{rec}-option for the record position to be written.

\begin{lstlisting}
write(iw, rec=1) account
\end{lstlisting}

It is possible to write to any record position when writing record. Reading record is done using the \fkeyw{read}-statements using the \fkeyw{rec}-option.

When reading or writing to direct access files there is an invisible cursor or pointer pointing to the current record. It is possible to manipulate this cursor using the \fkeyw{rewind}- and \fkeyw{backspace}-statements. The \fkeyw{rewind}-statement moves the pointer to the first record in the file. the \fkeyw{backspace}-statement moves the pointer one record back in the file, these operations are illustrated in figure~\ref{fig:io_rewind_backspace}.

\fignormal{io_rewind_backspace}{\fkeyw{rewind}- and \fkeyw{backspace}-statements}{fig:io_rewind_backspace}

It is also possible to truncate a direct access file at a given position using the \fkeyw{endfile}-statement, as illustrated in figure~\ref{fig:io_endfile}. All records after the current record will be truncated.

\figsmall{io_endfile}{Truncating a file using the \fkeyw{endfile}-statement}{fig:io_endfile}

The following code shows a complete example, writing to records to a direct access file:

\lstinputlisting{source/unformatted_io/main.f90}

%---------------------------------------------------------------------
\subsection{Error handling in I/O operations}
%---------------------------------------------------------------------

One problem with reading and writing files is that errors can occur on a system level, such as unavailable disk space, file system problems and non-existant files. If not handled, the program will crash in an unexpected way and prevent proper clean up code to be run. In Fortran I/O errors are handled using the \fkeyw{err}-option in all I/O related functions. Using the \fkeyw{err}-option a label can be defined where the execution continues when an I/O error occurs. 

In the following code a file is opened for reading. Error handling code is added for each I/O operation, providing a different label and response for all error conditions. 

\lstinputlisting{source/error_handling/main.f90}

Figure~\ref{fig:error_handling} shows the flow of the program.

\figsmall{error_handling}{Error handling i Fortran I/O operations}{fig:error_handling}

\newpage
It is also possible to determine the reason for ther error by using the \fkeyw{iostat} option. A variable is associated with the the \fkeyw{iostat}-option and when an error occurs the variable will be assigned an with an error code. The following code shows an example of how this option can be used:

\begin{lstlisting}
integer :: ierr
...
open(..., iostat=ierr)
...
read(unit=xx, iostat=ierr)
...
close(unit=xx, iostat=ierr)
\end{lstlisting}

If no error was encountered the associated variable with be assigned an error code of 0. Other codes can be:

\begin{itemize}
\item -2, End of record condition occurs in non-advancing I/O.
\item -1, End of file condition. 
\item >1, Standardised list of fortran error codes.
\end{itemize}


When the \fkeyw{iostat}-option is used all default error messages will be suppressed and code execution will continue. The best use of this option is combined with the \fkeyw{err}-option to provide clear error messages to users as show in the following code example:

\begin{lstlisting}
subroutine read_from_file(...)
    ...
    integer :: ierr
    ...
    read(unit=xx, err=101, iostat=ierr)
    ...
    return

101 print*, 'Error ', ierr, ' reading file.' 
    return
    
end subroutine
\end{lstlisting}

%---------------------------------------------------------------------
\section{String manipulation}
%---------------------------------------------------------------------

There are several ways of manipulating strings in Fortran. Strings can be concatenated with the operator, \foper{//}, as shown in the following example:

\begin{lstlisting}
c1 = 'Hej '
c2 = 'hopp!'
c = c1 // c2 ! = 'Hej hopp!'
\end{lstlisting}

Fortran does not have dynamic strings, so the size of the resulting string must be large enough for the concatenated string.

Substrings can be extracted using a syntax similar to the syntax used when indexing arrays.

\begin{lstlisting}
c3 = c(5:8) ! Contains the string 'hopp'
\end{lstlisting}

A common task in many codes is the conversion of numbers to and from strings. Fortran does not have any explicit functions these type of conversions, instead the the \fkeyw{read} and \fkeyw{write} statements can be used together with strings to accomplish the same thing. By replacing the file unit number with a character string variable, the string can be read from and written to using \fkeyw{read} and \fkeyw{write} statements.

To convert a floating point value to a string the following code can be used.

\begin{lstlisting}
character(255) :: mystring
real(8) :: myvalue
value = 42.0
write(mystring,'(G15.4)') value
! mystring now contains '      5.676'
\end{lstlisting}

To convert a value contained in string to a floating point value the read-statement is used.

\begin{lstlisting}
character(255) :: mystring
real(8) :: myvalue
mystring = '42.0'
read(mystring,*) myvalue
! myvalue now contains 42.0
\end{lstlisting}

A more complete example is shown in the following listing:

\lstinputlisting{source/strings/strings2.f90}

The program produces the following output.

\begin{lstlisting}
5
42
\end{lstlisting}

%---------------------------------------------------------------------
\section{Array features}
%---------------------------------------------------------------------

%---------------------------------------------------------------------
\subsection{Forall- and Where-statements}
%---------------------------------------------------------------------

Fortran has added a number of new loop-statements. The \fkeyw{forall}-statement has been added to optimise nested loops for execution on multiprocessor machines. The syntax is:

\begin{fsyntax}
\textbf{forall} (index = lower:upper [,index = lower:upper])\\
\ftab [body]\\
\textbf{end forall}
\end{fsyntax}

The following example shows how a \fkeyw{do}-statement can be replaced with a \fkeyw{forall}-statement.

\begin{lstlisting}[texcl,escapechar=\%]
do i=1,n
    do j=1,m
        A(i,j)=i+j
    end do
end do

! Is equivalent with

forall(i=1:n, j=1:m)
    A(i,j)=i+j
end forall
\end{lstlisting}

Another statement optimised for multiprocessor architectures is the \fkeyw{where}-statement. With this statement conditional operations on an array can be achieved efficiently. The syntax comes in two versions.

\begin{fsyntax}
\textbf{where} (logical-array-expr)\\
\ftab array-assignments\\
\textbf{end where}
\end{fsyntax}

and

\begin{fsyntax}
\textbf{where} (logical-array-expr)\\
\ftab array-assignments\\
\textbf{else where}\\
\ftab array-assignments\\
\textbf{end where}
\end{fsyntax}

The usage of the \fkeyw{where}-statement is best illustrated with an example.

\begin{lstlisting}
where (A>1)
    B = 0
else where
    B = A
end where
\end{lstlisting}

In this example two arrays with the same size are used in the \fkeyw{where}-statement. In this case the values in the \fvar{B} array are assigned 0 when an element in the A array is larger than 1 otherwise the element in \fvar{B} is assigned the same value as in the \fvar{A} array.

%---------------------------------------------------------------------
%\subsection{Pure procedures}
%---------------------------------------------------------------------

%---------------------------------------------------------------------
\subsection{Elemental procedures}
%---------------------------------------------------------------------


%---------------------------------------------------------------------
\section{Pointers}
%---------------------------------------------------------------------

\lstinputlisting{source/pointers/main.f90}

%---------------------------------------------------------------------
\section{System functions}
%---------------------------------------------------------------------

%---------------------------------------------------------------------
\subsection{C Interoperability}
%---------------------------------------------------------------------

\lstinputlisting{source/c_interop/main.f90}

%---------------------------------------------------------------------
\subsection{Access to computing environment}
%---------------------------------------------------------------------

%---------------------------------------------------------------------
\section{Object-oriented programming}
%---------------------------------------------------------------------

In procedural programming, data and subroutines are treated separately. Subroutines operate on provided data structures and variables. In object-oriented programming data and subroutines are combined into objects. Objects in numerical computing can be  be different matrix types, particles, vectors or solvers. The major benefits are that the actual data structures used in the implementation of an object can be hidden from the user of the object, enabling the developer of an object to improve the implementation without affecting users of the objects (encapsulation). Another important feature of object-oriented programming is the ability to inherit and extend functionality of objects (inheritance). This enables user of object and developers to extend and modify functionality of existing objects, relying on functionality of the parent object. 

In Fortran 2003 object-oriented features where added to the language, making Fortran almost as feature rich as other more recent languages. Most modern Fortran compilers today support the object-oriented features added in the 2003 standard, enabling developers to implement truly object-oriented numerical applications in Fortran.

The functionality and data structures of objects are defined in classes in most programming object-oriented languages. Classes can be seen as templates for objects. When an object is to be created the class is used as the template for the new object. Created objects are also called instances of a class.

In Fortran the object-oriented features are implemented by extending the derived datatype concepts of fortran 90. A derived datatype now has a \fkeyw{contains}-section in which the procedures of the objects are specified. Derived datatypes are now also by definition objects or instances of the derived type.

To illustrate the concepts, a simple particle object is defined as an object in Fortran. The particle object is defined as a derived data type in a module, particles. To eliminate any name clashes when creating new objects of this type, the derived type is given the name, \fvar{particle\_class}. This is also fits the object-oriented model of derived type being equivalent to classes. The initial class definition then becomes:

\begin{lstlisting}
module particles

    implicit none

    type particle_class
        real :: pos(3)
        real :: vel(3)
    end type particle_class

end module particles
\end{lstlisting}

The definition currently corresponds exactly to a derived data type in Fortran and can be used as such as well. To create instance of the \fvar{particle\_class} is equivalent to creating a variable of a specified derived data type:

\begin{lstlisting}
use particles
...
type(particle_class) :: particle
\end{lstlisting}

To access the variables of the instance the \fkeyw{\%} operator is used. In the following examplen the \fvar{pos}-variable is assigned the coordinate $(0,0,0)$.

\begin{lstlisting}
particle % pos = (/ 0.0, 0.0, 0.0 /)
\end{lstlisting}

However, accessing the instance variables goes against the principles of object-oriented programming, where one of the more important aspects is data encapsulation and hiding the internal workings of the objects. How do we use the new Fortran features to prevent the need to access the data structures directly? The first aspect to cover is initialisation of the instance  variables. To do this, a method, \fname{init}, will be added to our class. First, a \fkeyw{contains}-section with a procedure specification for the \fname{init} subroutine. As it is not allowed to have duplicate subroutine names in a module we assign an actual implementation subroutine using the => operator. Please note that no parameters lists are specified in this declaration. Secondly, to distinguish the member variables from other variables in the class implementation, as well as preventing name collisions with the names of the access methods, the member variables are prefixed with \fname{m\_}. The complete class then becomes:

\begin{lstlisting}
module particles

    implicit none

    type particle_class
        real :: m_pos(3)
        real :: m_vel(3)
    contains
        procedure :: init => particle_init
    end type particle_class
    ...
\end{lstlisting}

This declaration states that the class, \fkeyw{particle\_class}, has a member subroutine with the name, \fname{init}, defined later in the source code by the subroutine \fname{particle\_init}. To complete the class definition, the subroutine \fname{particle\_init} have to be added to the \fname{particles} module. 

All member subroutines can take a first dummy argument with containing a reference to the actual instance. This will be used to enable us to do our initialisation on the actual data structures of the instance. The \fname{init}-subroutine of our particle class then becomes:

\begin{lstlisting}
module particles

...

contains

subroutine particle_init(this)

    class(particle_class) :: this

    this % m_pos = (/0.0, 0.0, 0.0/)
    this % m_vel = (/0.0, 0.0, 0.0/)

end subroutine particle_init

end module particles
\end{lstlisting}

In the previous example the \fname{particle\_init}-subroutine has a dummy variable, \fvar{this}, which is used to access the actual instance variables of the class. This variable is automatically passed to the routine by the compiler. It is now possible to initialise our newly created instance without accessing the member variables directly. The code to create a new object and initialise it data structure then becomes:

\begin{lstlisting}
type(particle_class) :: particle

call particle % init
\end{lstlisting}

The dummy argument, \fname{init}, can be left out of the call to \fname{init}. The same concept of passing the instance variable as a argument in the class definition can also be found in Python, where a special variable, \fkeyw{self} is used in the member subroutines. Please also note that we call the method with the \fname{init} and not the actual implemented subroutine, \fname{particle\_init}. This enables us to have the same class interface in several classes.

%---------------------------------------------------------------------
\subsection{Access methods}
%---------------------------------------------------------------------

The class now has the ability to initialise its data variables. However, we don't have any ways of accessing the variables without accessing them directly. To solve this we have to add special methods for accessing internal class variables. First, methods for assigning the position and velocity of the particle is added in the class declaration:

\begin{lstlisting}
    ...
    type particle_class
        real :: m_pos(3)
        real :: m_vel(3)
    contains
        procedure :: init => particle_init
        procedure :: set_position => particle_set_position <*\hladded*>
        procedure :: set_velocity => particle_set_velocity <*\hladded*>
    end type particle_class
    ...
\end{lstlisting}

When this has been done the implementations of these subroutines are added in the \fkeyw{contains}-section of the \fname{parcticles}-modules:
    
\begin{lstlisting}
contains

...

subroutine particle_set_position(this, x, y ,z)

    class(particle_class) :: this
    real :: x, y, z

    this % m_pos = (/x, y, z/)

end subroutine particle_set_position

subroutine particle_set_velocity(this, vx, vy ,vz)

    class(particle_class) :: this
    real :: vx, vy, vz

    this % m_vel = (/vx, vy, vz/)

end subroutine particle_set_velocity

end module particles        
\end{lstlisting}

In the same way as in the \fname{init}-subroutine, \fname{this}, is used to access the member variables of the class instance. 

It is now possible to assign values to our instances without directly accessing the internal member variables as shown in the following code:

\begin{lstlisting}
call particle % set_position(1.0, 1.0, 1.0)
call particle % set_velocity(1.0, 1.0, 1.0)
\end{lstlisting}

To retrieve values from the instance, 2 additional subroutines are needed, \fname{get\_position} and \fname{get\_velocity} are added to the class definition and implementation.

\begin{lstlisting}
module particles

    implicit none

    type particle_class
        real :: m_pos(3)
        real :: m_vel(3)
    contains
        procedure :: init
        procedure :: set_position => particle_set_position
        procedure :: set_velocity => particle_set_velocity
        procedure :: get_position => particle_get_position <*\hladded*>
        procedure :: get_velocity => particle_get_velocity <*\hladded*>
    end type particle_class

contains

...

subroutine particle_get_position(this, x, y ,z)

    class(particle_class) :: this
    real, intent(out) :: x, y, z

    x = this % m_pos(1)
    y = this % m_pos(2)
    z = this % m_pos(3)

end subroutine particle_get_position

subroutine particle_get_velocity(this, vx, vy ,vz)

    class(particle_class) :: this
    real, intent(out) :: vx, vy, vz

    vx = this % m_vel(1)
    vy = this % m_vel(2)
    vz = this % m_vel(3)

end subroutine particle_get_velocity

...
\end{lstlisting}

It is now possible to access the instance variables using these subroutines as shown in the following example:

\begin{lstlisting}
real :: x, y, z
...
call particle % get_position(x, y, z)
\end{lstlisting}

It is also possible to use functions to access the instance variables, as shown in the following code:

\begin{lstlisting}
real :: x, y, z
...
x = particle % x()
\end{lstlisting}

\fname{x()} is a Fortran function member of \fname{particle\_class}.

%---------------------------------------------------------------------
\subsection{Pretty printing}
%---------------------------------------------------------------------

Other functionalities that could be integrated into the class is the ability to pretty print its state variables. Consider the following subroutine:

\begin{lstlisting}
subroutine particle_print(this)

    class(particle_class) :: this

    print*, 'Particle position'
    print*, '-----------------'
    write(*, '(3G10.3)') this % m_pos(1), this % m_pos(2), this % m_pos(3)

    print*, ''

    print*, 'Particle velocity'
    print*, '-----------------'
    write(*, '(3G10.3)') this % m_vel(1), this % m_vel(2), this % m_vel(3)

end subroutine particle_print
\end{lstlisting}

This subroutine will enable a user of the class to easily print the instance variables in a nice formatted way without actually accessing the instance variables:

\begin{lstlisting}
call particle % print
\end{lstlisting}

This code would give the following output:

\cmdmode

\begin{lstlisting}
 Particle position
 -----------------
  1.00      1.00      1.00    
 
 Particle velocity
 -----------------
  1.00      1.00      1.00    
\end{lstlisting}

An extension of this could be to provide subroutines for reading and writing the object instance to a file.

%---------------------------------------------------------------------
\subsection{Restricting access}
%---------------------------------------------------------------------

In the previous code examples the internal state variables where encapsulated using access methods. However, it is still possible for a user of the instance to access the member variable. This could let to users of the instance modifying the variables directly either willingly or by mistake, which could lead complicated bugs. To solve this Fortran enables the classes to mark these instance variable as private preventing access to them. Adding a \fkeyw{private}-directive in the class declaration hides these variables from the users of the instance, as shown in the following code:

\fmode

\begin{lstlisting}
type particle_class
private <*\hladded*>
    real :: m_pos(3)
    real :: m_vel(3)
contains
\end{lstlisting}

Assigning these variables as shown in the following example,

\begin{lstlisting}
particle % m_pos(1) = 0.0
\end{lstlisting}

will produce the following compilation error:

\cmdmode

\begin{lstlisting}
    particle % m_pos(1) = 0.0
                  1
Error: Component 'm_pos' at (1) is a PRIVATE component of 'particle_class'
\end{lstlisting}

\fmode

This means that using the \fkeyw{private} in class declaration effectively will prevent any users mistakenly accessing the private instance variables of any classes. 

The \fkeyw{private} before the variable declaration will make all variables private. It is also possible to selectively make variables public or private by removing the private declaration and adding the variable attribute \fkeyw{private} to the variable declaration as below:

\begin{lstlisting}
type particle_class
    real, private :: m_pos(3)
    real :: m_vel(3)
\end{lstlisting}

In this code \fname{m\_pos} is private and \fname{m\_vel} is public as all variables are public by default.

In a similar way it is possible to prevent access to member subroutines or functions using the \fkeyw{private}- or \fkeyw{public}-attributes on for each subroutine declaration as in the following example:

\begin{lstlisting}
type particle_class
private
    real :: m_pos(3)
    real :: m_vel(3)
contains
    procedure :: init => particle_init
    procedure :: set_position => particle_set_position
    procedure :: set_velocity => particle_set_velocity 
    procedure :: get_position => particle_get_position
    procedure :: get_velocity => particle_get_velocity
    procedure :: print => particle_print
    procedure, private :: setup => particle_setup <*\hladded*>
end type particle_class
\end{lstlisting}

Here a private method \fname{setup} has been added that can only be called for member subroutines of the \fname{particle}-class.

%---------------------------------------------------------------------
\subsection{Extending existing classes}
%---------------------------------------------------------------------

Classes in Fortran can be extended using the special attribute \fkeyw{extends} in its type definition. In the following example the previous \fname{particle\_class} is extended by the \fname{sphere\_particle\_class} to handle a spherical particle. 

\begin{lstlisting}
type, extends(particle_class) :: sphere_particle_class
private
    real :: m_radius
contains
    procedure :: set_radius => sphere_set_radius
    procedure :: get_radius => sphere_get_radius
end type sphere_particle_class
\end{lstlisting}

In the type definition a private member variable, \fname{m\_radius} and its access methods, \fname{set\_radius} and \fname{get\_radius} are added. The access methods are similar to the access method of the \fname{particle\_class}.

\begin{lstlisting}
subroutine sphere_set_radius(this, r)

    class(sphere_particle_class) :: this
    real :: r

    this % m_radius = r

end subroutine sphere_set_radius

real function sphere_get_radius(this)

    class(sphere_particle_class) :: this

    get_radius = this % m_radius

end function sphere_get_radius
\end{lstlisting}

The above type definition, will override the existing \fname{init}-routine from the \fname{particle\_class}. This means that the existing initialisation routine will not be called. To solve this, the \fname{sphere\_init}-routine needs to call the existing \fname{particle\_init}-routine from the extended class. The \fname{sphere\_init} initalisation routine is shown below:

\begin{lstlisting}
subroutine sphere_init(this)

    class(sphere_particle_class) :: this
    
    ! --- Calling inherited init routine.

    call this % particle_class % init() 

    this % m_radius = 1.0

end subroutine sphere_init
\end{lstlisting}

The format of a call to a inherited routine is:

\begin{fsyntax}
call [instance reference] \% [type definition name] \% [routine name]
\end{fsyntax}

%---------------------------------------------------------------------
\subsection{Overriding class methods}
%---------------------------------------------------------------------

In the previous chapter a first example of 































%\include{chapter_fortran_cpp}
%%---------------------------------------------------------------------
%---------------------------------------------------------------------
\chapter{Introduction to Parallel programming}
%---------------------------------------------------------------------
%---------------------------------------------------------------------

\section{Threads}

\section{MPI}

\section{OpenMP}

%---------------------------------------------------------------------
%---------------------------------------------------------------------
\chapter{Fortran and Python}
%---------------------------------------------------------------------
%---------------------------------------------------------------------

When developing numerical codes today, it is more and more important to be able to combine benefits form many programming languages. One language that have gained popularity in numerical computing i Python. Python is a powerful dynamic scripting language, which is easy to use and combined with the Scipy toolkit it can provide a environment very similar to MATLAB. Python also supports the development of a multitude of different kinds of applications ranging from graphical user interfaces to web interfaces and web services. 

By combining Fortran and Python, the performance of Fortran can be combined with the easy of use and flexibility of Python. This chapter describes a method for combining these languages using a special tool, f2py.

%---------------------------------------------------------------------
\section{Python extension modules}
%---------------------------------------------------------------------

In addition to implement Python modules in Python, modules can also be implemented in C using a special API, which usually is found in the libpython library. To illustrate how an extension module is developed, a simple sum function will be implemented in an extension module, calcualtions, using the Python extension API.

A typical Python extension module requires 3 parts:

\begin{itemize}
\item Exported functions defined using the Python extension API.
\item Module function table declaring all exported functions in the module.
\item Module initialisation function for initialising the module and function table.
\end{itemize}

An exported function must be declared in a format that can be understood by Python. Our exported sum function is declared as shown in the following code:

\cmode

\begin{lstlisting}
static PyObject* 
sum(PyObject *self, PyObject *args)
\end{lstlisting}

The left side declares the return values from the function. This declaration must always be there even if the function does not return anything. Next, the sum function is declared. All exported functions have the same arguments. \cname{PyObject* self} is a pointer to the module instance that the function belongs to. The second argument is a special Python object containing the arguments that the function is called with.

Next, the input arguments must be parsed. This is done using the \cname{PyArg\_ParseTuple}-function in the Python API. This function parses the input arguments, \cvar{args}, for the required parameters. If no match is found the function returns NULL, which will trigger an exception in the Python interpreter. If a match is found the function will assign values to provided C-variables. Argument parsing for our sum function is shown in the following code:

\begin{lstlisting}
// C variables that will contain input values

double a;
double b;

// Parse input arguments

if (!PyArg_ParseTuple(args, "dd", &a, &b))
    return NULL;
\end{lstlisting}

First, variables, \cvar{a} and \cvar{b}, are declared for storing the actual input arguments. The \cname{PyArg\_ParseTuple}-function takes the input parameter, args, and processes this according to a signature string describing the required Python arguments. Our function sum takes two double values as input arguments, the signature string for this is ''dd''. 

Now we have all input data, so now we do our actual computation in C:

\begin{lstlisting}
double c = a + b;
\end{lstlisting}

To be able to use the computed value in Python it has to be converted to a PyObject. This can be done using the function \cname{Py\_BuildValue}. This function is similar to the \cname{PyArg\_ParseTuple}-function as it also uses the signature string to define what Python-datatypes to create. This is used in the last part of the \cname{sum}-function to return a Python-datatype.

\begin{lstlisting}
return Py_BuildValue("d", c);
\end{lstlisting}

The complete sum function then becomes:

\begin{lstlisting}
static PyObject* 
sum(PyObject *self, PyObject *args)
{
    double a;
    double b;

    // Parse input arguments

    if (!PyArg_ParseTuple(args, "dd", &a, &b))
        return NULL;

    // Do our computation

    double c = a + b;

    // Return the results

    return Py_BuildValue("d", c);
}
\end{lstlisting}

To be able to compile this function as an extension module, a function table and module initialisation have to be added. The additional code required is shown below:

\begin{lstlisting}
// Module function table.

static PyMethodDef
module_functions[] = {
    { "sum", sum, METH_VARARGS, "Calculate sum." },
    { NULL }
};

// Module initialisation

void
initcext(void)
{
    Py_InitModule3("cext", module_functions, "A minimal module.");
}
\end{lstlisting}

To build the extension module, the \fname{setuptools} module in NumPy is used. The following \fname{setup.py} is used to build the extension module:

\pymode

\begin{lstlisting}
from numpy.distutils.core import setup, Extension

setup(
	ext_modules = [
		Extension("cext",
			sources=["calculations_c.c"]),
	]
)\end{lstlisting}

Building the module from the command line is then done using the following command:

\cmdmode

\begin{lstlisting}
> python setup.py build
running build
running build_ext
building 'calculations' extension
gcc -fno-strict-aliasing -I/Users/lindemann/anaconda/include -arch x86_64 -DNDEBUG -g -fwrapv -O3 -Wall -Wstrict-prototypes -I/Users/lindemann/anaconda/include/python2.7 -c calculations.c -o build/temp.macosx-10.5-x86_64-2.7/calculations.o
gcc -bundle -undefined dynamic_lookup -L/Users/lindemann/anaconda/lib -arch x86_64 -arch x86_64 build/temp.macosx-10.5-x86_64-2.7/calculations.o -L/Users/lindemann/anaconda/lib -o build/lib.macosx-10.5-x86_64-2.7/calculations.so
\end{lstlisting}

The following example shows how the module can be used like any other module in Python:

\pymode

\begin{lstlisting}
>>> import cext
>>> dir(cext)
['__doc__', '__file__', '__name__', '__package__', 'sum']
>>> s = cext.sum(2.0, 3.0)
>>> print s
5.0
>>>
\end{lstlisting}

%---------------------------------------------------------------------
\section{Integrating Fortran in extension modules}
%---------------------------------------------------------------------

To integrate Fortran in a Python extension module, requires us to compile and link Fortran code into the extension module. To illustrate this, the example in the previous section will be modified to call a fortran subroutine to perform the computation. To link a Fortran routine with a C, the calling convention in Fortran must be adapted to C. In the following example the \fname{iso\_c\_binindg}, \fname{value} and \fname{bind} is used to define a Fortran routine that uses the C calling convention and C datatypes to make the linking easier:

\fmode

\begin{lstlisting}
subroutine forsum(a, b, c) bind(C, name='forsum')

	use iso_c_binding

	real(c_double), value :: a, b
	real(c_double)        :: c

	c = a + b

end subroutine forsum
\end{lstlisting}

The code in the Python extension module is now updated to call the Fortran routine as shown below:

\cmode

\begin{lstlisting}
static PyObject* 
sum(PyObject *self, PyObject *args)
{
    double a;
    double b;
    double c;

    // Parse input arguments

    if (!PyArg_ParseTuple(args, "dd", &a, &b))
        return NULL;

    // Do our computation

    forsum(a, b, &c);

    // Return the results

    return Py_BuildValue("d", c);
}
\end{lstlisting}

The reason for the \& operator is to pass the \cname{c}-variable as a reference to the Fortran routine.

To build the modified extension module, the Fortran routine must be compiled separately and then provided as a \fname{.o}-file to the \fname{setup.py} script:

\pymode

\begin{lstlisting}
from numpy.distutils.core import setup, Extension

setup(
	ext_modules = [
		Extension("fext",
			sources=["fext.c"],
			extra_objects=["forsum.o"])
	]
)
\end{lstlisting}

It is also possible to transfer matrices between Fortran and Python. However, it requires even more complicated binding code. Instead of doing this by hand, special tools can be used to automatically generate the binding code for us as well as enabling us to use NumPy arrays to transfer matrices between Fortran and Python in an efficient way.

%---------------------------------------------------------------------
\section{F2PY}
%---------------------------------------------------------------------

F2PY is a tool developed by Pearu Peterson that parses Fortran code, generates Python wrapper code and compiles it as a Python extension module. F2PY automatically create wrapper code for Fortran arrays, so that NumPy arrays can be passed directly to the generated functions. 

To illustrate the process of generating an extension module with F2PY the following simple Fortran routine will be wrapped as a module:

\fmode

\begin{lstlisting}
subroutine simple(a,b,c)

	real, intent(in) :: a, b
	real, intent(out) :: c

	c = a + b

end subroutine simple
\end{lstlisting}

To be able to use F2PY effectively it is important that the \fname{intent}-attribute is used on the subroutine arguments. If not specified, F2PY, will treat all subroutine parameters as input-variables and no output parameters can be passed back to the the calling Python routine.

To create a Python module from the source code we execute the \cli{f2py}-command on the command line as show below:

\cmmode

\begin{lstlisting}
> f2py -m fortmod -c simple.f90
...
3n535b8krwsz88vl8bm0000gn/T/tmp5STblc/src.macosx-10.5-x86_64-2.7/fortranobject.o /var/folders/w6/1zqjp3n535b8krwsz88vl8bm0000gn/T/tmp5STblc/simple.o -L/opt/local/lib/gcc49/gcc/x86_64-apple-darwin14/4.9.1 -L/Users/lindemann/anaconda/lib -lgfortran -o ./fortmod.so
Removing build directory /var/folders/w6/1zq...
\end{lstlisting}

In the build directory there should now be a \fname{fortmod.so} or a \fname{fortmod.pyd} depending on the platform used.

The new module is loaded and used as shown in the following example:

\pymode

\begin{lstlisting}
>>> import fortmod
>>> print fortmod.simple(2.0, 3.0)
5.0
\end{lstlisting}

F2PY will automatically generate built-in documentation in the module. To display this documentation the \pyvar{\_\_doc\_\_} property is used, as shown in the following example:

\begin{lstlisting}
>>> print fortmod.__doc__
This module 'fortmod' is auto-generated with f2py (version:2).
Functions:
  c = simple(a,b)
.
>>> print fortmod.simple.__doc__
c = simple(a,b)

Wrapper for ``simple``.

Parameters
----------
a : input float
b : input float

Returns
-------
c : float
\end{lstlisting}

As show above, F2PY generates documentation both for the generated module as well as for the individual functions.

Already now it is clear that using F2PY is significantly easier that hand-coding Python wrappers for Fortran. F2PY takes care of all the steps. 

%---------------------------------------------------------------------
\subsection{Passing arrays}
%---------------------------------------------------------------------

F2PY will automatically handle conversion of NumPy arrays when calling a Fortran extension module. However, it is important to note that NumPy by default uses C ordered arrays. These will be automatically converted to Fortran ordered arrays. For smaller arrays the overhead is not so large, but for large arrays the overhead can be significant. To avoid the automatic conversion, NumPy arrays should be created with the \pyvar{order='F'} option in the array constructor, as shown in the following example:

\pymode

\begin{lstlisting}
A = ones((10,10), 'f', order='F')
\end{lstlisting}

Using this option will pass the allocated memory for the NumPy array directly to the Fortran routine without conversion.

%---------------------------------------------------------------------
\subsection{A more complete example - Matrix multiplication}
%---------------------------------------------------------------------

To illustrate the use of arrays in a Fortran extension module we create a Fortran subroutine that takes two input arrays and returns the matrix multiplication of these two arrays, The first version of the function is shown below:

\fmode

\begin{lstlisting}
! A[r,s] * B[s,t] = C[r,t]
subroutine matrix_multiply(A,r,s,B,t,C)
	integer :: r, s, t
	real, intent(in) :: A(r,s)
	real, intent(in) :: B(s,t)
	real, intent(out) :: C(r,t)

	C = matmul(A,B)
end subroutine matrix_multiply
\end{lstlisting}

Input variables \fvar{r, s, t} define the sizes of the incoming matrices. We use the Fortran attributes \fvar{intent(in)} and \fvar{intent(out)} to tell F2PY what should be treated as an input variable or an output variable. Creating a Fortran extension module with F2PY on the above routine produces the following corresponding Python routine (from the generated documentation):

\cmdmode

\begin{lstlisting}
c = matrix_multiply(a,b,[r,s,t])

Wrapper for ``matrix_multiply``.

Parameters
----------
a : input rank-2 array('f') with bounds (r,s)
b : input rank-2 array('f') with bounds (s,t)

Other Parameters
----------------
r : input int, optional
    Default: shape(a,0)
s : input int, optional
    Default: shape(a,1)
t : input int, optional
    Default: shape(b,1)

Returns
-------
c : rank-2 array('f') with bounds (r,t)
\end{lstlisting}

\pymode

We can see in the documentation that the syntax of the Python routine is:

\begin{lstlisting}
c = matrix_multiply(a,b,[r,s,t])
\end{lstlisting}

The Fortran output argument, \pyvar{c} is returned on the left side and the input arguments, \pyvar{a, b} are input parameters to the Fortran routine. Please note that the size input parameters will be provided by the generated function and are not required when calling the routine from Python.

The created extension module can be uses from Python as shown in the following code:

\begin{lstlisting}
from numpy import *
from fortmod import *

A = ones((6,6), 'f', order='F') * 10.0
B = ones((6,6), 'f', order='F') * 20.0

C = matrix_multiply(A, B)

print C
\end{lstlisting}

Output from the Python code is:

\cmdmode

\begin{lstlisting}
[[ 1200.  1200.  1200.  1200.  1200.  1200.]
 [ 1200.  1200.  1200.  1200.  1200.  1200.]
 [ 1200.  1200.  1200.  1200.  1200.  1200.]
 [ 1200.  1200.  1200.  1200.  1200.  1200.]
 [ 1200.  1200.  1200.  1200.  1200.  1200.]
 [ 1200.  1200.  1200.  1200.  1200.  1200.]]
\end{lstlisting}

Output variables, \pyvar{c}, from Fortran will be automatically created. It is not possible to reference data in an already existing \pyvar{c} array as shown in the following example:

\pymode

\begin{lstlisting}
A = ones((6,6), 'f', order='F') * 10.0
B = ones((6,6), 'f', order='F') * 20.0
C = zeros((6,6), 'f', order='F')

print "id of C before multiply =",id(C)

C = matrix_multiply(A, B)

print "id of C after multiply =",id(C)
\end{lstlisting}

In this example, an array \pyvar{C} is created before the call to our Fortran routine. The id or memory location is queried using the \pymethod{id()} and displayed before and after the call. The output is:

\cmdmode

\begin{lstlisting}
id of C before multiply = 4299985824
id of C after multiply = 4340070160
\end{lstlisting}

The \pyvar{C} array is apparently overwritten. This is due to how the Python language is designed. An euqality operator will replace the reference to the first \pyvar{C} instance with a new instance. The next section covers how to pass variables that can be modified by Fortran.

%---------------------------------------------------------------------
\subsection{Matrix mulitplication with modifiable output variables}
%---------------------------------------------------------------------

If the Fortran extension module should be able to modify the contents of the incoming arrays, the \fvar{intent(inout)} attribute must be used. This tells F2PY to generate code that handles this. Our modified matrix multiplication subroutine then becomes:

\fmode

\begin{lstlisting}
! A[r,s] * B[s,t] = C[r,t]
subroutine matrix_multiply2(A,r,s,B,t,C)
	integer :: r, s, t
	real, intent(in) :: A(r,s)
	real, intent(in) :: B(s,t)
	real, intent(inout) :: C(r,t)

	C = matmul(A,B)
end subroutine matrix_multiply2
\end{lstlisting}

The only difference is the \fvar{intent(inout)} attribute on the \fvar{C} array declaration. However, the generated Python routine is quite different:

\cmdmode

\begin{lstlisting}
matrix_multiply2(a,b,c,[r,s,t])

Wrapper for ``matrix_multiply2``.

Parameters
----------
a : input rank-2 array('f') with bounds (r,s)
b : input rank-2 array('f') with bounds (s,t)
c : in/output rank-2 array('f') with bounds (r,t)

Other Parameters
----------------
r : input int, optional
    Default: shape(a,0)
s : input int, optional
    Default: shape(a,1)
t : input int, optional
    Default: shape(b,1)
\end{lstlisting}

Now all input parameters are given on the right side. Now it is possible to directly modify the \pyvar{c} variable in the Fortran code and pass any changes back to Python, without copying the data. The memory address of the array is the same as used by the NumPy array in the Python code. The following code shows how to use the modified Fortran extension:

\fmode

\begin{lstlisting}
A = ones((6,6), 'f', order='F') * 10.0
B = ones((6,6), 'f', order='F') * 20.0
C = zeros((6,6), 'f', order='F')

print "id of C before multiply =",id(C)

matrix_multiply2(A, B, C)

print "id of C after multiply =",id(C)

print C
\end{lstlisting}

For this code to work it is now required to create the array, \pyvar{C}, before calling the Fortran extension. This is due to the fact that the memory area for the array needs to exist before the call as the pointer to the array is passed directly to the Fortran code. The output of the Python code is shown below:

\cmdmode

\begin{lstlisting}
id of C before multiply = 4302082976
id of C after multiply = 4302082976
[[ 1200.  1200.  1200.  1200.  1200.  1200.]
 [ 1200.  1200.  1200.  1200.  1200.  1200.]
 [ 1200.  1200.  1200.  1200.  1200.  1200.]
 [ 1200.  1200.  1200.  1200.  1200.  1200.]
 [ 1200.  1200.  1200.  1200.  1200.  1200.]
 [ 1200.  1200.  1200.  1200.  1200.  1200.]]
\end{lstlisting}

From the output, we can see that the memory of the array is the same before and after the call to the Fortran extension module.




























%---------------------------------------------------------------------
%---------------------------------------------------------------------
\chapter{Managing Fortran projects}
%---------------------------------------------------------------------
%---------------------------------------------------------------------

When projects get larger, compiling and maintenance issues can become quite complex. Many projects also needs to be able to build on different platforms and operating environments. Using simple shell scripts often work when the projects are small. However, shell scripts often lack the ability to handle compilation dependencies between source files, requiring a complete rebuild of the project for each modification. Tools such as CMake and Make can solve many of these problems efficiently.


% --------------------------------------------------------------------
\section{Make}
% --------------------------------------------------------------------

Make is a tool that can build software according to special rules defined in a makefile. Make automatically handles dependencies between source files and only rebuilds parts of the software that are affected by the change.

A makefile consists of a series of rules, dependencies and actions. The general syntax for a makefile is:

\begin{msyntax}
target: [dependencies]\\
\ftab system command
\end{msyntax}

A simple rule to compress a file, myfile.txt, is shown below:

\mmode

\begin{lstlisting}
myfile.gz: myfile.txt
	cat myfile.txt | gzip > myfile.gz
\end{lstlisting}

In this example a rule, \mfname{myfile.gz}, is created to compress the file. The rule depends on the file, \mfname{myfile.txt}. The action for compressing the file is shown in the second row. When you run the make command in the directory the following output is shown:

\cmdmode

\begin{lstlisting}
$ ls
Makefile	myfile.txt
$ make
cat myfile.txt | gzip > myfile.gz
$ ls
Makefile	myfile.gz	myfile.txt
\end{lstlisting} 

If make is run again it will recognise that the \mfname{myfile.txt} has not been changed and not execute the action again.

\begin{lstlisting}
$ make
make: 'myfile.gz' is up to date.
\end{lstlisting}

If the \mfname{myfile.txt} is changed, make will recognise this and run the specified action again.

\begin{lstlisting}
$ touch myfile.txt
$ make
cat myfile.txt | gzip > myfile.gz
$ 
\end{lstlisting}

% --------------------------------------------------------------------
\subsection{Compiling code with make}
% --------------------------------------------------------------------

Using these rules a build system for compiling source code can be implemented. To compile a simple Fortran application the are some steps that needs to be done:

\begin{enumerate}
\item Compile the source files to an object files (.o).
\item Link the object files to an executable.
\end{enumerate}

For a single source file 2 rules are needed. One rule for compiling the source file to an object file and one rule for linking the object file to an executable. A simple makefile for a single source file application is shown below:

\mmode

\begin{lstlisting}
myprog: myprog.o 
	gfortran myprog.o -o myprog

myprog.o: myprog.f90 
	gfortran -c myprog.f90
\end{lstlisting}

In the above example the executable, \mfname{myprog}, depends on the object file \mfname{myprog.o}. The second rule defines how the object file \mfname{myprog.o} is created from the source file, \mfname{myprog.f90}, which is also listed as a dependency for the rule. Running make on this makefile produces the following output:

\cmdmode

\begin{lstlisting}
$ ls
Makefile	myprog.f90
$ make
gfortran -c myprog.f90
gfortran myprog.o -o myprog
\end{lstlisting}

Make first creates the \mfname{myprog.o} object file as this is a dependency for the creating the \mfname{myprog} executable. Next, the executable, \mfname{myprog}, is created by using the gfortran compiler to create an executable from the object file. When running make again, make will check for modifications and only execute actions if necessary. 

Using make on a single source file is perhaps not the most useful thing. However, when compiling multiple files using make becomes more useful. To extend our above example to multiple source files we add the needed dependencies to the rule for building the executable, \mfname{myprog}. We also need an additional rule for building our additional sourcefile, \mfname{mymodule.f90}.

\mmode

\begin{lstlisting}
myprog: myprog.o mymodule.o
	gfortran myprog.o mymodule.o -o myprog

myprog.o: myprog.f90 
	gfortran -c myprog.f90

mymodule.o: mymodule.f90
	gfortran -c mymodule.f90
\end{lstlisting}

The interesting happens when the \mfname{mymodule.f90} file is updated:

\cmdmode

\begin{lstlisting}
$ touch mymodule.f90 
$ make
gfortran -c mymodule.f90
gfortran myprog.o mymodule.o -o myprog
\end{lstlisting}

Make detects the change in the \mfname{mymodule.f90} file and only compiles this file. As the \mfname{myprog.f90} was not updated the existing object file can be reused. This is why it is a good idea to use make in large projects. Modifying a single source file in a large application will only rebuild what is needed to satisfy the dependencies.

% --------------------------------------------------------------------
\subsection{Fortran 90 Module dependencies}
% --------------------------------------------------------------------

One problem compiling Fortran 90 code and modules is module dependencies. When compiling a module the compiler creates \mfname{.mod}-files which can be compared to automatically generated header files in C. When compiling a module which uses another module the used module must be compiled first, so that the \mfname{.mod}-file is available for the compiler. 

In the following exaple we have a module, \mfname{module\_main.f90}, which uses \mfname{module\_truss.f90}. If we update the previous makefile we get the following makefile:

\mmode

\begin{lstlisting}
myprog: module_main.o module_truss.o
	gfortran module_main.o module_truss.o -o myprog

module_main.o: module_main.f90
	gfortran -c module_main.f90

module_truss.o: module_truss.f90
	gfortran -c module_truss.f90
\end{lstlisting}

Running make produces the following output:

\cmdmode

\begin{lstlisting}
$ make
module_main.f90:3.5:

 use truss
     1
Fatal Error: Can't open module file 'truss.mod' for reading at (1): No such file or directory
make: *** [module_main.o] Error 1
\end{lstlisting}

The compiler complains that it is missing the \mfname{.mod}-file, \mfname{truss.mod}, to be able to compile main module. To solve this an additional dependency, \mfname{module\_truss.o}, is added to the \mfname{module\_main.o} build rule. This means that to build the \mfname{module\_main.o} file the \mfname{module\_truss.o} file must first be build. The updated make file is shown below:

\mmode

\begin{lstlisting}
myprog: module_main.o module_truss.o
	gfortran module_main.o module_truss.o -o myprog

module_main.o: module_main.f90 module_truss.o
	gfortran -c module_main.f90

module_truss.o: module_truss.f90
	gfortran -c module_truss.f90
\end{lstlisting}

Running make again will produce the desired results:

\cmdmode

\begin{lstlisting}
$ make
gfortran -c module_truss.f90
gfortran -c module_main.f90
gfortran module_main.o module_truss.o -o myprog
\end{lstlisting}%$

From the above output it can be seen that make figures out the dependencies and builds the \mfname{module\_truss.f90} first which produces the needed \mfname{truss.mod} which is needed when compiling the \mfname{module\_main.f90} file.

% --------------------------------------------------------------------
\subsection{Using variables in make}
% --------------------------------------------------------------------

To specify explicit commands in the make file rules can make the makefiles difficult to maintain. Too solve this, make supports variables in the same way as in normal bash-scripts. To use the value of a variable in the makefile, the name of the variable is enclosed in \mfname{\$(...)}. In the following example, the variable, \mvar{FC}, is used to specify which compiler that is going to be used. The compiler flags are specified in the \mvar{FFLAGS} variable and the name of the application binary is specified in the \mvar{EXECUTABLE} variable. In this example a special clean rule has been added to clean all build files generated when compiling the application. In the rule the \mvar{EXECUTABLE} is used to make the rule more generic. 

\mmode

\begin{lstlisting}
FC=gfortran
FFLAGS=-c
EXECUTABLE=myprog

$(EXECUTABLE): myprog.o mymodule.o
	$(FC) myprog.o mymodule.o -o myprog

myprog.o: myprog.f90 
	$(FC) $(FFLAGS) myprog.f90

mymodule.o: mymodule.f90
	$(FC) $(FFLAGS) mymodule.f90
	
clean:
	rm -rf *.o *.mod $(EXECUTABLE)
\end{lstlisting}

Running make produces the desired result, but with a more flexible make file.

\cmdmode

\begin{lstlisting}
$ make
gfortran -c myprog.f90
gfortran -c mymodule.f90
gfortran myprog.o mymodule.o -o myprog
\end{lstlisting}

% --------------------------------------------------------------------
\subsection{Internal macros}
% --------------------------------------------------------------------

To create even more generic makefiles and rules, make also has some useful internal macros that can be used. The most important internal macros are:

\begin{center}
\begin{tabular}{ ll }
  \verb|$@| & The target of the current rule executed. \\
  \verb|$^| & Name of all prerequisites \\
  \verb|$<| & Name of the first prerequisite \\
\end{tabular}
\end{center}

\mmode

Using \verb|$^| and \verb|$@| a more generic build rule for linking our application can be created

\begin{lstlisting}
FC=gfortran
FFLAGS=-c
EXECUTABLE=myprog

$(EXECUTABLE): myprog.o mymodule.o
	$(FC) $^ -o $@
...
\end{lstlisting}

Here, \verb|$^|, is used to list all prerequisites for this build, \mfname{mymodule.o myprog.o}. The \verb|$@|, denotes the current target as the output file for the compiler, in this case \mfname{\$(EXECUTABLE)} or \mfname{myprog}.

The rules for compiling source code can also be updated in a similar way:

\begin{lstlisting}
...
myprog.o: myprog.f90 
	$(FC) $(FFLAGS) $< -o $@
...
\end{lstlisting}	

Here the \verb|$<| variable denotes the first prerequisite, \mfname{myprog.f90}. The target macro, \verb|$@|, is also used to define the outputfile for the compiler.

There are several more internal macros that can be used in makefiles. For more information please see the GNU Make documenation \cite{gnumake12}.

% --------------------------------------------------------------------
\subsection{Suffix rules}
% --------------------------------------------------------------------

If a project consists of a larger number of source files, a large number of rules must be written. Make, solves this by implementing so called explicit rules. These rules can be regarded as a recipy for how to go from one extension, \mfname{.f90} to another \mfname{.o}. A explicit rule for compiling a Fortran source file to an object file then becomes:

\mmode

\begin{lstlisting}
FC=gfortran
FFLAGS=-c
EXECUTABLE=myprog

...

.f90.o:
	$(FC) $(FFLAGS) $< -o $@
\end{lstlisting}

This rule eliminates all the compilation rules used in the previous sections and makes the makefile more compact. To make the explicit rules work for compiling Fortran code, make needs to now which suffixes are used for Fortran source code. This is done with the special rule \mvar{.SUFFIXES}. The following example shows the completed makefile with the suffix rule:

\begin{lstlisting}
FC=gfortran
FFLAGS=-c
EXECUTABLE=myprog

$(EXECUTABLE): myprog.o mymodule.o
	$(FC) $^ -o $@

.f90.o:
	$(FC) $(FFLAGS) $< -o $@
	
clean:
	rm -rf *.o *.mod $(EXECUTABLE)

.SUFFIXES: .f90 .f03 .f .F
\end{lstlisting}

% --------------------------------------------------------------------
\subsection{Wildcard expansion and substitution}
% --------------------------------------------------------------------

Some times it can be beneficial to create lists of files by using wildcards. To do this in make, the \verb+$(wildcard ...)+ function can be used. To create a list of f90 source files the following assignment can be used:

\mmode

\begin{lstlisting}
F90_FILES := $(wildcard *.f90)
\end{lstlisting}

Please note the \verb+:=+ assignment operator used in conjunction with make function calls. 

When we have a list of source files, a list of object-files can easily be created by using the \mfunc{patsubst} function. This uses patterns to substitute the file suffixes from .f90 to .o. The assignment statement then becomes:

\begin{lstlisting}
OBJECTS := $(patsubst %.f90, %.o, $(F90_FILES))
\end{lstlisting}

The rule to link all object files into an executable then becomes:

\begin{lstlisting}
$(EXECUTABLE): $(OBJECTS)
	$(FC) $^ -o $@
\end{lstlisting}

This a much more generic rule, which can be reused for other projects without any change.

% --------------------------------------------------------------------
\subsection{Pattern rules}
% --------------------------------------------------------------------

The suffix rules defined in the previous section are provided by GNU make for compatibility with older makefiles. The recommended way of implementing suffix rules is using so called pattern rules.

A pattern rules specifies a ''Recipe'' for a rule that can handle multiple targets of a specific type. Using the \verb+%+ operator in the target specification to match filenames for which the generic rule will apply. A rule to compile Fortran source code to object files is written using pattern rules as follows:

\begin{lstlisting}
%.o: %.f90
	$(FC) $(FFLAGS) $< -o $@
\end{lstlisting}

This defines a recipe for make how to create an object-file from a .f90 source file. This rule is implicitly used when make encounters an object-file (implicit pattern rule).

The completed makefile with wildcards and pattern rules is shown below:

\mmode

\begin{lstlisting}
FC=gfortran
FFLAGS=-c
EXECUTABLE=myprog

F90_FILES := $(wildcard *.f90)
OBJECTS := $(patsubst %.f90, %.o, $(F90_FILES))

$(EXECUTABLE): $(OBJECTS)	
	$(FC) $^ -o $@

%.o: %.f90
	$(FC) $(FFLAGS) $< -o $@
	
clean:
	rm -rf *.o *.mod $(EXECUTABLE)
\end{lstlisting}

Please note that when using pattern rules the \mvar{.SUFFIXES} is not needed.

Even if the described makefile automatically can compile all source files, dependencies between Fortran 90 modules are not handled. The easiest way of handling module dependencies are to explicitly express these dependencies in the make file. To illustrate this, consider the following example:

\begin{description}
\item[myprog.f90] Main fortran program. Uses the mymodule module located in the mymodule.f90 source file.
\item[mymodule.f90] Module mymodule. Uses the myutils module in the myutils.f90 source file.
\item[myutils.f90] Module myutils. Self contained module without dependencies.
\end{description}

To build this example, we need to build myutils.f90 first as the mymodule.f90 needs the myutils.mod file created when myutils.f90 is compiled. To enable this dependency an additional rule is added to our make file:

\begin{lstlisting}
mymodule.o: myutils.o
\end{lstlisting}

This tells make that the object-file mymodule.o depends on myutils.o and makes sure that it will be built first. If we update the makefile in the previous section to handle this it becomes:

\mmode

\begin{lstlisting}
FC = gfortran
FFLAGS = -c
EXECUTABLE = myprog

F90_FILES := $(wildcard *.f90)
OBJECTS := $(patsubst %.f90, %.o, $(F90_FILES))

$(EXECUTABLE): $(OBJECTS) $(MODFILES)
	$(FC) $^ -o $@
	
mymodule.o: myutils.o

%.o %.mod: %.f90
	$(FC) $(FFLAGS) $< -o $@
	
clean:
	rm -rf *.o *.mod $(EXECUTABLE)

\end{lstlisting}

When executing this makefile with make, myutils.f90, will be the first target to be built.

\cmdmode

\begin{lstlisting}
$ make
gfortran -c myutils.f90 -o myutils.o
gfortran -c mymodule.f90 -o mymodule.o
gfortran -c myprog.f90 -o myprog.o
gfortran mymodule.o myprog.o myutils.o -o myprog
\end{lstlisting}

For more advanced make file use, the CMake tool is a better tool. CMake is covered in the next section.



% --------------------------------------------------------------------
\section{CMake}
% --------------------------------------------------------------------

When projects become large the time needed for maintaining the build system increases. This is often due to the fact that different OS environments needs to be handled in different ways and this has to be included in the makefile. CMake is a tool that can generate targeted makefiles and project files for most existing development environments. CMake works by parsing special files, CMakeLists.txt, and generating the needed makefiles and project files.

% --------------------------------------------------------------------
\subsection{Compiling code with cmake}
% --------------------------------------------------------------------

To use CMake, a CMakeLists.txt file has to be created. This is a normal text files with special CMake statements in it. Usually this files starts with a \cmfunc{cmake\_minimum\_required(VERSION 2.6)}. This prevents the CMakeLists.txt file to be used by a too old cmake. The first actual statement is usually \cmfunc{project(...)}-function defining the name of the project. 

\cmmode

\begin{lstlisting}
cmake_minimum_required(VERSION 2.6)
project(simple)
\end{lstlisting}

The name of the project is not the same as the executable but is used when generating project files for development environments. 

CMake by default does not support Fortran, so a special function, \cmfunc{enable\_language}-function is used to enable this:

\begin{lstlisting}
enable_language(Fortran)
\end{lstlisting}

To create an executable the \cmfunc{add\_executable}-function is used. This command takes an executable name as the first argument and a list of source files. 

\begin{lstlisting}
add_executable(simple myprog.f90)
\end{lstlisting}

The completed CMakeLists.txt file then becomes:

\begin{lstlisting}
cmake_minimum_required(VERSION 2.6)
project(simple)
enable_language(Fortran)

add_executable(simple myprog.f90)
\end{lstlisting}

Now when we have a CMakeLists.txt file it is possible to run \cli{cmake .} in the same directory to create the needed makefiles to build the project:

\cmdmode

\begin{lstlisting}
$ ls
CMakeLists.txt	myprog.f90
$ cmake .
-- The C compiler identification is GNU 4.2.1
-- The CXX compiler identification is Clang 4.0.0
-- Checking whether C compiler has -isysroot
-- Checking whether C compiler has -isysroot - yes
-- Checking whether C compiler supports OSX deployment target flag
-- Checking whether C compiler supports OSX deployment target flag - yes
-- Check for working C compiler: /usr/bin/gcc
-- Check for working C compiler: /usr/bin/gcc -- works
-- Detecting C compiler ABI info
-- Detecting C compiler ABI info - done
-- Check for working CXX compiler: /usr/bin/c++
-- Check for working CXX compiler: /usr/bin/c++ -- works
-- Detecting CXX compiler ABI info
-- Detecting CXX compiler ABI info - done
-- The Fortran compiler identification is GNU
-- Check for working Fortran compiler: /opt/local/bin/gfortran
-- Check for working Fortran compiler: /opt/local/bin/gfortran  -- works
-- Detecting Fortran compiler ABI info
-- Detecting Fortran compiler ABI info - done
-- Checking whether /opt/local/bin/gfortran supports Fortran 90
-- Checking whether /opt/local/bin/gfortran supports Fortran 90 -- yes
-- Configuring done
-- Generating done
-- Build files have been written to: /Users/.../simple
$ ls
CMakeCache.txt		CMakeLists.txt		cmake_install.cmake
CMakeFiles		Makefile		myprog.f90
\end{lstlisting}

As show in the above output, cmake, has generated a lot of files one of them being a normal makefile. To build the project, the normal make command can be used.

\begin{lstlisting}
$ make
Scanning dependencies of target simple
[100%] Building Fortran object CMakeFiles/simple.dir/myprog.f90.o
Linking Fortran executable simple
[100%] Built target simple
\end{lstlisting}

CMake generates a lot of files when run. Which can make the source tree quite cluttered. The recommended way of running CMake is to create a separate build directory and generate the build files in this directory. This is done in the following example:

\begin{lstlisting}
$ mkdir build
$ cd build
$ cmake ..
-- The C compiler identification is GNU 4.2.1
.
.
-- Generating done
-- Build files have been written to: /Users/.../simple/build
\end{lstlisting}

Make is then run in this directory as before. In this approach it is easy to remove the build files by removing the build directory.

% --------------------------------------------------------------------
\subsection{Building debug and release versions}
% --------------------------------------------------------------------

By default CMake generates build files for compiling debug versions of an applicaiton. That is using no optimisation and with debug symbols. Controlling the build type can be done by assigning the variable \cmvar{CMAKE\_BUILD\_TYPE} to either \cmvar{Release} or \cmvar{Debug} when executing CMake. Variables can be set on the command line by using the switch -D as shown in the following example:

\begin{lstlisting}
$ cmake -D CMAKE_BUILD_TYPE=Release ..
-- Configuring done
-- Generating done
-- Build files have been written to: /Users/lindemann/Development/progsci_book/source/cmake_examples/simple/build
\end{lstlisting}

% --------------------------------------------------------------------
\subsection{Adding library dependencies}
% --------------------------------------------------------------------

In the previous examples the binaries have been built without any library dependencies. To add link dependencies, the \cmfunc{target\_link\_libraries} can be used. To add the libraries, \cmvar{libblas} and \cmvar{libm} as dependencies of the executable, the CMakeList.txt becomes:

\cmmode

\begin{lstlisting}
cmake_minimum_required(VERSION 2.6)
project(simple)
enable_language(Fortran)

add_executable(simple myprog.f90)
target_link_libraries(simple blas m)
\end{lstlisting}

To show what switches that are actually used when building the executable, the \cmvar{CMAKE\_VERBOSE\_MAKEFILE}, is set to \cmvar{ON}. This will show the actual commands used during the build.

\cmdmode

\begin{lstlisting}
$ mkdir build
$ cd build/
$ cmake -D CMAKE_VERBOSE_MAKEFILE=ON ..
-- The C compiler identification is GNU 4.2.1
-- The CXX compiler identification is Clang 4.0.0
...
-- Generating done
-- Build files have been written ...
$ make
...
/opt/local/bin/gfortran [...]/mymodule.f90.o  -o multiple  -lblas -lm 
...
\end{lstlisting}

Which shows that the libraries have been added to the actual compilation command.

% --------------------------------------------------------------------
\subsection{Variables and conditional builds}
% --------------------------------------------------------------------

Often when compiling code under different platforms, special flags and commands have to be used. CMake supports conditional statements in the CMakeLists.txt files to handle these cases. To test for a Unix-build the following if statement can be used:

\cmmode

\begin{lstlisting}
if (UNIX)
	message("This is a Unix build.")
endif (UNIX)
\end{lstlisting}

\cmvar{UNIX} is predefined variable that is true when building on Unix-type system. When running CMake on a Unix-type system will print ''This is a Unix build.'' on the console.

CMake also has an else-statement. The following code, creates a build target and adds different build options depending on the platform used:

\begin{lstlisting}
if (UNIX)
	add_executable(multiple myprog.f90 mymodule.f90)
	target_link_libraries(multiple blas m)
else (UNIX)
	if (WIN32)
		add_executable(multiple myprog.f90 mymodule.f90)
		target_link_libraries(multiple blas32)
	else (WIN32)
		message("Not supported configuration.")
	endif (WIN32)
endif (UNIX)
\end{lstlisting}

It is also possible to use variables in CMake. Variables can be both strings and lists of strings. A variable is created by using the \cmfunc{set}-function. The following example shows how a simple string variable is created:

\begin{lstlisting}
set(MYVAR "Hello, world!")
\end{lstlisting}

To use the actual value of a variable, it has to be preceded by a \$ enclosed by curly brackets as shown in the following example:

\begin{lstlisting}
set(MYVAR "Hello, world!")
message(${MYVAR})
\end{lstlisting}

This will print the contents of the variable, \cmvar{MYVAR}. If not enclosed it will print the name of the variable.

Variables can also be lists of values which can be iterated over. Creating a list is also done using the \cmfunc{set}-function, as shown in this example:

\begin{lstlisting}
set(MYLIST a b c)
message(${MYLIST})
\end{lstlisting}

Here, \cmvar{MYLIST}, containing 3 strings. The \cmfunc{message}-function will concatenate the items in the list and the resulting output of running cmake will be:

\cmdmode

\begin{lstlisting}
$ cmake ..
abc
-- Configuring done
-- Generating done
-- Build files have been written to: ...
\end{lstlisting}

Using a list variable it is also possible to do an iteration using a \cmfunc{for}-statement, which the following example shows:

\cmmode

\begin{lstlisting}
set(MYLIST a b c)
foreach(i ${MYLIST})
	message(${i})
endforeach(i)
\end{lstlisting}

Running this using CMake produced the following output:

\cmdmode

\begin{lstlisting}
$ cmake ..
a
b
c
-- Configuring done
-- Generating done
-- Build files have been written to: ...
\end{lstlisting}

% --------------------------------------------------------------------
\subsection{Controlling optimisation options}
% --------------------------------------------------------------------

Optimisation options can differ between compilers. To control the optimisation options in CMake, conditional builds using if-statements can be used. First, the used compiler needs to be queried. The path to the actual compiler is stored in the \cmvar{CMAKE\_Fortran\_COMPILER}. To create an if-statement on the compiler the compiler command must be extracted from the compiler path. This can be accomplished using the \cmfunc{get\_filename\_component}

\cmmode

\begin{lstlisting}
get_filename_component (Fortran_COMPILER_NAME ${CMAKE_Fortran_COMPILER} NAME)
\end{lstlisting}

This command extracts the filename component of the path and stores it in the variable \cmvar{Fortran\_COMPILER\_NAME}. Next, an if-statement has to implemented that queries for different compilers. A string comparison can be done using the \cmfunc{STREQUAL} operator in CMake. Compilation flags for CMake are stored in \cmvar{CMAKE\_Fortran\_FLAGS\_RELEASE} for release mode flags and \cmvar{CMAKE\_Fortran\_FLAGS\_DEBUG} for debug flags. An example fo this king of conditional compilation statement is shown below (from \cite{cmakecond12}):

\cmmode

\begin{lstlisting}
if (Fortran_COMPILER_NAME STREQUAL "gfortran")
  set (CMAKE_Fortran_FLAGS_RELEASE "-funroll-all-loops -fno-f2c -O3")
  set (CMAKE_Fortran_FLAGS_DEBUG   "-fno-f2c -O0 -g")
elseif (Fortran_COMPILER_NAME STREQUAL "ifort")
  set (CMAKE_Fortran_FLAGS_RELEASE "-f77rtl -O3")
  set (CMAKE_Fortran_FLAGS_DEBUG   "-f77rtl -O0 -g")
elseif (Fortran_COMPILER_NAME STREQUAL "g77")
  set (CMAKE_Fortran_FLAGS_RELEASE "-funroll-all-loops -fno-f2c -O3 -m32")
  set (CMAKE_Fortran_FLAGS_DEBUG   "-fno-f2c -O0 -g -m32")
else (Fortran_COMPILER_NAME STREQUAL "gfortran")
  message ("No optimized Fortran compiler flags are known, we just try -O2...")
  set (CMAKE_Fortran_FLAGS_RELEASE "-O2")
  set (CMAKE_Fortran_FLAGS_DEBUG   "-O0 -g")
endif (Fortran_COMPILER_NAME STREQUAL "gfortran")
\end{lstlisting}
 
% --------------------------------------------------------------------
\subsection{Generating project files for development environments}
% --------------------------------------------------------------------

CMake is not limited to generating makefiles, it can also generate project files for a number of graphical development environments. Supported generators in CMake can be listed by running the \cmfunc{cmake}-command without parameters. The following list is produced on a Mac OS X based machine:

\cmdmode

\begin{lstlisting}
enerators

The following generators are available on this platform:
  Unix Makefiles = Generates standard UNIX makefiles.
  Xcode          = Generate Xcode project files.
  CodeBlocks - Unix Makefiles = Generates CodeBlocks project files.
  Eclipse CDT4 - Unix Makefiles = Generates Eclipse CDT 4.0 project files.
  KDevelop3      = Generates KDevelop 3 project files.
  KDevelop3 - Unix Makefiles  = Generates KDevelop 3 project files.
\end{lstlisting}

This lists covers most common development environments for Mac OS X. When running on a Windows machine, generators for Visual Studio and other development environments for that platform will be available as well.

To generate build files for a different generator the \cmvar{-G}-switch is used. In the following example build files for the Eclipse-environment are generated. 

\begin{lstlisting}
$ mkdir build_eclipse
$ cd build_eclipse/
$ cmake -G "Eclipse CDT4 - Unix Makefiles" ../multiple/
-- The C compiler identification is GNU 4.2.1
-- The CXX compiler identification is Clang 4.0.0
-- Could not determine Eclipse version, assuming at least 3.6 (Helios). Adjust CMAKE_ECLIPSE_VERSION if this is wrong.
...
-- Generating done
-- Build files have been written to: ...
$ ls -la
total 112
drwxr-xr-x   8 lindemann  staff    272 Aug 29 20:07 .
drwxr-xr-x  13 lindemann  staff    442 Aug 29 20:06 ..
-rw-r--r--   1 lindemann  staff  14343 Aug 29 20:07 .cproject
-rw-r--r--   1 lindemann  staff   5527 Aug 29 20:07 .project
-rw-r--r--   1 lindemann  staff  17808 Aug 29 20:07 CMakeCache.txt
drwxr-xr-x  21 lindemann  staff    714 Aug 29 20:07 CMakeFiles
-rw-r--r--   1 lindemann  staff   4770 Aug 29 20:07 Makefile
-rw-r--r--   1 lindemann  staff   1562 Aug 29 20:07 cmake_install.cmake
\end{lstlisting}

When generation is completed this directory can be added to a Eclipse workspace as a project.

Please note that in the above example we are using a build directory not located in the source tree. This is the recommended way for an Eclipse based project.



% --------------------------------------------------------------------
%\subsection{Writing documentation}
% --------------------------------------------------------------------
%---------------------------------------------------------------------
%---------------------------------------------------------------------
\chapter{Photran (in progress)}
%---------------------------------------------------------------------
%---------------------------------------------------------------------

Photran is a integrated development environment, IDE, for
Fortran based on the Eclipse-project. The user interface
resembles the one found in commercial alternatives such as
Microsoft Visual Studio or Absoft Fortran. This chapter gives a
short introduction on how to get started with this development
enviroment

\section{Starting Photran}

Photran is started by choosing ''Programs/Fortran Python
Software Pack/Photran IDE'' in the start-menu in Windows. When
Photran has been started a dialog is shown asking for a location of a workspace directory, see
figure~\ref{fig:ph_select_workspace}. A workspace is a
directory containing Photran configuration and project files.
If the checkbox, \menuitem{Use this as the default...}, is
checked this question will not appear the next time Photran is
started, the selected workspace will be used by default.

\figmedium{photran/photran001.png}{Choice of workspace
directory}{fig:ph_select_workspace}

When Photran is started for the first time, a welcome screen is shown. This screen will not be used in this chapter. Click on

\fignc{0.1\textwidth}{ph_photran_start2}{}

\noindent to show Photran's normal user interface layout, as
shown in figure~\ref{fig:photran_default_layout}

\figmedium{photran/photran004.png}{Photran default interface layout}{fig:photran_default_layout}

\section{Creating a Photran Makefile project}

The Photran IDE is centered around projects. This means that
source files and build files are added to projects which
Photran are maintained by Photran. Unfortunately, Photran does
not have as advanced project management features as its parent
project Eclipse. Photran can't generate makefiles automatically from the source files contained in the project. To solve this
the Fortran Python Software Pack comes with a Fortran Make File Generator that generates a Makefile from files located in the
project directory. The following example shows how to create
and configure a Makefile project in Photran.

First a new project is created in Photran by selecting
\menuitem{File/New/Ohter}. This brings up a dialog listing the
available project types in Photran, see
figure~\ref{fig:photran_project_types}.

\figmedium{photran/photran006.png}{Project
type}{fig:photran_project_types}

\noindent Select \menuitem{Fortran/Fortran Project} in the list shown. Click \button{Next}.

In the next page, figure~\ref{fig:photran_project_type_and_name}, enter the name of the project and select the ''Makefile project'' in the ''Project type'' list.

\figmedium{photran/photran007.png}{Project name and type}{fig:photran_project_type_and_name}

\noindent To be able to build a project a toolchain must be selected. A toolchain is set of tools and compilers that is used to build a project. Linux user can choose ''Linux GCC'' or ''GCC Toolchain''. On Windows the toolchains ''GCC Toolchain'', ''MinGW GCC'' or ''Cygwin GCC'' can be selected depending on the tools installed. If the Fortran Python Software Pack is installed ''MinGW GCC'' must be chosen for Windows. Windows users should uncheck the box ''Show project types and toolchains only if they are supported on the platform.''. This will show all available toolchains even if Photran can't detect them. Click \button{Next} to go to the last configuration page. In this step an error parser must be configured in the advanced settings. Click on \button{Advanced Settings...}. This brings up the advanced configuration dialog. Select \menuitem{Fortran Build/Settings}. In the \guitab{Binary Parsers} parsers for different executable formats can be selected, see figure~\ref{fig:binary_parsers}.

\figsmall{photran/photran009.png}{Binary parser configuration}{fig:binary_parsers}

On Windows the ''PE Windows Parser'' should be selected. On Linux the ''Elf Parser'' should be selected. In the \guitab{Error Parsers}, see figure~\ref{fig:error_parsers}, parsers for compiler error messages can be selected. The closest match for the gfortran compiler is the ''Fortran Error Parser for G95 Fortran'' selection.

\figmedium{photran/photran010.png}{Error parser configuration}{fig:error_parsers}

Click \button{OK} to save the settings and close the advanced settings dialog. The project is now ready to be created. Click \button{Finish} to create the project. Before the project is saved Photran shows a dialog with the option of switching to the Fortran Perspective, see figure~\ref{fig:fortran_perspective}.

\figmedium{photran/photran011.png}{Switching to Fortran Perspective}{fig:fortran_perspective}

\noindent A perspective in Photran is a pre-configured layout of the development environment. Photran comes with a Fortran perspective and a Fortran Debug perspective used when debugging Fortran applications.

\subsection{Adding a new source file}

A new source file is added to the project by selecting \menuitem{File/New/Other} and selecting ''Source File'' from the Fortran Folder. Click \button{Next}. In the next page the name of the source file is entered. Click \button{Finish} to create the file and add it to the project.

\section{Building the project}

When all source files have been added to the project it can be built from the \menuitem{Project/Build All...} menu. This will execute the build process. Output from the process can be seen in the \guitab{Console} tab in the lower pane of the Photran window, as shown in figure~\ref{fig:build_process_output}.

\figmedium{photran/photran_build_process001.png}{Output from the build process}{fig:build_process_output}

Any errors in the build process are also shown in this tab.

\section{Running the project}

When the project has been built successfully it can be run by selecting \menuitem{Run/Run} from the menu. Output from the program is redirected to the \guitab{Console} in the lower part of the window, as shown in figure~\ref{fig:output_from_program}.

\figmedium{photran/photran_running_project002.png}{Output from running program}{fig:output_from_program}




%---------------------------------------------------------------------
%---------------------------------------------------------------------
\chapter{Qt Creator for Fortran}
%---------------------------------------------------------------------
%---------------------------------------------------------------------

Qt Creator is a integrated development environment, IDE, for C++ and Qt, but can be easily adapted as a development environment for Fortran using plugins provided together with this book.

The user interface of Qt Creator resembles the one found in commercial alternatives such as Microsoft Visual Studio or Inte Visual Fortran. This chapter gives a short introduction on how to get started with this development enviroment.

To be able to use the Fortran Project templates and compile Fortran code in Qt Creator, require the following pre-requisites:

\begin{itemize}
\item A working Fortran compiler available in the search path. On Windows this is best achieved by installing gfortran and related tools from the MinGW project. [FIX] Reference. On Mac OS X the gfortran compiler is available in the MacPorts distribution [FIX] Reference.
\item CMake installed and available as a command line tools. CMake is available to download from XXX [FIX].
\item Fortran Templates for Qt Creator which can be downloaded here. These templates work for all platforms. [FIX]
\end{itemize}

%---------------------------------------------------------------------
\section{Starting Qt Creator}
%---------------------------------------------------------------------

Starting Qt Creator can be done from the Start-menu on Windows and from the application launcher on Mac OS X. When Qt Creator has been launched the main windows as shown in figure~\ref{fig:qtcreator_main_window}, is shown.

\fignormal{qtcreator_main_window}{Qt Creator main window}{fig:qtcreator_main_window}

On the left side of the window, 2 toolbars are shown. The top toolbar controls the main program modes of Qt Creator. The lower toolbar is the project build toolbar, which controls how the projects are built and run.

\section{Qt Creator main program modes}

There are 7 program modes in Qt Creator, controlling the workflow of the development environment. The different modes are listed below:

\begin{itemize}
\item Welcome mode - Shows a welcome screen, providing shortcuts for many of the common operations of the development environment.
\item Edit mode - This is probarbly the most used mode of the development environment. This mode provides access to the files within a project as well as an source code editor supporting most languages.
\item Design mode - In this mode user interfaces for Qt can be designed. This mode will not be used for Fortran development.
\item Debug mode - This mode will be activated when the application is run in debug mode, for interactive debugging your Fortran application.
\item Project mode - This mode provides access to settings that applies to the current project.
\item Analyse mode - Provides access to profiling tools. This mode will not be used in this book.
\item Help mode - Provides access to the online documenation provided by the development environment.
\end{itemize}

Switching between modes are in many cases done automatically. Debugging an application will automatically switch to debug mode. Creating or opening a project will automatically switch to edit mode.

%---------------------------------------------------------------------
\section{Installing highlighting rules for Fortran}
%---------------------------------------------------------------------

As Qt Creator is not by default used for Fortran development, highlighting rules for Fortran are not installed. Qt Creator has a built in mechanism for installing highlighting rules automatically. The mecahnism is reached from the the preferences dialog. In the preferences dialog, select the ''Text Editor'' section and then the ''Generic Highlighter'' tab. In this tab select ''Download Definitions...'' to show a dialog for selecting and downloading highlighting schemes for Qt Creator. Figure~\ref{fig:qt_fortran_highlight_3} and \ref{fig:qt_fortran_highlight_4} illustrates the process.

\fignormal{qt_fortran_highlight_3}{Qt Creator main window}{fig:qt_fortran_highlight_3}
\figsmall{qt_fortran_highlight_4}{Qt Creator main window}{fig:qt_fortran_highlight_4}

%---------------------------------------------------------------------
\section{Creating a Fortran project}
%---------------------------------------------------------------------

To use Qt Creator as a development environment for Fortran, a project has to be created. A project defines, which files that are required for building the program as well as any required settings. By default Qt Creator uses its own custom project format, but can also handle CMake based project, which is also what the installed Fortran plugins use. 

To create a new project select \menuitem{File/New File or Project...} from the Qt Creator main menu. This brings up a dialog for selecting new files and projects. Select \menuitem{Non-Qt Project/Plain Fortran Project (CMAke Build)} as shown in the following figure:

%\fignormal{qt_create_project_1}{Select project type}{fig:qt_create_project_1}
\fignc{0.6\textwidth}{qt_create_project_1}

In the next step the name and location of the project will be set, as in the following figure. All files for the project will be created in this directory.

%\fignormal{qt_create_project_2}{Select project name and location}{fig:qt_create_project_2}
\fignc{0.6\textwidth}{qt_create_project_2}

As this project uses CMake, a build directory must be specified. This is done, so that the source directory remains clean from all files uses during the build. It is also possible to select a build directory within the source tree. 

%\fignormal{qt_create_project_3}{Select build location}{fig:qt_create_project_3}
\fignc{0.6\textwidth}{qt_create_project_3}

In the next step of the guide CMake is run with any needed options. Click \button{Run CMake} to run the CMake configuration process. If not errors are encountered click \button{Done} to continue to the next step.

%\fignormal{qt_create_project_4}{Select project type}{fig:qt_create_project_4}
\fignc{0.6\textwidth}{qt_create_project_4}

In the final step, Qt Creator creates the project files and displays the editor window with the resulting project files shown in the left pane as shown in the following figure:

%\fignormal{qt_create_project_5}{Select project type}{fig:qt_create_project_5}
\fignc{0.8\textwidth}{qt_create_project_5}

%\subsection{Adding a new source file}
%---------------------------------------------------------------------
\subsection{Building the project}
%---------------------------------------------------------------------

Building the project is done either by selecting \menuitem{Build/Build All} or \menuitem{Build/Build [Projectname]}. It is also possible to use the "Hammer" button in the bottom left of the window, to initiate the build. 

To see the results of the build the button \menuitem{Compile Output} can be clicked. This brings up a pane showing the compilation output as shown below:

\fignc{0.6\textwidth}{qt_creator_compile_pane}

The compile output pane will also show any errors during the build as in the following output:

\fignc{0.6\textwidth}{qt_creator_compile_error}

%---------------------------------------------------------------------
\subsection{Running the project}
%---------------------------------------------------------------------

Running the finished application can be done by selecting \menuitem{Build/Run}. This will run the first target in the project. If the CMakeLists.txt file contains more targets (add\_executable), the selected executable to run can be selected by clicking on the ''Terminal'' icon in the lower left toolbar. This brings up a menu in which you can select the target to run as shown in the following figure:

\fignc{0.5\textwidth}{qt_creator_select_run_target}





%\chapter{Development tools}
%\section{Build management}
%\chapter{Fortran and Python}
%\chapter{Fortran and C/C++}
%\include{chapter_opengl2d}
%\include{chapter_absoft}
%\include{chapter_vf}

\pmode

%\include{chapter_objectpascal}
%\include{chapter_delphi}
%\include{chapter_dll}
%\include{chapter_wizard}
%%---------------------------------------------------------------------
%---------------------------------------------------------------------
\chapter{Till�mpningsexempel: Kontinuerlig balk} \label{ch:application}
%---------------------------------------------------------------------
%---------------------------------------------------------------------

\fmode

F�r att belysa den kompletta processen att koppla ber�kningskod
skriven i Fortran till ett anv�ndargr�nssnitt utvecklat i Delphi,
kommer utvecklingsprocessen f�r ett ber�kningsprogram f�r
kontinuerliga balkar att beskrivas. Alla steg kommer ej att g�s
igenom, f�r detaljinstruktioner h�nvisas till tidigare kapitel.
Den kompletta k�llkoden f�r programmet �terfinns i
bilaga~\ref{app:application_source}

%---------------------------------------------------------------------
\section{Ber�kningsdel (Fortran)}
%---------------------------------------------------------------------

Ber�kningsdelen implementeras som ett dynamiskt l�nk bibliotek
(DLL) enligt Kapitel 6. Kommunikationen mellan ber�kningsdelen och
anv�ndargr�nssnittet sker genom en huvudsubrutin \textit{calc}.
Indata och utdata fr�n ber�kningsdelen skickas genom ett antal
parametrar i \textit{calc} rutinen. �verf�ringen till dessa sker
genom referens, dvs det anropande programmet allokerar indata- och
utdata-variabler och f�r �ver dessa till ber�kningsdelen som
referenser. Ber�kningsdelen manipulerar direkt dessa variabler
utan on�digt kopierande mellan programdelarna. Figur 8{\-}1
illustrerar relationerna mellan de olika delarna i det f�rdiga
programmet.

\fignormal{kompendiumv4Fig181.eps}{Relationer mellan
programdelar}{fig:components}

F�r att ber�kningen av den kontinuerliga balken skall kunna
genomf�ras beh�vs ett antal underrutiner:

\begin{xlist}
\item Elementrutin f�r att skapa elementstyvhetsmatris och
elementlastvektor, \fmethod{beam2e} %
\item Assembleringsrutin, \fmethod{assemElementLoad}%
\item L�sningsrutiner, \fmethod{solveq} och \fmethod{solve}%
\item Rutin f�r ber�kning av deformationer och snittkrafter l�ngs
balken, \fmethod{beam2s}
\end{xlist}

Rutinerna ovan placeras i en separat Fortranmodul kallad
\fmodule{beam}. I denna l�ggs ocks� eventuella deklarationer, som
beh�vs i hela ber�kningsdelen.

%---------------------------------------------------------------------
\subsection{Modulen beam}
%---------------------------------------------------------------------

F�r att huvudrutinen skall vara l�tt att �verblicka och modifiera
har alla n�dv�ndiga rutiner f�r ber�kning av den kontinuerliga
balken placerats i en egen modul med namnet \fmodule{beam}.
Modulen �r uppbyggd enligt f�ljande:

\begin{lstlisting}[texcl]
module beam

    ! Precision p� flyttal

    integer, parameter :: ap=selected_real_kind(15,300)

contains

    subroutine beam2e(...)
    .
    .
    end subroutine beam2e

    subroutine beam2s(...)
    .
    .
    end subroutine beam2s

    subroutine solveq(...)
    .
    .
    end subroutine solveq

    subroutine solve(...)
    .
    .
    end subroutine solve

end module beam
\end{lstlisting}

Konstanten \fvar{ap} som deklareras i b�rjan av modulen �r
tillg�nglig f�r alla rutiner och program som anv�nder modulen.
Detta g�r det ocks� l�tt f�r huvudrutinen \fmethod{calc} att
deklarera korrekt flyttalstyp f�r anv�ndning mot modulen
\fmodule{beam}.

Elementet som anv�nds i programmet �r en enkel Bernoulli-balk
h�mtad fr�n [9]. Geometri och frihetsgrader visas i
figur~\ref{fig:beam_type}.

\figmedium{kompendiumv4Fig182.eps}{Enkel
Bernoulli-balk}{fig:beam_type}

P� grund av att balkens frihetsgrader sammanfaller med det globala
systemet kan elementets styvhetsmatris s�ttas upp direkt utan
n�gra transformationer. Elementet �r implementerat i rutinen
\fmethod{beam2e} vilken tar l�ngd \fvar{L}, last \fvar{q} och
elasticitetsmodulen \fvar{E} och tv�rsnittsegenskaperna \fvar{I}
och \fvar{A} som indata. Utdata �r styvhetsmatrisen \fvar{Ke}
[4x4] och elementlastvektorn \fvar{fe} [4x1]. Rutinen har f�ljande
syntax.

\begin{fsyntax}
\textbf{call} beam2e(L,q,E,A,I,Ke,fe)
\end{fsyntax}

Assemblering g�rs med rutinen \fmethod{assemElementLoad}. Denna
rutin assemblerar in godtycklig elementstyvhetsmatris \fvar{Ke}
och elementkraftvektor \fvar{fe} in den globala styvhetsmatrisen
utifr�n topologivektorn \fvar{Edof}. Syntaxen �r f�ljande:

\begin{fsyntax}
\textbf{call} assemElementLoad(Edof,K,Ke,f,fe)
\end{fsyntax}

L�sning av ekvationssystemet sker med hj�lp av rutinen
\fmethod{solveq}. Denna rutin �r egentligen en inkapsling av en
annan l�sningsrutin \fmethod{solve}, f�r anpassning till samma
konvention f�r indata som anv�nds i CALFEM [11]. Syntaxen f�r
\fmethod{solveq} �r:

\begin{fsyntax}
\textbf{call} solveq(K, f, bcPrescr, bcValue, a, Q)
\end{fsyntax}

Indata till rutinen �r den globala styvhetsmatrisen \fvar{K}, den
globala lastvektorn \fvar{f}, heltalsvektorn \fvar{bcPrescr}, med
ettor p� de frihetsgrader som skall vara f�reskrivna och nollor i
�vrigt, och vektorn \fvar{bcValue} med de f�reskrivna v�rdena.
Utdata �r den globala f�rskjutningsvektorn \fvar{a} och
reaktionsvektorn \fvar{Q}.

N�r l�sningen av ekvationssystemet �r slutf�rd ber�knas
elementkrafterna med rutinen \fmethod{beam2s}. Rutinen har
f�ljande syntax:

\begin{fsyntax}
call beam2s(L, q, E, A, I, Ed, np, ShearForces, Moments, \&
Deflections)
\end{fsyntax}

Parametrarna \fvar{L}, \fvar{E}, \fvar{A}, \fvar{I} och \fvar{q}
har beskrivits ovan. \fvar{Ed} inneh�ller de lokala
elementf�rskjutningarna. Antalet ber�kningspunkter inklusive start
och slutnod anges med \fvar{np}. Utdata fr�n rutinen lagras i
vektorerna \fvar{ShearForces}, \fvar{Moments}
och \fvar{Deflections}. \\

%---------------------------------------------------------------------
\subsection{Huvudrutinen calc}
%---------------------------------------------------------------------

Rutinen \fmethod{calc} kan man kalla f�r ber�kningsdelens
huvudrutin. Det �r denna rutin som kommer att anropas av
huvudprogrammet. N�r en huvudrutin av denna typ skall definieras
�r det viktigt att t�nka igenom vilka indata- och utdatavariabler
som skall f�ras �ver i anropet. Det �r ocks� viktigt att se till
att datatyperna st�mmer �verens med huvudprogrammets datatyper.
Indatavariablerna f�r detta fall visas i
tabell~\ref{tbl:input_variables_calc}.

\begin{table}[!htb]
\begin{center}
\begin{tabular}{|l|p{0.5\textwidth}|}
\hline
Variabel &  Beskrivning \\
\hline
\ftype{integer(4) :: nBeams}  &   Antalet balkar \\
\ftype{integer(4) :: nMaterials}  &  Antalet definierade material \\
\ftype{real(8) :: BeamLengths(*)} &  Vektor inneh�llande l�ngderna f�r de ing�ende balkdelarna. Antalet element m�ste vara minst nBeams. \\
\ftype{real(8) :: BeamLoads(*)} & Vektor inneh�llande de utbredda lasterna f�r de ing�ende balkdelarna. Antalet element m�ste vara  minst nBeams. \\
\ftype{integer(4) :: BeamProps(*)} & Heltalsvektor inneh�llande materialkoder f�r ing�ende balkdelar. Antalet element m�ste vara minst nBeams. \\
\ftype{integer(4) :: BCTypes(*)} &  Heltalsvektor inneh�llande randvillkorstyper f�r de ing�ende noderna. \\
 & \\
 & Randvillkorstyperna �r: \\
 & 0 = Inget f�reskrivet randvillkor \\
 & 1 = F�reskrivet v�rde rotation \\
 & 2 = F�reskrivet v�rde f�rskjutning \\
 & 3 = F�reskrivet v�rde rotation och f�rskjutning \\
 & \\
\ftype{real(8) :: BCDisplValues(*)} & F�reskrivna f�rskjutningsv�rden f�r de ing�ende noderna. Antalet element m�ste vara minst nBeams+1. \\
\ftype{real(8) :: BCRotValues(*)} &  F�reskrivna rotationsv�rden f�r de ing�ende noderna. M�ste vara minst nBeams+1 element. \\
\ftype{real(8) :: Materials(3,*)} &  Matris inneh�llande material- och tv�rsnittsegenskaper. Varje kolumn lagrar E, A, I v�rden. Antalet element m�ste vara minst nMaterials. \\
\ftype{integer(4) :: EvaluationPoints} &  Antalet ber�kningspunkter p� balkdelarna \\
\hline
\end{tabular}
\end{center}
\caption{Indatavariabler f�r calc} \label{tbl:input_variables_calc}
\end{table}

Utdatavariablerna i \fmethod{calc} �r de resultat som skall
returneras till huvudprogrammet. I detta fallet skall
deformationer, moment, tv�rkrafter, globala f�rskjutningar och
reaktionskrafter returneras. Variablerna visas i
tabell~\ref{tbl:output_variables_calc}.

\begin{table}[!htb]
\begin{center}
\begin{tabular}{|l|p{0.5\textwidth}|}
\hline
Variabel &  Beskrivning \\
\hline
\ftype{real(8) :: Displacements} &   Vektor med globala f�rskjutningar. Antalet element m�ste vara minst $(nBeams+1)*2$. \\
\ftype{real(8) :: Reactions} &  Vektor med globala reaktionskrafter. Antalet element m�ste vara minst $(nBeams+1)*2$. \\
\ftype{real(8) :: Moments(EvaluationPoints,*)} & Moment l�ngs balksegment. Varje kolumn representerar ett balksegment. Antalet kolumner m�ste vara minst nBeams. \\
\ftype{real(8) :: ShearForces(EvaluationPoints,*)} & Tv�rkrafter l�ngs balksegment. Varje kolumn representerar ett balksegment. Antalet kolumner m�ste vara minst nBeams. \\
\ftype{real(8) :: Deflections(EvaluationPoints,*)} & Deformationer l�ngs balksegment. Varje kolumn representerar ett balksegment. Antalet kolumner m�ste vara minst nBeams. \\
\hline
\end{tabular}
\end{center}
\caption{Utdatavariabler f�r calc} \label{tbl:output_variables_calc}
\end{table}

I ovanst�ende deklarationer anges inte flyttalsnogrannheten med
den f�rdefinierade konstanten \fvar{ap}. Detta beror p� att
datatyperna mellan ber�kningsdelen och huvudprogrammet m�ste
st�mma �verrens. Skulle konstanten \fvar{ap} variera kan man inte
vara s�ker p� att det �r samma typ av variabler i huvudprogram och
ber�kningsdel. I huvudprogrammet anv�nds datatyperna
\ftype{double} och \ftype{integer} vilket i Visual Fortran
motsvaras av \ftype{real(8)} och \ftype{integer(4)}.

\newpage
F�r att rutinen \fmethod{calc} skall kunna n�s fr�n ett
huvudprogram m�ste speciella instruktioner l�ggas in f�r att
kompilatorn skall l�gga in dem p� r�tt s�tt i det dynamiskt
l�nkade biblioteket (DLL). De f�rsta instruktionerna som l�ggs in
�r att rutinen skall exporteras (\fkeyw{dllexport)} och att
anropskonventionen skall vara av typen \fkeyw{stdcall}. Denna
anropskonvention �r den som fungerar b�st tillsammans med Borland
Delphi. Instruktioner av denna typ l�ggs alltid precis efter
subrutin-deklarationen p� f�ljande s�tt:

\begin{lstlisting}[texcl]
subroutine calc(nBeams, nMaterials, BeamLengths, &
    BeamLoads, BeamProps, &
    BCTypes, BCDisplValues, BCRotValues,&
    Materials, Displacement, Reaction, &
    EvaluationPoints, ShearForces, Moments,&
    Deflections)

    !dec\$attributes dllexport, stdcall :: calc
    .
    .
\end{lstlisting}

P� grund av att det inte finns n�gon klar standard �ver hur
indata- och utdatavariabler skall skickas mellan program skrivna i
olika spr�k m�ste detta ocks� specificeras. Det finns tv�
huvudsakliga s�tt att skicka variabler mellan program, som
referens eller som v�rden. N�r en variabel skickas som referens
skickas inte inneh�llet i variabeln till underprogrammet, utan den
variabel som underprogrammet tar emot motsvaras av en variabel i
huvudprogrammet. Skickas variabler som v�rden, kopieras inneh�llet
i variabeln till en ny variabel som f�rs �ver i anropet. Figur
8{\-}3 visar de tv� anropsalternativen.

\fignormal{kompendiumv4Fig183.eps}{Anropsmetoder}{fig:calling_conventions}

Vilken metod som v�ljs styrs av typ av data och storleken p�
denna. Grundregler:

\begin{xlist}
\item Alla returvariabler skickas med referens
(\fkeyw{reference)}. %
\item Skal�ra indatavariabler skickas som v�rden (\fkeyw{value}).
Detta f�rhindrar att den anropande rutinen av misstag modifierar
ett variabelv�rde.%
\item St�rre matriser eller vektorer b�r skickas som referenser
(\fkeyw{reference)}. %
\item Str�ngar skall skickas som referenser (\fkeyw{reference)}.%
\end{xlist}

Kompilatorinstruktionerna f�r anropsvariablerna i \textit{calc}
ser ut p� f�ljande s�tt:

\begin{lstlisting}
!dec\$attributes value     :: nBeams,nMaterials
!dec\$attributes reference :: BeamLengths,BeamLoads,
!dec\$attributes reference :: BeamProps
!dec\$attributes reference :: BCTypes,BCDisplValues
!dec\$attributes reference :: BCRotValues,Materials
!dec\$attributes reference :: Displacement,Reaction
!dec\$attributes value     :: EvaluationPoints
!dec\$attributes reference :: ShearForces,Moments
!dec\$attributes reference :: Deflections
\end{lstlisting}

P� grund av att indatavariablerna till \fmethod{calc} beskriver en
geometrisk modell av den kontinuerliga balken, best�r stora delar
av rutinen av kod f�r att skapa topologi och randvillkor.
\fmethod{calc} anv�nder ocks� dynamisk minnesallokering f�r att
problem av godtycklig storlek skall kunna hanteras. Processen i
\fmethod{calc} kan beskrivas med f�ljande steg:

\begin{enumerate}
\item Definition av element topologi (\fvar{Edof}).%
\item Allokering av global styvhetsmatris och lastvektor.%
\item Ber�kning av styvhetsmatriser och assemblering med
\fmethod{beam2e} och \fmethod{assemElementLoad}.%
\item Definition av randvillkor. %
\item L�sning av ekvationssystem med \fmethod{solveq} och
\fmethod{solve}.%
\item Ber�kning av deformation, moment och tv�rkrafter med
\fmethod{beam2s}. %
\end{enumerate}

\noindent Den kompletta programkoden f�r \textit{calc} och modulen
\textit{beam} kan hittas i bilaga~\ref{app:application_source}.

%---------------------------------------------------------------------
\subsection{Ber�knings-DLL i Fortran}
%---------------------------------------------------------------------

Det dynamiska l�nkbiblioteket eller DLL-filen skapas genom att
skapa ett ''Fortran Dynamic Link Library'' projekt i Visual
Fortran med namnet \pfname{beamcalc}. Detta g�rs enligt kapitel
6.1. Huvudrutinen \fmethod{calc} l�ggs till projektet som
\ffname{calc.f90} och modulen \fmethod{beam} som
\ffname{beam.f90}. N�r projektet kompileras skapas filen
\ffname{beamcalc.dll} i katalogen \ffname{Debug} eller
\ffname{Release} i projektkatalogen. F�r att kunna koppla denna
DLL-fil till Delphi m�ste information om namn p� de tillg�ngliga
rutinerna tas fram. Information om dessa f�s som i
avsnitt~\ref{sec:exported_functions} genom att anv�nda
''QuickView'' eller kommandot \cli{dumpbin} vid kommandoprompten.
kommandot \cli{Dumpbin} listar f�ljande om \textit{beamcalc.dll}:

\begin{quotation}
\begin{small}
\noindent
\begin{verbatim}
Microsoft (R) COFF Binary File Dumper Version 6.00.8447
Copyright (C) Microsoft Corp 1992-1998. All rights reserved.


Dump of file beamcalc.dll

File Type: DLL

  Section contains the following exports for beamcalc.dll

           0 characteristics
    38BEA316 time date stamp Thu Mar 02 18:21:26 2000
        0.00 version
           1 ordinal base
           2 number of functions
           2 number of names

    ordinal hint RVA      name

          2    0 0000101E _calc@60
          1    1 00001028 calc

  Summary

        1000 .data
        1000 .idata
        1000 .rdata
        1000 .reloc
        9000 .text
\end{verbatim}
\end{small}
\end{quotation}

Det �r deklarationen \ftype{\_calc@60} som skall anv�ndas i
deklarationen av ber�kningsdelen i huvudprogrammet. Siffran 60
utrymmet som beh�vs f�r att f�ra �ver variablerna i anropet.

%---------------------------------------------------------------------
\section{Huvudprogram (Delphi)}
%---------------------------------------------------------------------

Huvudprogrammet f�r programmet skrivs i Borland Delphi. F�ljande
delar kommer att beh�vas huvudprogrammet:

\begin{xlist}
\item Huvudformul�r med uppritning av geometri, laster, randvillkor och resultat. (\pfname{frmMain.dfm, main.pas}) %
\item Formul�r f�r hantering av en materiallista. (\pfname{frmMaterials.dfm, beammat.pas}) %
\item Formul�r f�r att modifiera geometri och last. (\pfname{frmBeamProp.dfm, beamprop.pas}) %
\item Formul�r f�r att modifiera randvillkor. (\pfname{frmBCs.dfm, beambc.pas}) %
\item Formul�r f�r presentation av resultat i numerisk form. (\pfname{frmResults.dfm, beamresult.pas}) %
\item Programenhet (unit) f�r hantering av balkmodell. (\pfname{beammodel.pas}) %
\item Programenhet (unit) f�r hantering av uppritning. (\pfname{beamdraw.pas}) %
\item Programenhet (unit) f�r interface till ber�knings-DLL. (\pfname{beamcalc.pas}) %
\end{xlist}

Alla funktioner f�r hantering av balkmodellen �r inkapslade i tre
programenheter (units) \punit{BeamModel}, \punit{BeamDraw} och
\punit{BeamCalc}. Dessa programenheter �r skrivna enligt
''svarta-l�dan'' principen. Med detta menas att alla variabler i
programenheterna endast kan n�s genom ett antal underprogram. Med
denna metod kapslas implementeringen in och det �r l�ttare att i
ett senare skede �ndra denna. Relationerna mellan de olika
programenheterna visas i figur~\ref{fig:unit_relations}.

\fignormal{kompendiumv4Fig184.eps}{Relationer mellan
programenheter}{fig:unit_relations}

Genom att rutinerna f�r uppritning och hantering av balkmodell
skiljs fr�n anv�ndargr�nssnittskoden blir programmet l�ttare att
underh�lla. Det �r ocks� l�ttare att �ndra anv�ndargr�nssnittet
eftersom h�ndelserutinerna endast inneh�ller minimalt med
funktionalitet.

%---------------------------------------------------------------------
\subsection{Programenheten BeamModel}
%---------------------------------------------------------------------

\punit{BeamModel} inneh�ller alla funktioner som beh�vs f�r att
hantera beskrivningen av en kontinuerlig balk. Enheten inneh�ller
ocks� funktioner f�r att l�sa och skriva modellen till fil samt
att k�ra ber�kningen med hj�lp av enheten \punit{BeamCalc}.
Lagringen av modellen sker i ett antal vektorer och matriser
definierade med samma datatyper som i Fortran-koden. Detta g�r att
vi enkelt kan anropa ber�kningsdelen utan att beh�va anv�nda
tempor�ra variabler. Deklarationen av datatyperna visas i f�ljande
kodutdrag.

\begin{lstlisting}
NumberOfBeams      : integer;
NumberOfMaterials  : integer;
BeamLengths        : TDoubleBeamVector;
BeamLoads          : TDoubleBeamVector;
BeamProps          : TIntBeamVector;
.
.
BCTypes            : TIntBCVector;
BCDisplValues      : TDoubleBCVector;
BCRotValues        : TDoubleBCVector;
BCPositions        : TDoubleBCVector;

Materials          : TMaterialMatrix;

Reaction           : TReactionVector;
Displacement       : TDisplacementVector;
ShearForces        : TResultMatrix;
Moments            : TResultMatrix;
Deflections        : TResultMatrix;
\end{lstlisting}

Rutiner f�r att hantera balkmodellen kan delas in i ett antal
kategorier:

\begin{xlist}
\item Materialhantering %
\item Balkrutiner %
\item Randvillkor %
\item Resultat %
\item Modellhantering %
\end{xlist}

F�r att p� ett enkelt s�tt hantera materialegenskaper kan en lista
av material l�ggas upp. Varje balksegment i modellen refererar
sedan till n�got av dessa. Hanteringen av materialen sker med
hj�lp av fem rutiner visade i tabell~\ref{tbl:material_handling}.

\begin{table}[!htb]
\begin{center}
\begin{tabular}{|p{0.4\textwidth}|p{0.5\textwidth}|}
\hline
Rutin &  Beskrivning \\
\hline
\pmethod{AddMaterial(E, A, I : double)} &  L�gger till ett material sist i material listan. \\
\hline
\pmethod{RemoveMaterial(idx : integer)} &  Tar bort materialet p� positionen idx. Efterf�ljande material flyttas upp ett steg i listan. \\
\hline
\pmethod{SetMaterial(idx : integer; E, A, I : double)} &   Tilldelar ett material p� positionen idx i listan givna egenskaper. \\
\hline
\pmethod{GetMaterial(idx : integer; var E, A, I : double)} &   Tilldelar variablerna E, A, I egenskaperna f�r materialet i position idx. \\
\hline
\pmethod{GetNumberOfMaterials : integer} & Returnerar antalet material i listan. \\
\hline
\end{tabular}
\end{center}
\caption{Rutiner f�r hantering av material} \label{tbl:material_handling}
\end{table}


Varje balksegment beskrivs med en l�ngd, utbreddlast och ett
material index. Rutinerna f�r hantering av balksegment �r snarlika
dem f�r material, se tabell~\ref{tbl:beam_segment_handling}.

\begin{table}[!htb]
\begin{center}
\begin{tabular}{|p{0.4\textwidth}|p{0.5\textwidth}|}
\hline
Rutin &  Beskrivning \\
\hline
\pmethod{AddBeam(l, q : double; material : integer)} &   L�gger till ett balksegment p� slutet med l�ngden l, lasten q och materialindex material. \\
\hline
\pmethod{RemoveBeam(idx : integer)} &    Tar bort balksegmentet p� position idx. Efterf�ljande balkar flyttas upp. \\
\hline
\pmethod{SetBeam(idx : integer; l, q : double; material : integer)} &    Tilldelar balksegmentet idx l�ngden l och lasten q. \\
\hline
\pmethod{GetBeam(idx : integer; var l, q : double; var material : integer)} &    Tilldelar l, q och material v�rden f�r balksegmentet idx. \\
\hline
\pmethod{GetNumberOfBeams : integer} &   Returnerar antalet balksegment f�r modellen. \\
\hline
\pmethod{GetTotalLength : double} & Returnerar totall�ngden f�r den kontinuerliga balken. \\
\hline
\end{tabular}
\end{center}
\caption{Rutiner f�r hantering av balksegment} \label{tbl:beam_segment_handling}
\end{table}

Randvillkoren f�r balkmodellen lagras per nod. Antalet randvillkor
som kan definieras �r d� \pmethod{GetNumberOfBeams} + 1. En nod
kan antingen vara fri eller ha f�reskrivna f�rskjutningar eller
rotationer. Randvillkorstypen best�ms av konstanterna
\ptype{bcFree, bcFixedDispl, bcFixedRot} och \ptype{bcFixed}.
Hanteringen av randvillkoren sker med rutinerna beskrivna i
tabell~\ref{tbl:bc_handling}.

\begin{table}[!htb]
\begin{center}
\begin{tabular}{|p{0.4\textwidth}|p{0.5\textwidth}|}
\hline
Rutin &  Beskrivning \\
\hline
\pmethod{SetBC(idx : integer; bcType : integer; bcDisplValue, bcRotValue : double)} &    Tilldelar noden idx ett randvillkor av typen bcType. bcDisplValue och bcRotValue anger v�rdet p� eventuella f�reskrivna f�rskjutningar. \\
\hline
\pmethod{GetBC(idx : integer; var bcType : integer; var bcDisplValue, bcRotValue : double)} &    Tilldelar bcType, bcDisplValue och bcRotValue randvillkorsegenskaperna f�r nod idx. \\
\hline
\pmethod{GetNodePos(idx : integer) : double} &   Returnerar avst�ndet fr�n b�rjan av balken till nod idx. \\
\hline
\end{tabular}
\end{center}
\caption{Rutiner f�r hantering av randvillkor} \label{tbl:bc_handling}
\end{table}

Ber�kning av en balkmodell sker med hj�lp av rutinen
\pmethod{Execute}. Genom att vi har anv�nt ''svarta-l�dan''
principen beh�ver en anv�ndare av \punit{BeamModel} aldrig veta
detaljerna om hur anropet till ber�kningsdelen g�r till.
Ber�kningen exekveras genom ett enkelt anrop till \pmethod{Execute}
utan n�gra parametrar. Genom att de variabler som beh�vs f�r
ber�kningen �r synliga f�r \pmethod{Execute} besparas anv�ndare
besv�ret att direkt anropa programenheten \punit{BeamCalc} med
alla variabler som beh�vs f�r detta. Efter det att ber�kningen har
avslutats ber�knar ocks� \pmethod{Execute} max-v�rden p� alla
resultat, s� att dessa �r tillg�ngliga vid uppritningen av
modellen. Programkoden f�r \pmethod{Execute} visas nedan.

\begin{lstlisting}[texcl]
procedure Execute;
var
    i,j : integer;
begin

  // Anropa fortran kod f�r ber�kning

  BeamCalc.Calc(
    NumberOfBeams,
    NumberOfMaterials,
    BeamLengths,
    BeamLoads,
    BeamProps,
    BCTypes,
    BCDisplValues,
    BCRotValues,
    Materials,
    Displacement,
    Reaction,
    MaxEvaluationPoints,
    ShearForces,
    Moments,
    Deflections);

  // Ber�kna max v�rden f�r skalning

  MaxMoment:=-1e300;
  MaxDeflection:=-1e300;
  MaxShearForce:=-1e300;

  for i:=1 to NumberOfBeams do
  begin
    for j:=1 to MaxEvaluationPoints do
    begin
      if abs(Moments[i,j])>MaxMoment then
        MaxMoment:=abs(Moments[i,j]);
      if abs(Deflections[i,j])>MaxDeflection then
        MaxDeflection:=abs(Deflections[i,j]);
      if abs(ShearForces[i,j])>MaxShearForce then
        MaxShearForce:=abs(ShearForces[i,j]);
    end;
  end;
end;
\end{lstlisting}

N�r en ber�kning avslutats finns alla resultaten lagrade i
\punit{BeamModel}. �tkomst av dessa variabler sker genom genom
rutinerna beskrivna i tabell~\ref{tbl:result_handling}.

\begin{table}[!htb]
\begin{center}
\begin{tabular}{|p{0.4\textwidth}|p{0.5\textwidth}|}
\hline
Rutin &  Beskrivning \\
\hline
\pmethod{GetMoment(BeamIdx, PointIdx : integer) : double} & Returnerar momentet i punkten PointIdx f�r balken BeamIdx. BeamIdx m�ste vara <= GetNumberOfBeams. PointIdx m�ste vara mindre �n konstanten MaxEvaluationPoints vilken finns definierad i BeamModel. \\
\hline
\pmethod{GetDeflection(BeamIdx, PointIdx : integer) : double} &  Returnerar deformationen i punkten PointIdx f�r balken BeamIdx. \\
\hline
\pmethod{GetShearForce(BeamIdx, PointIdx : integer) : double} &  Returnerar tv�rkraften i punkten PointIdx f�r balken BeamIdx. \\
\hline
\pmethod{GetMaxDeflection : double} &    Returnerar max utb�jning f�r alla balksegment. \\
\hline
\pmethod{GetMaxMoment : double} &    Returnerar max moment f�r alla balksegment. \\
\hline
\pmethod{GetMaxShearForce : double} &    Returnerar max tv�rkraft f�r alla balksegment. \\
\hline
\end{tabular}
\end{center}
\caption{Rutiner f�r hantering av resultat} \label{tbl:result_handling}
\end{table}

\punit{BeamModel} innh�ller ocks� funktioner f�r att l�sa och
skriva modellen till fil. Filformatet beskrivs nedan:

\begin{psyntax}
$\{Antalet~material~n\}$\\
$\{E_1~A_1~I_1\}$\\
. \\
. \\
$\{E_n~A_n~I_n\}$\\
$\{Antalet~balksegment~k\}$\\
$\{L_1~q_1~Materialidx_1\}$\\
. \\
. \\
$\{L_k~q_k~Materialidx_k\}$\\
$\{Randvillkorstyp_1~Forskjutning_1~Rotation_1\}$ \\
. \\
. \\
$\{Randvillkorstyp_{k+1}1~Forskjutning_{k+1}~Rotation_{k+1}\}$ \\
\end{psyntax}

Hanteringen av balkmodellen sker med f�ljande funktionerna i
tabell~\ref{tbl:model_handling}.

\begin{table}[!htb]
\begin{center}
\begin{tabular}{|p{0.4\textwidth}|p{0.5\textwidth}|}
\hline
Rutin &  Beskrivning \\
\hline
\pmethod{SetModelName(FileName : string)} & S�tter filnamnet f�r balkmodellen. \\
\pmethod{GetModelName : string} &    Returnerar filnamnet f�r modellen. \\
\pmethod{NewModel} &     Raderar befintliga data i balkmodellen. \\
\pmethod{Save} &     Sparar modellen med filnamn angivet med \pmethod{SetModelName}. Om filnamn ej �r angivet sparas modellen med namnet \pfname{noname.bml}. \\
\pmethod{Open} &     �ppnar modellen med filnamn angivet med \pmethod{SetModelName}. Om filnamn ej �r angivet �ppnas modellen \pfname{noname.bml}. \\
\hline
\end{tabular}
\end{center}
\caption{Modellhanteringsrutiner} \label{tbl:model_handling}
\end{table}

%---------------------------------------------------------------------
\subsection{Programenheten BeamCalc}
%---------------------------------------------------------------------

\punit{BeamCalc} �r kopplingen mellan huvudprogrammet och
ber�kningsdelen. I denna deklareras subrutinen \pmethod{calc} i det
dynamiskt l�nkade Fortran-biblioteket. En koppling till ett
dynamiskt l�nkat bibliotek i Delphi brukar oftast g�ras genom att
definiera en programenhet med samma namn som biblioteket. I denna
programenhet placeras sedan deklarationerna av de ing�ende
rutinerna precis som vanligt efter \pkeyw{interface}-delen i
programenheten. I \textit{implementeringsdelen} placeras ist�llet
f�r den kompletta rutinen en extern referens till en DLL. F�ljande
kod visar hur detta set ut f�r \pmethod{calc}-rutinen i
ber�kningsdelen.

\begin{lstlisting}
.
.

interface

uses BeamModel;

procedure Calc(
  nBeams,
  nMaterials : integer;
  var BeamLengths : TDoubleBeamVector;

  .
  .

implementation

procedure Calc(
  nBeams,
  nMaterials : integer;
  var BeamLengths : TDoubleBeamVector;

  .
  .

  var Moments : TResultMatrix;
  var Deflections : TResultMatrix);
  stdcall;
  external '..\fortran\beamcalc\debug\beamcalc.dll'
  name '_calc@60';
\end{lstlisting}

Direktivet \pkeyw{stdcall} anger vilken anropskonvention som
anv�nds, i detta fall \pkeyw{stdcall}. \pkeyw{external}  anger att
rutinen finns implementerad i \pfname{beamcalc.dll} med namnet
\pname{\_calc@60}. S�kv�gen angiven i ovanst�ende kod �r bra att
ha under utvecklingen av programmet, eftersom man ej hela tiden
beh�ver kopiera DLL-filen mellan Visual Fortran-projektet och
Delphi-projektet. N�r programmet �r f�rdigutvecklat b�r endast
namnet \pfname{beamcalc.dll} st� kvar i deklarationen. L�ter man
s�kv�gen vara kvar m�ste DLL-filen ligga precis enligt
deklarationen f�r att programmet skall fungera.

\punit{BeamCalc} anv�nder de datatyper som �r definierade i
\punit{BeamModel} f�r att deklarera de matriser och vektorer som
skall f�ras �ver. Variabler som �verf�rs med referens motsvaras av
en \pkeyw{var}-deklaration i Pascal.

%---------------------------------------------------------------------
\subsection{Programenheten BeamDraw}
%---------------------------------------------------------------------

\punit{BeamDraw} hanterar all uppritning i programmet. F�r att
uppritningen skall kunna hanteras p� ett flexibelt s�tt har den
delats upp i ett antal olika rutiner. Denna uppdelning g�r att de
olika rutinerna kan kombineras p� olika s�tt f�r att skapa �nskat
resultat. F�r att uppritningsrutinerna skall vara oberoende av
n�got programf�nster, tar alla rutiner en rityta av typen
\ptype{TCanvas} som indata. Detta g�r att uppritningsrutiner kan
styras, som att t ex rita direkt p� en rityta f�r en skrivare.
F�ljande uppritningsrutiner finns i \punit{BeamDraw}.

\begin{table}[!htb]
\begin{center}
\begin{tabular}{|l|p{0.4\textwidth}|}
\hline
Rutin &  Beskrivning \\
\hline
\pmethod{DrawBackground(ACanvas : TCanvas)} &    Uppritning av bakgrund. \\
\pmethod{DrawGeometry(ACanvas : TCanvas)} &  Ritar upp balkgeometrin. \\
\pmethod{DrawLoads(ACanvas : TCanvas)} &     Ritar upp lasterna. \\
\pmethod{DrawBCs(ACanvas : TCanvas)} &   Ritar upp randvillkor. \\
\pmethod{DrawDimensions(ACanvas : TCanvas)} &    Ritar upp m�ttlinjer. \\
\pmethod{DrawDeflections(ACanvas  : TCanvas)} &  Ritar upp deformationer. \\
\pmethod{DrawMoments(ACanvas : TCanvas)} &   Ritar upp moment. \\
\pmethod{DrawShearForces(ACanvas : TCanvas)} &   Ritar upp tv�rkrafter. \\
\hline
\end{tabular}
\end{center}
\caption{Uppritningsrutiner i \punit{BeamDraw}} \label{tbl:draw_handling}
\end{table}

F�r att styra hur stor rityta som finns tillg�nglig anv�nds
f�ljande rutin

\begin{psyntax}
SetDrawArea(width, height : \textbf{integer})
\end{psyntax}

Denna anropas l�mpligtvis vid h�ndelserna \pprop{FormShow} och
\pprop{FormResize}.

%---------------------------------------------------------------------
\subsection{Huvudformul�r}
%---------------------------------------------------------------------

Huvudformul�ret i programmet skall inneh�lla meny, verktygsf�lt
och en rityta. Genom att funktionalitetet i programmet delats upp
p� ett antal programenheter kommer huvudformul�ret att endast en
inneh�lla begr�nsad m�ngd kod och endast anrop till
programenheterna. F�r att ytterligare minska m�ngden kod som
beh�ver skrivas kommer s� kallade h�ndelselistor att anv�ndas.
H�ndelselistan (ActionList) �r en ny komponent som inf�rdes i
Delphi 4. I denna komponent kan man samla alla h�ndelser i en
central lista. Menyer och verktygsf�lt kan sedan referera till
denna lista ist�llet f�r att hantera egna h�ndelser. Tidigare
beh�vdes tv� h�ndelser f�r en och samma funktion om samma funktion
�terfanns i b�de menyn och p� verktygsf�ltet.

H�ndelselistan skapas genom att v�lja
\figbutton{btn_comp_actionlist.eps} i komponentpaletten under
fliken standard och klicka p� formul�ret. Komponenten ges namnet
\pobject{actMain}. Redigering av listan g�rs genom att
dubbelklicka p� denna, d� visas ett egenskapsf�nster.

\figsmall{kompendiumv4Fig185.eps}{Egenskapsf�nster f�r
h�ndelselista}{fig:action_list_edit1}

En ny tom h�ndelse skapas genom att klicka i vertygsf�ltet, d�
visas f�ljande i egenskapsf�nstret.

\figsmall{kompendiumv4Fig186.eps}{Ny h�ndelse}{fig:new_action}

Egenskaper f�r h�ndelser redigeras i objektinspektorn, precis som
andra komponenter i Delphi. Figur~\ref{fig:action_property} visar
egenskaperna f�r en h�ndelse (Action).

\figsmall{action_property.eps}{Egenskaper f�r en
h�ndelse}{fig:action_property}

\pprop{Caption} �r en beskrivning som anv�nds av de kontroller,
som st�djer detta t ex menyer och knappar. \pprop{Category} saknar
egentligen funktion, men anv�nds f�r att h�lla ordning p� en stor
m�ngd h�ndelser. \pprop{Checked} anv�nds p� samma s�tt som vid en
kryssruta. �r denna satt till \pkeyw{true} anger detta att ett
alternativ �r valt. Visning av detta �r beroende p� vilken
kontroll h�ndelsen �r kopplad till. Egenskapen \pprop{Enabled}
aktiverar eller deaktiverar ett alternativ. Detta visas ofta genom
att alternativet blir gr�tt i kontroller. \pprop{ImageIndex} anger
vilken bild i en \pprop{ImageList}-kontroll, som skall kopplas
till ett givet alternativ. F�ljande tabell visar vilka h�ndelser
och egenskaper som �r definierade f�r exemplet.

\begin{table}[!htb]
\begin{center}
\begin{tabular}{|l|l|l|l|}
\hline
Name  & Category    & ImageIndex  & Caption \\
\hline
\pobject{actNew}  & File    & 0   & Ny balk \\
\pobject{actOpen} & File    & 1   & �ppna... \\
\pobject{actSave} & File    & 2   & Spara \\
\pobject{actSaveAs}   & File    & 2   & Spara som... \\
\pobject{actExit} & File    -& 1  & Avsluta \\
\pobject{actMaterials}    & Input   & 4   & Material... \\
\pobject{actProperties}   & Input   & 3   & Balk egenskaper \\
\pobject{actBC}   & Input   & 5   & Randvillkor... \\
\pobject{actDeflections}  & Result  & 10  & Nedb�jningar \\
\pobject{actMoments}  & Result  & 11  & Moment \\
\pobject{actShearForces}  & Result  & 12  & Tv�rkrafter \\
\pobject{actNumerical}    & Result  & 13  & Numeriskt... \\
\pobject{actCalc} & Calc    & 7   & Ber�kna... \\
\pobject{actAddSegment}   & Input   & 8   & L�gg till balk \\
\pobject{actRemoveSegment}    & Input   & 9   & Ta bort balk \\
\hline
\end{tabular}
\end{center}
\caption{H�ndelser f�r balkprogrammet} \label{tbl:actions}
\end{table}


En \pobject{ImageList}-kontroll \pobject{imlMain} anv�nds f�r att
lagra de bilder som skall finnas i menyn och verktygsf�ltet. Se
avsnitt~\ref{sec:imagelist_control} f�r en mer detaljerad
beskrivning av \pobject{ImageList}-kontrollen.
Figur~\ref{fig:imagelist_images} visar bilderna som anv�nds i
\pobject{imlMain}.

\fignormal{images_in_imagelist.eps}{Bilder i
ImageList-kontroll}{fig:imagelist_images}

Verktygsf�lt och menyer kopplas nu till h�ndelselistan.
Menykontrollen \pobject{mnuMain} l�ggs till och motsvarande menyer
f�r h�ndelserna skapas. Kopplingen av h�ndelser till menyn sker
genom egenskapen \pprop{Action} i objektinspektorn.
Figur~\ref{fig:connecting_menu_action} visar hur en h�ndelse
\pobject{actNew} kopplas till menyalternativet \menuitem{Arkiv/Ny
balk}.

\fignormal{kompendiumv4Fig187.eps}{Koppling av meny till
h�ndelse}{fig:connecting_menu_action}

F�r att bilderna skall synas i alla kontroller, m�ste egenskapen
\textit{Images} s�ttas f�r alla kontroller. Nytt f�r Delphi 4 �r
att menyer kan inneh�lla bilder p� samma s�tt som i Microsoft Word
eller Visual Fortran. Detta ger en mer konsekvent koppling mellan
menyer och verktygsf�lt.

Verktygsf�ltet i balkprogrammet kommer att vara en kombination av
komponenterna \pobject{ControlBar} och \pobject{Toolbar}.
\pobject{ControlBar}-kontrollen \pobject{ctlbToolbars} kommer att
hantera verktygsf�lten i balkprogrammet. Kontrollen placeras med
hj�lp av egenskapen \pprop{Align = alTop} under menyn. Egenskapen
\pprop{AutoSize} s�tts ocks� till \pobject{true} f�r att
automatisk skalning till verktygsf�ltens storlekar skall ske.
Egenskapen \pprop{BevelInner} s�tts till \pvar{bvNone} f�r att
vertygsf�lten skall passa in b�ttre layoutm�ssigt.
Verktygsf�lteten skapas genom att v�lja en Toolbar-kontroll under
fliken \pobject{Win32} i komponentpaletten och klicka ut denna p�
\pobject{ControlBar-}kontrollen. F�r att verktygsf�lten skall f�
ett mer korrekt utseende s�tts egenskapen
\pprop{EdgeBorders/ebTop} till \pkeyw{false} och \pprop{Flat} till
\pkeyw{true}. Knappar i verktygsf�lten l�ggs till genom att
markera detta och klicka h�gerknappen och v�lja \menuitem{New
Button}. Knapparna knyts p� samma s�tt som menyerna till
h�ndelserna genom egenskapen \pprop{Action}. P� detta s�tt skapas
nu f�ljande verktygsf�lt vilka kopplas till motsvarande
h�ndelselistor:

\fignormal{kompendiumv4Fig188.eps}{Verktygsf�lt i
balkprogram}{fig:beam_toolbars}

Den sista komponenten som skall placeras p� huvudformul�ret �r en
ram (Bevel) f�r att markera ritytan. Egenskapen \pprop{Align} f�r
ramen s�tts till \ptype{alClient}.

Nu har alla delar av gr�nssnittet placerats ut och konfigurerats
grafiskt. N�sta steg �r att koppla kod till h�ndelser och
h�ndelselistor i programmet. Innan kod f�r h�ndelserna i
programmet l�ggs till skall ett par variabler l�ggas till
formul�rdeklarationen:

\begin{lstlisting}
.
.
private
  { Private declarations }
  FHaveName : boolean;
  FDirty : boolean;
  FShowMoments : boolean;
  FShowShearForces : boolean;
  FShowDeflections : boolean;
public
  { Public declarations }
end;
\end{lstlisting}

\pvar{FHaveName} anger om anv�ndare tidigare d�pt och sparat
modellen. Denna variabel anv�nds f�r att avg�ra om alternativet
spara beh�ver �ppna en dialogruta f�r att fr�ga efter ett filnamn.
\pvar{FDirty} anger om modellegenskaperna har modifierats efter
det att en ber�kning har utf�rts. Det tre sista variablerna kommer
att anv�ndas f�r att ange vilka resultat som skall ritas upp
samtidigt.

F�r att kunna anv�nda de tidigare beskrivna programenheterna
(unit:s) \punit{BeamModel} och \punit{BeamDraw} m�ste dessa l�ggas
till i formul�rets \pkeyw{use} deklaration.

\begin{lstlisting}[escapechar=\_]
uses
  Windows, Messages, SysUtils, Classes, ...,
  Forms, Dialogs, ActnList, ImgList, ...,
  ExtCtrls, StdCtrls, Menus, _\underline{BeamModel, BeamDraw};
\end{lstlisting}

N�r programmet startas m�ste ofta variabler initieras och andra
inst�llningar g�ras. Detta l�ggs l�mpligast i formul�rets
\pprop{OnCreate} h�ndelse. Denna h�ndelse kan tilldelas genom
dubbelklicka p� huvudformul�ret. F�ljande kod l�ggs in i
pprop{FormCreate}.

\begin{lstlisting}
procedure TfrmMain.FormCreate(Sender: TObject);
begin
  BeamModel.AddMaterial(1.0, 1.0, 1.0);
  BeamModel.AddBeam(4.0, -1.0, 1);
  BeamModel.SetBC(1, bcFixedDispl, 0.0, 0.0);
  BeamModel.SetBC(2, bcFixedDispl, 0.0, 0.0);
  FHaveName:=false;
  FShowMoments:=false;
  FShowDeflections:=true;
  actDeflections.Checked:=true;
  FShowShearForces:=false;
  FDirty:=true;
end;
\end{lstlisting}

De f�rsta 4 raderna skapar en enkel balk, fritt upplagd p� tv�
st�d. \pvar{FHaveName} s�tts till \pkeyw{false} f�r att
indikera att vi ej har sparat modellen. Vid en ber�kning v�ljs att
visa deformationerna som f�rinst�llt v�rde. \pvar{FDirty} s�tts
till \pkeyw{true} f�r att ange att en ber�kning ej har gjorts.

F�r att balkmodellen skall synas i formul�ret m�ste formul�rets
\pprop{OnPaint} h�ndelse tilldelas kod. Detta g�rs genom att
v�lja huvudformul�ret i objekt-inspektorn och dubbelklicka p�
\pprop{OnPaint} under fliken \guitab{Events}. I denna h�ndelse
anv�nder vi de rutiner som finns i \punit{BeamDraw} f�r att rita
upp balkmodellen. Varje formul�r har en rityta (Canvas). Denna
rityta skickar vi som indata till ritrutinerna. Dessa kommer d�
att rita direkt p� formul�ret. Koden f�r \pprop{OnPaint} visas
nedan.

\begin{lstlisting}
procedure TfrmMain.FormPaint(Sender: TObject);
begin
  BeamDraw.DrawBackground(Canvas);
  BeamDraw.DrawLoads(Canvas);
  BeamDraw.DrawBCs(Canvas);
  BeamDraw.DrawGeometry(Canvas);
  BeamDraw.DrawDimensions(Canvas);
  if (not FDirty) then
  begin
    if (FShowDeflections) then
      BeamDraw.DrawDeflections(Canvas);
    if (FShowMoments) then
      BeamDraw.DrawMoments(Canvas);
    if (FShowShearForces) then
      BeamDraw.DrawShearForces(Canvas);
  end;
end;
\end{lstlisting}

Uppritning av resultat sker endast om \pvar{FDirty} �r satt till
\pkeyw{false}, f�r att f�rhindra visning av resultat som ej
motsvarar modellegenskaperna. Kompileras och k�rs programmet i
detta skick kommer huvudf�nstret att se ut som figur~\ref{fig:invalid_drawing}.

\fignormal{kompendiumv4Fig189.eps}{Resultat av k�rning}{fig:invalid_drawing}

L�gg m�rke till att alla kontroller �r ``gr�ade''. Detta beror p�
att ingen kod �r kopplad till h�ndelser i h�ndelselistan.
Anledningen till att balkmodellen endast ritas i �vre v�nstra
h�rnet �r att ritytan ej har definierats i \punit{BeamDraw} med
rutinen \pmethod{SetDrawArea}. F�r att hantera detta definieras
h�ndelserna \pprop{OnShow} och \pprop{OnResize}. \pprop{OnShow}
anropas n�r f�nstret f�rst visas. \pprop{OnResize} anropas n�r
f�nstret �ndrar storlek. F�ljande kod visar dessa h�ndelserutiner.

\begin{lstlisting}
procedure TfrmMain.FormResize(Sender: TObject);
begin
  BeamDraw.SetDrawArea(ClientWidth, ClientHeight);
end;

procedure TfrmMain.FormShow(Sender: TObject);
begin
  BeamDraw.SetDrawArea(ClientWidth, ClientHeight);
end;
\end{lstlisting}

\pprop{ClientHeight} och \pprop{ClientWidth} anger storleken p�
det inre omr�det av f�nstret. Kompileras och k�rs programmet nu
ser f�nstret ut som i figur~\ref{fig:correct_resize}.

\fignormal{kompendiumv4Fig190.eps}{Formul�r med \pprop{OnResize}
och \pprop{OnShow} implementerade}{fig:correct_resize}

Det som �terst�r �r att koppla kod till h�ndelselistan. Detta g�rs
genom att f�rst ta fram egenskapsf�nstret f�r h�ndelselistan.
D�refter dubbelklickar man p� en h�ndelse (Action) f�r att koppla
kod till denna. Den f�rsta h�ndelsen vi kopplar kod till �r
\menuitem{Arkiv/Ny balk}. Denna h�ndelse skall radera modellen och
�terst�lla programmet till startl�get igen. F�ljande kod anges:

\begin{lstlisting}
procedure TfrmMain.actNewExecute(Sender: TObject);
begin
  BeamModel.NewModel;
  BeamModel.AddMaterial(1.0, 1.0, 1.0);
  BeamModel.AddBeam(4.0, 0.0, 1);
  BeamModel.SetBC(1, bcFixedDispl, 0.0, 0.0);
  BeamModel.SetBC(2, bcFixedDispl, 0.0, 0.0);
  FHaveName:=false;
  Self.Invalidate;
end;
\end{lstlisting}

Detta �r i princip samma kod som i h�ndelsen \pprop{OnCreate} med
skillnaden att \pvar{BeamModel.NewModel} anropas f�r att
nollst�lla balkmodellen. \pmethod{Self.Invalidate} anger att
f�nstret beh�ver ritas om.

F�r att koppla kod till n�sta tre h�ndelser beh�ver tv�
dialog-kontroller skapas. En \pobject{OpenDialog} som ges namnet
\pobject{dlgOpen} och en \pobject{SaveDialog} med namnet
\pobject{dlgSave.} I h�ndelsen f�r \menuitem{Arkiv/�ppna...} visas
f�rst en dialogruta f�r att anv�ndaren skall kunna v�lja vilken
fil som skall �ppnas. Efter detta anv�nds rutinerna i
\punit{BeamModel} f�r att �ppna den valda modellen.

\begin{lstlisting}
procedure TfrmMain.actOpenExecute(Sender: TObject);
begin
  if dlgOpen.Execute then
  begin
    BeamModel.NewModel;
    BeamModel.SetModelName(dlgOpen.FileName);
    BeamModel.Open;
    FHaveName:=true;
    Self.Invalidate;
  end;
end;
\end{lstlisting}

\pvar{FHaveName} s�tts till \pkeyw{true} eftersom vi har f�tt ett
filnamn. F�nstret uppdateras sen med hj�lp av
\pmethod{Self.Invalidate}. H�ndelsen f�r \menuitem{Arkiv/Spara
som...} ser ut som f�r att �ppna en fil med den skillnaden att
\pobject{dlgSave} anropas ist�llet f�r \pobject{dlgOpen} och att
rutinen \pmethod{Save} i \punit{BeamModel} anropas. Arkiv/Spara
kontrollerar f�rst om modellen tidigare sparats genom att
kontrollera flaggan \pvar{FHaveName}. Om ett filnamn har
tilldelats sparas modellen utan vidare fr�gor, annars anropas
rutinen f�r Arkiv/Spara.

\begin{lstlisting}
procedure TfrmMain.actSaveExecute(Sender: TObject);
begin
  if FHaveName then
    BeamModel.Save
  else
    begin
      actSaveAsExecute(Self);
    end;
end;
\end{lstlisting}

Knapparna \keyb{[+]} och \keyb{[-]} anv�nds f�r att l�gga till och
ta bort balksegment p� den kontinuerliga balken. Textrutan anv�nds
f�r att ange l�ngden p� balksegmentet, som skall l�ggas till.
H�ndelserna f�r dessa knappar anv�nder rutinerna
\pmethod{BeamModel.AddBeam} och \pmethod{BeamModel.RemoveBeam} f�r
att uppdatera modellen. H�ndelsen f�r att l�gga till ett
balksegment visas nedan.

\begin{lstlisting}
procedure TfrmMain.actAddSegmentExecute(Sender: TObject);
begin
  BeamModel.AddBeam(
    StrToFloat(edtNewLength.Text), 0.0, 1);
  SetBC(GetNumberOfBeams+1,bcFixedDispl,0.0,0.0);
  FDirty:=true;
  Self.Invalidate;
end;
\end{lstlisting}

H�ndelsen f�r att ta bort ett segment ser ut p� liknande
s�tt. \\

Ber�kning av balken sker i h�ndelsen \pprop{actCalc}. Ber�kningen
anropas genom att anropa \pmethod{BeamModel.Execute}. Efter det
att ber�kningen �r genomf�rd s�tts \pvar{FDirty} till
\pkeyw{false} f�r att ange att modellen �r aktuell och att
resultaten kan ritas upp. Koden f�r h�ndelsen visas nedan.

\begin{lstlisting}
procedure TfrmMain.actCalcExecute(Sender: TObject);
begin
  BeamModel.Execute;
  FDirty:=false;
  Self.Invalidate;
end;
\end{lstlisting}

H�ndelserna f�r Resultat/Deformationer, Resultat/Moment och
Resultat/\-Tv�rkrafter s�tter flaggorna f�r visning av olika
resultat och beg�r att f�nstret skall ritas om. Koden f�r
Resultat/Deformationer visas nedan.

\begin{lstlisting}
procedure TfrmMain.actDeflectionsExecute(Sender: TObject);
begin
  FShowDeflections:=not FShowDeflections;
  actDeflections.Checked:=FShowDeflections;
  Self.Invalidate;
end;
\end{lstlisting}

�vriga h�ndelser f�r visning ser ut p� samma s�tt.
\\
De flesta h�ndelser �r nu implementerade i huvudformul�ret.
�terst�ende h�ndelser behandlas i de kommande kapitlen. Formul�ret
har nu f�ljande utseende.

\fignormal{kompendiumv4Fig191.eps}{Huvduformul�r utan kopplingar
till dialogrutor}{fig:main_form_no_dialogs}

%---------------------------------------------------------------------
\subsection{Balkegenskapsformul�r }
%---------------------------------------------------------------------

I balkegenskapsformul�ret skall egenskaper som l�ngd, utbredd last
och material kunna redigeras f�r varje balksegment. Figur 8{\-}14
visar formul�rets layout samt namn p� ing�ende kontroller.

\figmedium{kompendiumv4Fig192.eps}{Balkegenskapsformul�r}{fig:beam_properties_window}

Formul�ret h�mtar information om balken fr�n programenheten
\punit{BeamModel,} varf�r denna m�ste l�ggas till i
implementeringsdelen av formul�ret.

\begin{lstlisting}
.
.
implementation

{$R *.DFM}

uses BeamModel;
\end{lstlisting}

F�r att hantera uppdatering av kontroller p� ett enkelt s�tt
implementeras tre rutiner f�r detta �ndam�l. En variabel f�r
aktuellt segment deklareras ocks�. F�ljande kod visar hur dessa
rutiner l�ggs till i formul�rdeklarationen.

\begin{lstlisting}
.
.
private
    { Private declarations }
    FCurrentBeam : integer;

    procedure SetData;
    procedure GetData;
    procedure FillListBoxes;
.
.
\end{lstlisting}

N�r dessa rutiner deklareras kan Delphi skriva
implementeringskoden automatiskt genom att placera textmark�ren i
koden och trycka \keyb{[Ctrl]+[Shift]+C}. Rutinen \pmethod{FillListBoxes}
fyller combobox-kontrollerna \pobject{cboSegment} och
\pobject{cboMaterial} med v�rden fr�n \punit{BeamModel}.

\begin{lstlisting}
procedure TfrmBeamProps.FillListBoxes;
var
    i : integer;
begin
  cboSegment.Clear;
  for i:=1 to BeamModel.GetNumberOfBeams do
    cboSegment.Items.Add(IntToStr(i));
  cboSegment.ItemIndex:=0;

  cboMaterial.Clear;
  for i:=1 to BeamModel.GetNumberOfMaterials do
    cboMaterial.Items.Add(IntToStr(i));
  cboMaterial.ItemIndex:=0;
end;
\end{lstlisting}

�vriga kontroller fylls med data i rutinen \textit{SetData}.

\begin{lstlisting}
procedure TfrmBeamProps.SetData;
var
    BeamLength : double;
    BeamLoad : double;
    BeamProp : integer;
begin
  FCurrentBeam:=cboSegment.ItemIndex+1;
  BeamModel.GetBeam(FCurrentBeam, BeamLength, BeamLoad,
                    BeamProp);
  edtLength.Text:=FloatToStr(BeamLength);
  edtLoad.Text:=FloatToStr(BeamLoad);
  cboMaterial.ItemIndex:=BeamProp-1;
end;
\end{lstlisting}

Rutinen \pmethod{GetData} h�mtar data fr�n kontrollerna och lagrar
dessa i \punit{BeamModel}.

\begin{lstlisting}
procedure TfrmBeamProps.GetData;
var
    BeamLength : double;
    BeamLoad : double;
    BeamProp : integer;
begin
  BeamLength:=StrToFloat(edtLength.Text);
  BeamLoad:=StrToFloat(edtLoad.Text);
  BeamProp:=cboMaterial.ItemIndex+1;
  BeamModel.SetBeam(FCurrentBeam, BeamLength, BeamLoad,
                    BeamProp);
end;
\end{lstlisting}

F�r att fylla kontrollerna n�r formul�ret visas har kod kopplats
till h�ndelsen \pprop{OnShow} vilken anropas precis innan
formul�ret skall visas. I denna h�ndelse anropas f�rst
\pmethod{FillListBoxes} f�r att fylla combobox-kontrollerna och
d�refter \pmethod{SetData} f�r att fylla �vriga textrutor.

\begin{lstlisting}
procedure TfrmBeamProps.FormShow(Sender: TObject);
begin
  FillListBoxes;
  SetData;
end;
\end{lstlisting}

Anv�ndaren byter aktuellt balksegment genom att v�lja i listan f�r
kontrollen \pobject{cboSegment}. F�r att f�nga upp detta har kod
kopplats till kontrollens \pprop{OnChange} h�ndelse. I denna
h�mtas f�rst aktuella data med \pmethod{GetData} f�r att lagras i
\punit{BeamModel}. D�refter anropas \pmethod{SetData} f�r att
fylla kontrollerna med data fr�n det valda balksegmentet.

\begin{lstlisting}
procedure TfrmBeamProps.cboSegmentChange(Sender: TObject);
begin
  GetData;
  SetData;
end;
\end{lstlisting}

Vid st�ngning av formul�ret med \pobject{btnClose} eller genom att
klicka p� krysset i �vre h�rnet, m�ste aktuella data i
kontrollerna ocks� sparas till \punit{BeamModel}. Detta g�rs
genom att koppla kod till h�ndelsen \pprop{OnClose}, vilket
anropas innan formul�ret skall st�ngas. H�ndelsen anropar d�
\pmethod{GetData} f�r att h�mta data fr�n kontrollerna. D�refter
tilldelas variabeln \pvar{Action} \ptype{caHide} f�r att
bekr�fta st�ngningen.

\begin{lstlisting}
procedure TfrmBeamProps.FormClose(Sender: TObject;
  var Action: TCloseAction);
begin
  GetData;
  Action:=caHide;
end;
\end{lstlisting}

F�r att visa formul�ret p� sk�rmen m�ste h�ndelsen
\pprop{actProperties} i huvudformul�ret implementeras. I
h�ndelsen visas formul�ret i modalt l�ge, vilket inneb�r att
anv�ndaren endast f�r tillg�ng till detta f�nster och att
huvudformul�ret l�ses. N�r anv�ndaren st�ngt formul�ret �terg�r
programk�rningen till den anropande funktionen. Koden f�r
h�ndelsen visas nedan.

\begin{lstlisting}
procedure TfrmMain.actPropertiesExecute(Sender: TObject);
begin
  frmBeamProps.ShowModal;
  FDirty:=true;
  Self.Invalidate;
end;
\end{lstlisting}

N�r formul�ret visas s�tt \pvar{FDirty} till \pkeyw{true} f�r
att f�rhindra att inaktuella resultat visas. Formul�ret ritas
ocks� om.

%---------------------------------------------------------------------
\subsection{Materialformul�r}
%---------------------------------------------------------------------

I materialformul�ret kan en lista av material- och tv�rsnittsdata
redigeras. H�r finns ocks� knappar f�r att l�gga till och ta bort
data i listan.

\fignormal{kompendiumv4Fig193.eps}{Materialformul�r}{fig:material_window}

Formul�ret fungerar p� samma s�tt som balkegenskapsformul�ret med
metoderna \pmethod{SetData, GetData} och \pmethod{FillListBoxes}
f�r att fylla kontrollerna. Det som skiljer �r knapparna f�r att
l�gga till och ta bort material. F�r att l�gga till ett material
anv�nds rutinen \pmethod{BeamModel.AddMaterial} i
\punit{BeamModel}, d�refter uppdateras listboxen med de nya
materialet. F�r att anv�ndaren direkt skall kunna redigera
materialet s�tts aktuellt material i listboxen till den sista
positionen. D�refter anropas \pmethod{SetData} f�r att fylla
kontrollerna med aktuellt material. Koden f�r L�gg till visas
nedan.

\begin{lstlisting}
procedure TfrmMaterials.btnAddClick(Sender: TObject);
begin
  BeamModel.AddMaterial(1, 1, 1);
  FillListBoxes;
  FCurrentMaterial:=BeamModel.GetNumberOfMaterials;
  lbMaterials.ItemIndex:=FCurrentMaterial-1;
  SetData;
end;
\end{lstlisting}

L�gg m�rke till att \pprop{lbMaterials.ItemIndex} har l�gsta
index 0 och \punit{BeamModel}:s index alltid b�rjar p� 1, varf�r
1 dras ifr�n \pvar{FCurrentMaterial}.

F�r att ta bort ett material anv�nds
\pmethod{BeamModel.RemoveMaterial}. Denna rutin ta bort ett
material p� givet index. Material efter aktuellt index flyttas upp
ett steg. Koden f�r denna h�ndelse visas nedan.

\begin{lstlisting}
procedure TfrmMaterials.btnRemoveClick(Sender: TObject);
begin
  BeamModel.RemoveMaterial(lbMaterials.ItemIndex+1);
  FillListBoxes;
  FCurrentMaterial:=1;
  SetData;
end;
\end{lstlisting}

Visning av formul�ret sker p� samma s�tt som f�r
balkegenskapsformul�ret genom att implementera en h�ndelse i
huvudformul�ret. I detta fall \pprop{actMaterials}.

%---------------------------------------------------------------------
\subsection{Randvillkorsformul�r}
%---------------------------------------------------------------------

I randvillkorsformul�ret kan f�rskjutningar f�reskrivas f�r alla
noder i den kontinuerliga balken.

\figmedium{kompendiumv4Fig194.eps}{Randvillkorsformul�r}{fig:bc_window}

Detta formul�r �r uppbyggt p� samma s�tt som
balkegenskapsformul�ret med \pmethod{SetData}, \pmethod{GetData}
och \pmethod{FillListBoxes}. Formul�ret anv�nder metoderna
\pmethod{BeamModel.SetBC} och \pmethod{BeamModel.GetBC} f�r att
manipulera randvillkoren definierade i \punit{BeamModel}.

Visning av formul�ret sker p� samma s�tt som f�r
balkegenskapsformul�ret genom att implementera en h�ndelse i
huvudformul�ret. I detta fall \pprop{actBC}.\\

%---------------------------------------------------------------------
\subsection{Resultatformul�r}
%---------------------------------------------------------------------

Resultatformul�ret visar en tabell med ber�kningsresultatet f�r
varje balksegment. V�xling mellan balksegmenten kan ske med hj�lp
av combobox-kontrollen l�ngst upp p� formul�ret.

\fignormal{kompendiumv4Fig195.eps}{Resultatformul�r}{fig:result_window}

Hanteringen av formul�ret sker med tv� metoder
\pmethod{FillListBoxes}, \pmethod{FillGrid} och variabeln
\pvar{FCurrentBeam}. Den sista variabeln anger vilket balksegment
som skall visas. \pmethod{FillListBoxes} fyller kontrollen
\pobject{cboBeam} med aktuellt antal balkar. \pmethod{FillGrid}
fyller tabellen med v�rden enligt kod nedan.

\begin{lstlisting}
procedure TfrmResults.FillGrid;
var
    i : integer;
    BeamLength : double;
    BeamLoad : double;
    BeamProp : integer;
begin
  stgrResults.Cells[0,0]:='x (m)';
  stgrResults.Cells[1,0]:='Moment (Nm)';
  stgrResults.Cells[2,0]:='Tv�rkraft (N)';
  stgrResults.Cells[3,0]:='Deform. (m)';

  FCurrentBeam:=cboBeam.ItemIndex+1;

  BeamModel.GetBeam(FCurrentBeam, BeamLength, BeamLoad,
                    BeamProp);

  for i:=1 to MaxEvaluationPoints do
  begin

    stgrResults.Cells[0,i]:=
      format('%.4g',
        [(BeamLength/(MaxEvaluationPoints-1))*(i-1)]);

    stgrResults.Cells[1,i]:=
      format('%.4g',[BeamModel.GetMoment(FCurrentBeam,i)]);

    stgrResults.Cells[2,i]:=
      format('%.4g',
        [BeamModel.GetShearForce(FCurrentBeam,i)]);

    stgrResults.Cells[3,i]:=
      format('%.4g',
        [BeamModel.GetDeflection(FCurrentBeam,i)]);
  end;
end;
\end{lstlisting}

Visning av formul�ret sker p� samma s�tt som f�r
balkegenskapsformul�ret genom att implementera en h�ndelse i
huvudformul�ret. I detta fall \pprop{actResults}.

%---------------------------------------------------------------------
\subsection{Det f�rdiga programmet}
%---------------------------------------------------------------------

Ett n�stan komplett ber�kningsprogram har nu tagits fram.
Figur~\ref{fig:finished_application} visar det f�rdiga
anv�ndargr�nssnittet. Figuren ovan visar ocks� hur verktygsf�lten
kan dras loss och placeras p� godtycklig plats p� sk�rmen, samt
vilka typer av resultat, som kan erh�llas.

Programmet �r givetvis inte f�rdigimplementerat. Det saknas ett
antal delar f�r att det skall bli anv�ndbart. Nedan listas en del
f�rslag till ut�kningar:

\begin{xlist}
\item Felhantering. \\
\item Ber�kning av max/min-v�rden och visning av dessa i den grafiska redovisningen. \\
\item Punktlaster. \\
\item Utskrift av ber�kningsrapport till skrivare. \\
\item Installationsprogram. \\
\item Lastfallshantering. \\
\end{xlist}

\fignormal{kompendiumv4Fig196.eps}{Program f�r ber�kning av
kontinuerliga balkar}{fig:finished_application}

\clearpage


\begin{thebibliography}{}
\addcontentsline{toc}{chapter}{Litteraturförteckning}
\bibitem{metcalf00} Michael Metcalf and John Reid, Fortran 90/95 Explained, Oxford University Press Inc, New York, 1996
\bibitem{ottosen92} Niels Ottosen \& Hans Petersson, Introduction to the Finite Element Method, Prentice Hall International (UK) Ltd, 1992
\bibitem{fortran90} Fortran 90, ISO/IEC 1539 : 1991, International Organisation for Standardization, http://www.iso.ch/
\bibitem{calfem99} CALFEM -- A finite element toolbox to MATLAB version 3.3, Division of Structural Mechanics, 1999
\bibitem{gnumake12} GNU Make Manual, http://www.gnu.org/software/make/manual/, 2012
\bibitem{cmakecond12} Comments on CMakeLists.txt, http://www.vtk.org/Wiki/CMakeForFortranExample\#Comments\_on\_CMakeLists.txt, 2012
\end{thebibliography}


\appendix

%---------------------------------------------------------------------
%---------------------------------------------------------------------
\chapter{Quick compilation guide} \label{app:compiling_fortran}
%---------------------------------------------------------------------
%---------------------------------------------------------------------

This chapters is a quick guide on how to compile simple Fortran application needed for the exercises in this book. The examples uses the gfortran compiler.

%---------------------------------------------------------------------
\section{Compiling single source Fortran programs}
%---------------------------------------------------------------------

To compile a simple Fortran 90 source file into an executable, execute the \fname{gfortran}-command with the source file as the only argument:

\cmdmode

\begin{lstlisting}
$ ls
myprog.f90
$ gfortran myprog.f90 
$ ls
a.out		myprog.f90
\end{lstlisting}

This produces an executable called \fname{a.out}, which can be executed using the following command:

\begin{lstlisting}
$ ./a.out 
 Hello, World!
\end{lstlisting}

By default, the name of the executable will be \fname{a.out}, which is not always the best name for an executable. To tell the compiler to name the executable to something more meaningful the command line switch, \fname{-o}, can be used as shown in the following example:

\begin{lstlisting}
$ gfortran myprog.f90 -o myprog
$ ls
myprog		myprog.f90
$ ./myprog 
 Hello, World!
\end{lstlisting}

%---------------------------------------------------------------------
\section{Compiling multi-source Fortran programs}
%---------------------------------------------------------------------

Often a Fortran application consists of multiple source files. To compile multiple source files, gfortran supports adding additional source files as parameters on the command line as shown below:

\begin{lstlisting}
$ gfortran mymodule.f90 myprog.f90 -o myprog
$ ls
mymodule.f90	mymodule.mod	myprog		myprog.f90
$ ./myprog 
 Hello, World!
\end{lstlisting}

However, the order of the source files are important. If a module depends on a another module the dependent module needs to be built first as the first module uses a special \fname{.mod}-file generated during the compilation. 

%---------------------------------------------------------------------
\section{Compiling with optimisation levels}
%---------------------------------------------------------------------

By default, gfortran, does not optimise the generated code in anyway. To increase performance optimisation options must be specified. There are 3 levels default optimisation available in the \fname{-O1}, \fname{-O2} and \fname{-O3} compiler command line options. 

\begin{itemize}
\item O1 - Tries to reduce code size and execution time, but without increasing compilation time.
\item O2 - Adds allmost all optimisation that does not increase the code size. No loop unrolling is done.
\item O3 - Applies all optimisation options even those that increases code size. Loop unrolling is done.
\end{itemize}

To compile with optimisation just add the above options as the first option to the compiler command as shown in the following example:

\begin{lstlisting}
$ gfortran -O3 mymodule.f90 myprog.f90 -o myprog
\end{lstlisting}

%---------------------------------------------------------------------
\section{Compiling for debugging}
%---------------------------------------------------------------------

To debug a compiled executable using gdb or a graphical debugger, the executable needs to have debugging information included in the binary. By default gfortran does not add any debugging information in the executable. To tell the compiler to include this in the binary, the \fname{-g}-switch must be used. The following commands show how a debug enabled executable is built:

\begin{lstlisting}
$ gfortran -g mymodule.f90 myprog.f90 -o myprog
\end{lstlisting}

It is possible to add debug information to an optimised code as well. However, the execution path of an optimised executable is not always obvious. It is also possible that certain variables have been eliminated by the optimisation options of the compiler.

%---------------------------------------------------------------------
\section{Compiling with more detailed code checking}
%---------------------------------------------------------------------

Gfortran by default does not report all code issues in the source code. The level of code checks can be increased by using the \fname{-pedantic} or \fname{-Wall} switches. The \fname{-pedantic} option warns of the use of extensions to the used Fortran standard (by default Fortran 95). Can also be used together with the switches, \fname{-std=f95}, \fname{-std=f2003} and \fname{-std=f2008} to check for extension to other Fortran standards. In the below example the code is compiled with the \fname{-pedantic} option:

\begin{lstlisting}
$ gfortran -pedantic mymodule.f90 myprog.f90 -o myprog
myprog.f90:3.1:

 implicit none
 1
Warning: Nonconforming tab character at (1)
myprog.f90:5.1:

 print*, 'Hello, World!'
 1
Warning: Nonconforming tab character at (1)
\end{lstlisting}

In the above example, gfortran complains about the use of a tab-character in the source files. The tab-character is not part of the Fortran standard.

The \fname{-Wall} switch tells the compiler to check for code practices that should be avoided. The example below shows how it can be used:

\begin{lstlisting}
$ gfortran -Wall mymodule.f90 myprog.f90 -o myprog
myprog.f90:3.1:

 implicit none
 1
Warning: Nonconforming tab character at (1)
myprog.f90:5.1:

 print*, 'Hello, World!'
 1
Warning: Nonconforming tab character at (1)
\end{lstlisting}

%---------------------------------------------------------------------
\section{Compiling with runtime checks}
%---------------------------------------------------------------------

Some application errors can not be detected at compile time. To check for these errors, gfortran can add checks in the executable for these. To enable a certain check the \fname{-fcheck}-switch can be used to enable specific checks.  Common checks are:

\begin{itemize}
\item \fname{-fcheck:bounds} - Accessing array elements outside its bounds.
\item \fname{-fcheck:do} - Modification of loop variables.
\item \fname{-fcheck:mem} - Memory allocation and deallocation.
\item \fname{-fcheck:pointer} - Runtime checks for pointer handling.
\item \fname{-fcheck:array-temps} - Check for when an array-temporary has to be created for passing an argument.
\item \fname{-fcheck:all} - Add all available runtime checks.
\end{itemize}

It is important to remove these checks in the final code as they add an additional overhead in the execution speed.






%---------------------------------------------------------------------
%---------------------------------------------------------------------
\chapter{Exercises} \label{app:exercises}
%---------------------------------------------------------------------
%---------------------------------------------------------------------

\newcommand{\excn}[1]{\textsf{\textbf{#1}}}

%---------------------------------------------------------------------
\section{Fortran}
%---------------------------------------------------------------------

\fmode

%---------------------------------------------------------------------

\begin{tabular}{lp{0.8\textwidth}}
\excn{1-1} & Which of the following names can be used as Fortran variable names? \\
& \\
& a) \fvar{number\_of\_stars} \\
& b) \fvar{fortran\_is\_a\_nice\_language\_to\_use}\\
& c) \fvar{2001\_a\_space\_odyssey} \\
& d) \fvar{more\$\_money}\\
& \\
\excn{1-2} & Declare the following variables in Fortran: 2 scalar integers
\fvar{a}, and \fvar{b}, 3 floating point scalars \fvar{c}, \fvar{d}
and \fvar{e}, 2 character strings \fvar{infile} and \fvar{outfile}
and a logical variable \fvar{f}.\\
& \\
\excn{1-3} & Declare a floating point variable \textit{a} that can
represent values between 10$^{-150}$ and 10$^{150}$ with 14 significatn numbers. \\
& \\
\end{tabular}

%---------------------------------------------------------------------

\begin{tabular}{lp{0.8\textwidth}}
\excn{1-4} & What is printed by the following program?\\
&
\begin{minipage}{0.8\textwidth}
\begin{lstlisting}
program precision

      implicit none

      integer, parameter :: ap = &
        selected_real_kind(15,300)

      real(ap) :: a, b

      a = 1.234567890123456
      b = 1.234567890123456_ap

      if (a==b) then
           write(*,*) 'Values are equal.'
      else
           write(*,*) 'Values are different.'
      endif

      stop

end program precision
\end{lstlisting}
\end{minipage} \\
& \\
\excn{1-5} & Declare a $[3 \times 3]$ floating point array
,\fvar{Ke}, and an 3 element integer array, \fvar{f}\\
& \\
\excn{1-6} & Declare an integer array, \fvar{idx}, with the following indices \\
&\fvar{[0, 1, 2, 3, 4, 5, 6, 7]}.\\
& \\
\excn{1-7} & Give the following assignments:\\
& Floating point array, \fvar{A}, is assigned the value 5.0 at
(2,3).\\
& Integer matrix, \fvar{C}, is assigned the value 0 at row 2.\\
& \\
\excn{1-8} & Give the following if-statements:\\
& \\
& If the value of the variable,\fvar{i}, is greater than 100
print 'i is greater than 100!'\\
& \\
& If the value of the logical variable, \fvar{extra\_filling},
is true print 'Extra filling is ordered.', otherwise print
'No extra filling.'.\\
& \\
\excn{1-9} & Give a case-statment for the variable, \textit{a},
printing 'a is 1' when a is 1, 'a is between 1 and 20' for
values between 1 and 20 and prints 'a is not between 1 and 20'
for all other values. \\
\end{tabular}

%---------------------------------------------------------------------

\begin{tabular}{lp{0.8\textwidth}}
\excn{1-10} & Write a program consisting of a do-statement 1 to 20 with the control variable, \fvar{i}.
For values, \fvar{i}, between 1 till 5, the value of \fvar{i} is printed, otherwise 'i>5' is printed.
The loop is to be terminated when \fvar{i} equals 15.\\
& \\
\excn{1-11} & Write a program declaring a floating point matrix, \fvar{I}, with the dimensions
$[10 \times 10]$ and initialises it with the identity matrix. \\
\excn{1-12} & Give the following expressions in Fortran: \\
& \\
& a) $\frac{1}{\sqrt{2}}$ \\
& b) $e^{x} \sin ^{2} x$ \\
& c) $\sqrt{a^{2} +b^{2}}$ \\
& d) $\left| x-y\right|$ \\
& \\
\excn{1-13} & Give the following matrix and vector expressions in Fortran.
Also give appropriate array declarations: \\
& \\
& a) $\mathbf{AB}$ \\
& b) $\mathbf{A^{T} A}$ \\
& c) $\mathbf{ABC}$ \\
& d) $\mathbf{a\cdot b}$ \\
& \\
\excn{1-14} & Show expressions in Fortran calculating
maximum, mininmum, sum and product of the elements of an array.\\
& \\
\excn{1-15} & Declare an allocatable 2-dimensional floating point array and
a 1-dimensional floating point vector. Also show program-statements how memory
for these variables are allocated and deallocated.\\
& \\
\excn{1-16} & Create a subroutine, \fname{identity}, initialising a {\underline {arbitrary}}
two-dimensionl to the identity matrix. Write a program illustrating the use of the subroutine.\\
& \\
\excn{1-17} & Implement a function returning the value of the the following expression: \\
& \\
& $e^{x} \sin ^{2} x$ \\
\end{tabular}

%---------------------------------------------------------------------

\begin{tabular}{lp{0.8\textwidth}}
\excn{1-18} & Write a program listing \\
& $f(x)=\sin x$ from $-1.0$ to $1.0$ in intervals of $0.1$. The output from the program
should have the following format:\\
&
\begin{minipage}{0.8\textwidth}
\begin{lstlisting}
         111111111122222222223
123456789012345678901234567890
 x      f(x)
-1.000 -0.841
-0.900 -0.783
-0.800 -0.717
-0.700 -0.644
-0.600 -0.565
-0.500 -0.479
-0.400 -0.389
-0.300 -0.296
-0.200 -0.199
-0.100 -0.100
 0.000  0.000
 0.100  0.100
 0.200  0.199
 0.300  0.296
 0.400  0.389
 0.500  0.479
 0.600  0.565
 0.700  0.644
 0.800  0.717
 0.900  0.783
 1.000  0.841
\end{lstlisting}
\end{minipage} \\
& \\
\excn{1-19} & Write a program calculating the total length of a piecewise linear curve. The curve is defined
in a textfile \ffname{line.dat}. \\
& \\
& The file has the following structure:\\
& \\
&\{number of points n in the file\}\\
&\{x-coordinate point 1\} \{y-coordinate point 1\}\\
&\{x-coordinate point 2\} \{y-coordinate point 2\}\\
&.\\
&.\\
&\{x-coordinate point n\} \{y-coordinate point n\}\\
& \\
&The program must not contain any limitations regarding the number of points in
the number of points in the curve read from the file.\\
\end{tabular}

%---------------------------------------------------------------------

\begin{tabular}{lp{0.8\textwidth}}
\excn{1-20} & Declare 3 strings, \fvar{c1}, \fvar{c2} and
\fvar{c3} containing the words 'Fortran', 'is' och 'fun'. Merge these into a
new string, \fvar{c4}, making a complete sentence.\\
& \\
\excn{1-21} & Write a function converting a string into a floating point value.
Write a program illustrating the use of the function.\\
& \\
\excn{1-22} & Create a module, \fmodule{conversions}, containing the function in 1-21
and a function for converting a string to an integer value. Change the program in 1-21 to
use this module. The module is placed in a separate file, \ffname{conversions.f90} and
the main program in \ffname{main.f90}. \\
\end{tabular}

%%---------------------------------------------------------------------
%\section{Object Pascal}
%%---------------------------------------------------------------------
%
%\pmode
%
%\begin{tabular}{lp{0.8\textwidth}}
%\excn{2-1} & Vilka av följande beteckningar kan användas som variabelnamn
%i Object Pascal? \\
%& \\
%& a) \fvar{number\_of\_stars} \\
%& b) \fvar{pascal\_is\_a\_nice\_language\_to\_use}\\
%& c) \fvar{2001\_a\_space\_odyssey} \\
%& d) \fvar{more\$\_money}\\
%& \\
%\excn{2-2} & Deklarera följande variabler i Object Pascal: två
%heltalsskalärer \pvar{a} och \pvar{b}, tre flyttalsskalärer
%\pvar{c}, \pvar{d} och \pvar{e}, två
%teckensträngar \pvar{infile} och \pvar{outfile} och en logisk variabel \pvar{f}.\\
%& \\
%\excn{2-3} & Ange följande if-satser:\\
%& \\
%& Om värdet på variabeln i är större än 100 skriv ut
%'i större än hundra!'\\
%& \\
%& Om värdet på den logiska variabeln extra\_fyllning är sann
%skrivs 'Extra fyllning beställd.', annars skrivs 'Ingen extra
%fyllning.'.\\
%& \\
%\excn{2-4} & Ange en case-sats för variabeln \pvar{a} som skriver
%ut 'a är 1' för värdet 1, 'a är mellan 2 och 20' för värden mellan
%2 och 20, och skriver ut 'a är inte mellan 1 och 20' för alla
%andra tal.\\
%& \\
%\excn{2-5} & Ange en slinga 1 till 20 med styrvariabeln \pvar{i}.
%För \pvar{i} mellan 1 till 5 skrivs värdet på \pvar{i} ut på
%skärmen, annars skrivs
%'i>5' ut.\\
%& \\
%\excn{2-6} & Skriv ett program som deklarerar en flyttalsmatris
%\pvar{I} med storleken $[10 \times 10]$ och sedan initialiserar denna med enhetsmatrisen.\\
%& \\
%\excn{2-7} & Ange följande uttryck i Object Pascal: \\
%& \\
%& a) $\frac{1}{\sqrt{2}}$ \\
%& b) $e^{x} \sin ^{2} x$ \\
%& c) $\sqrt{a^{2} +b^{2}}$ \\
%& d) $\left| x-y\right|$ \\
%& \\
%\end{tabular}
%
%%---------------------------------------------------------------------
%
%\begin{tabular}{lp{0.8\textwidth}}
%\excn{2-8} & Deklarera en dynamisk flyttalsmatris och en dynamisk
%flyttalsvektor. Visa med programsatser hur minne allokeras och
%frigörs för dessa variabler\\
%& \\
%\excn{2-9} & Skapa en subrutin identity som initialiserar en
%godtycklig matris till enhetsmatrisen. Visa genom att skriva ett
%huvudprogram
%hur subrutinen anropas.\\
%& \\
%\excn{2-10} & Skapa en funktion som returnerar värdet av uttrycket \\
%& \\
%& $e^{x} \sin ^{2} x$ \\
%& \\
%& Skriv också ett program som skriver ut funktionsvärdet när $x =
%1.0$.\\
%& \\
%\excn{2-11} & Skriv ett program som listar \\
%& $f(x)=\sin x$ från --1.0 till 1.0 i steg om 0.1. Utskriften från programmet skall ha följande utseende:\\
%&
%\begin{minipage}{0.8\textwidth}
%\begin{lstlisting}
%         111111111122222222223
%123456789012345678901234567890
% x      f(x)
%-1.000 -0.841
%-0.900 -0.783
%-0.800 -0.717
%-0.700 -0.644
%-0.600 -0.565
%-0.500 -0.479
%-0.400 -0.389
%-0.300 -0.296
%-0.200 -0.199
%-0.100 -0.100
% 0.000  0.000
% 0.100  0.100
% 0.200  0.199
% 0.300  0.296
% 0.400  0.389
% 0.500  0.479
% 0.600  0.565
% 0.700  0.644
% 0.800  0.717
% 0.900  0.783
% 1.000  0.841
%\end{lstlisting}
%\end{minipage} \\
%\end{tabular}
%
%%---------------------------------------------------------------------
%
%\begin{tabular}{lp{0.8\textwidth}}
%\excn{2-12} & Deklarera tre strängar \pvar{c1}, \pvar{c2} och \pvar{c3} som innhåller orden
%'Object Pascal', 'är' och 'kul'. Slå ihop dessa till en ny sträng
%\pvar{c4}, så att de bildar en komplett mening.\\
%& \\
%\excn{2-13} & Skriv en enhet/unit med funktionerna: \pmethod{identity}
%för att initialisera en matris till enhetsmatrisen, \pmethod{zero}
%för att nollställa en matris och \pmethod{transpose} för att
%transponera en matris. Funktionerna skall använda
%\ptype{variant}-matriser och kunna hantera godtyckligt stora
%\ptype{variant}-matriser. Skriv
%också ett huvudprogram som använder rutinerna i enheten.\\
%\end{tabular}

%%---------------------------------------------------------------------
%---------------------------------------------------------------------
\chapter{L�sningar till �vningsuppgifter} \label{app:solutions}
%---------------------------------------------------------------------
%---------------------------------------------------------------------

\renewcommand{\excn}[1]{\textsf{\textbf{#1}}}

%---------------------------------------------------------------------
\section{Fortran}
%---------------------------------------------------------------------

\fmode

%---------------------------------------------------------------------

\begin{tabular}[t]{lp{0.8\textwidth}}
\excn{1-1} & Endast a) �r en korrekt Fortran variabel. b) innhe�ller
mer �n 31 tecken. c) b�rjar ej med en bokstav. d) inneh�ller ett
icke-alfanumeriskt tecken. \\
& \\
\excn{1-2} & \vspace{-2em}
\begin{lstlisting}
integer :: a, b
real(8) :: c, d, e
character(255) :: infile
character(255) :: outfile
logical :: f
\end{lstlisting}\\
& \\
\excn{1-3} & \vspace{-2em}
\begin{lstlisting}
integer, parameter :: ap = &
    selected_real_kind(14,150)

real(ap) :: a
\end{lstlisting} \\
& \\
\excn{1-4} & ''Talen �r olika.'' kommer att skrivas ut. Om inget
annat anges representeras skal�rkonstanter med 7 signifikanta
siffror och kan ta v�rden i omr�det $10^{-38}$ till $10^{38}$.
Skal�rkonstanten som tilldelas a i exemplet kommer ej att kunna
representeras med mer �n 7 signifikanta siffror. Skal�ren som
tilldelas b �r deklarerad med 15 signifikanta siffror med suffixet
\ftype{ap} och kommer att kunna representera hela det angivna
talet. Vid j�mf�relsen kommer variablerna \fvar{a} och \fvar{b}
att inneh�lla olika v�rden, trots att \fvar{a} och \fvar{b} �r deklarerade p� samma s�tt. \\
& \\
\end{tabular}

%---------------------------------------------------------------------

\begin{tabular}[t]{lp{0.8\textwidth}}
\excn{1-5} & \vspace{-2em}
\begin{lstlisting}
integer, parameter :: ap = &
    selected_real_kind(15,300)

real(ap) :: Ke(3,3)
real(ap) :: f(3)
\end{lstlisting} \\
& \\
\excn{1-6} & \vspace{-2em}
\begin{lstlisting}
integer :: idx(0:7)
\end{lstlisting} \\
& \\
\excn{1-7} & \vspace{-2em}
\begin{lstlisting}
1-7 A(2,3) = 5.0
B = 0.0
C(2,:) = 0
\end{lstlisting} \\
& \\
\excn{1-8} & \vspace{-2em}
\begin{lstlisting}
if (i>100) then
    write(*,*) 'i storre �n hundra!'
end if

if (extra_fyllning) then
    write(*,*) 'Extra fyllning best�lld.'
else
    write(*,*) 'Ingen extra fyllning.'
end if
\end{lstlisting} \\
& \\
\excn{1-9} & \vspace{-2em}
\begin{lstlisting}
select case (a)
    case 1
        write(*,*) 'a �r 1.'
    case 2:20
        write(*,*) 'a �r mellan 2 och 20.'
    case default
        write(*,*) 'a �r inte mellan 1 och 20.'
end select
\end{lstlisting} \\
& \\
\end{tabular}

%---------------------------------------------------------------------

\begin{tabular}[t]{lp{0.8\textwidth}}
\excn{1-10} & \vspace{-2em}
\begin{lstlisting}
do i=1,20
    if (i<6) then
        write(*,*) i
    else
        write(*,*) 'i>5'
    end if

    if (i==15) then
        break
    end if
end do
\end{lstlisting} \\
& \\
\excn{1-11} & \vspace{-2em}
\begin{lstlisting}
integer, parameter :: ap = &
    selected_real_kind(15,300)

real(ap) :: I(10,10)
integer :: row, col

do row=1,10
    do col=1,10
        if (row==col) then
            I(row,col) = 1.0
        else
            I(row,col) = 0.0
        end if
    end do
end do
\end{lstlisting} \\
& \\
\excn{1-12} & a) \fsol{1/sqrt(2.0\_ap)} \\
& b) \fsol{exp(x)*sin(x)**2} \\
& c) \fsol{sqrt(a**2 + b**2)} \\
& d) \fsol{abs(x-y))} \\
& \\
\excn{1-13} & a) \fsol{matmul(A,B)  } \\
& b) \fsol{matmul(transpose(A),A)} \\
& c) \fsol{matmul(A,matmul(B,C))} \\
& d) \fsol{dot\_product(a,b)} \\
& \\
\excn{1-14} & \vspace{-2em}
\begin{lstlisting}
b = maxval(A)
c = minval(A)
d = product(A)
e = sum(A)
\end{lstlisting} \\
\end{tabular}

%---------------------------------------------------------------------

\begin{tabular}[t]{lp{0.8\textwidth}}
\excn{1-15} & \vspace{-2em}
\begin{lstlisting}
integer, parameter :: ap = &
    selected_real_kind(15,300)

real(ap), allocatable :: A(:,:)
real(ap), allocatable :: b(:)

allocate(A(20,20))
allocate(b(30))

A = 0.0_ap
b = 0.0_ap

deallocate(A)
deallocate(b)
\end{lstlisting} \\
& \\
\excn{1-16} & \vspace{-2em}
\begin{lstlisting}
program subtest

    integer, parameter :: ap = ...

    real(ap) :: M(20,20)
    call identity(M,20)
    stop

contains

subroutine identity(A,n)

    implicit none
    integer, parameter :: ap = ...

    real(ap) :: A(n,*)
    integer :: row, col

    do row=1,n
        do col=1,n
            if (row==col) then
                A(row,col) = 1.0
            else
                A(row,col) = 0.0
            end if
        end do
    end do

    return

end subroutine identity
\end{lstlisting}\\
\end{tabular}

%---------------------------------------------------------------------

\begin{tabular}[t]{lp{0.8\textwidth}}
\excn{1-17} & \vspace{-2em}
\begin{lstlisting}
real function f(x)

    implicit none

    real(8) :: x

    f = exp(x)*sin(x)**2

    return

end function f
\end{lstlisting} \\
& \\
\excn{1-18} & \vspace{-2em}
\begin{lstlisting}
program exercise1_18

    implicit none
    integer, parameter :: ap = &
        selected_real_kind(15,300)

    real(ap) :: x
    real(ap) :: f

    write(*,'(T5,A,T21,A)') 'x','f(x)'

    x = -1.0_ap
    do while (x<1.05)
        f = sin(x)
        write(*,'(T4,F6.3,T20,F6.3)') x, f
        x = x + 0.1_ap
    end do

    stop

end program exercise1_18
\end{lstlisting} \\
\end{tabular}

%---------------------------------------------------------------------

\begin{tabular}[t]{lp{0.8\textwidth}}
\excn{1-19} & \vspace{-2em}
\begin{lstlisting}
program exercise1_19

    implicit none

    integer, parameter :: ap = &
        selected_real_kind(15,300)
    integer, parameter :: linjeFil = 15
    integer :: nPoints
    integer :: i
    real(ap) :: x, y, lastX, lastY
    real(ap) :: summedLength

    open(unit=linjeFil, file='linje.dat', &
        access='sequential', &
        action='read', status='old')

    read(linjeFil,*) nPoints

    summedLength = 0.0_ap

    do i=1,nPoints
        read(linjeFil,*) x, y

        if (i>1) then
            summedLength = summedLength + &
                   sqrt((x-lastX)**2+(y-lastY)**2)
        end if


        lastX = x
        lastY = y
    end do

    close(linjeFil)

    write(*,'(T1,A,G15.3)') 'Total langd = ', &
        summedLength

end program exercise1_19
\end{lstlisting}\\
\end{tabular}

%---------------------------------------------------------------------

\begin{tabular}[t]{lp{0.8\textwidth}}
\excn{1-20} & \vspace{-2em}
\begin{lstlisting}
program exercise1_20

    implicit none

    character(7) :: c1
    character(2) :: c2
    character(3) :: c3
    character(15) :: c4

    c1 = 'Fortran'
    c2 = '�r'
    c3 = 'kul'

    c4 = c1//' '//c2//' '//c3

    write(*,*) c4

    stop

end program exercise1_20
\end{lstlisting} \\
& \\
\excn{1-21} & \vspace{-2em}
\begin{lstlisting}
program excercise1_21

    implicit none
    character(20) :: c
    real(8) :: value

    c = '2.0'
    value = toReal(c)
    write(*,*) value

    stop

contains

real(8) function toReal(c)

    implicit none
    character(20), intent(in) :: c

    read(c, *) toReal
    return

end function toReal

end program excercise1_21
\end{lstlisting} \\
\end{tabular}

%---------------------------------------------------------------------

\begin{tabular}[t]{lp{0.8\textwidth}}
\excn{1-22} & \vspace{-2em}
\begin{lstlisting}
module conversions

    implicit none

contains

real(8) function toReal(c)

    implicit none
    character(20), intent(in) :: c
    read(c, *) toReal
    return

end function toReal

integer function toInteger(c)

    implicit none
    character(20), intent(in) :: c
    read(c, *) toInteger
    return

end function toInteger

end module conversions

! ------------------------------

program exercise2_22

    use conversions

    character(20) :: c
    character(20) :: d
    real(8) :: floatValue
    integer :: intValue

    c = '2.0'
    d = '42'

    floatValue = toReal(c)
    intValue = toInteger(d)

    write(*,*) floatValue
    write(*,*) intValue

    stop

end program exercise2_22
\end{lstlisting} \\
\end{tabular}

%%---------------------------------------------------------------------
%\section{Object Pascal}
%%---------------------------------------------------------------------
%
%\pmode
%
%\begin{tabular}[t]{lp{0.8\textwidth}}
%\excn{2-1} & 2-1 a) och b) �r korrekta namn p� variabler i Object
%Pascal. b) b�rjar med en siffra. c) inneh�ller ett icke
%alfanumeriskt tecken. \\
%& \\
%\excn{2-2} & \vspace{-2em}
%\begin{lstlisting}
%a, b : integer;
%c, d, e : double;
%infile, outfile : string;
%boolean : f;
%\end{lstlisting} \\
%& \\
%\excn{2-3} & \vspace{-2em}
%\begin{lstlisting}
%if i>100 then
%    writeln('i st�rre �n hundra!');
%
%if extra_fyllning then
%    writeln('Extra fyllning best�lld.')
%else
%    writeln('Ingen extra fyllning.');
%\end{lstlisting} \\
%& \\
%\excn{2-4} & \vspace{-2em}
%\begin{lstlisting}
%case a of
%    1     : writeln('a �r 1');
%    2..20 : writeln('a �r mellan 2-20.');
%else
%    writeln('a �r inte mellan 1 och 20.');
%end;
%\end{lstlisting} \\
%& \\
%\excn{2-5} & \vspace{-2em}
%\begin{lstlisting}
%for i:=1 to 20 do
%begin
%    if i>5 then
%        writeln('i>5')
%    else
%        writeln(i);
%end;
%\end{lstlisting} \\
%\end{tabular}
%
%%---------------------------------------------------------------------
%
%\begin{tabular}[t]{lp{0.8\textwidth}}
%\excn{2-6} & \vspace{-2em}
%\begin{lstlisting}
%program exercise2_6b;
%{$APPTYPE CONSOLE}
%uses
%    SysUtils;
%
%var
%    I : array [1..10, 1..10] of double;
%
%    row, col : integer;
%
%begin
%    for row:=1 to 10 do
%        for col:=1 to 10 do
%            if row=col then
%                I[row,col]:=1.0
%            else
%                I[row,col]:=0.0;
%end.
%\end{lstlisting} \\
%& \\
%\excn{2-6} & \vspace{-2em}
%\begin{lstlisting}
%program exercise2_6b;
%{$APPTYPE CONSOLE}
%uses
%    SysUtils;
%
%var
%    I : array [1..10, 1..10] of double;
%
%    row, col : integer;
%
%begin
%    for row:=1 to 10 do
%        for col:=1 to 10 do
%            if row=col then
%                I[row,col]:=1.0
%            else
%                I[row,col]:=0.0;
%end.
%\end{lstlisting} \\
%& \\
%\excn{2-7} & a) \psol{1/Sqrt(2)} \\
%& b) \psol{Exp(x)*Power(sin(x),2)} \\
%& c) \psol{Sqrt(Power(a,2)+Power(b,2))} \\
%& d) \psol{Abs(x-y)} \\
%\end{tabular}
%
%%---------------------------------------------------------------------
%
%\begin{tabular}[t]{lp{0.8\textwidth}}
%\excn{2-8} & \vspace{-2em}
%\begin{lstlisting}
%program exercise2_8;
%{$APPTYPE CONSOLE}
%uses
%    SysUtils;
%
%var
%    K : variant;
%    f : variant;
%
%begin
%    K:=VarArrayCreate([1,20,1,20], varDouble);
%    f:=VarArrayCreate([1,20], varDouble);
%
%    // Use matrices
%
%    VarClear(K);
%    VarClear(f);
%end.
%\end{lstlisting} \\
%& \\
%\excn{2-9} & \vspace{-2em}
%\begin{lstlisting}
%program exercise2_9;
%{$APPTYPE CONSOLE}
%uses
%  SysUtils;
%
%procedure Identity(var A : variant; n : integer);
%var
%    i, j : integer;
%begin
%    for i:=1 to n do
%        for j:=1 to n do
%            if i=j then
%                A[i,j]:=1.0
%            else
%                A[i,j]:=0.0;
%end;
%
%procedure Print(var A : variant; n : integer);
%var
%        i, j : integer;
%begin
%    for i:=1 to n do
%        for j:=1 to n do
%            write(A[i,j],', ');
%        writeln;
%    end;
%end;
%\end{lstlisting} \\
%\end{tabular}
%
%%---------------------------------------------------------------------
%
%\begin{tabular}[t]{lp{0.8\textwidth}}
%\excn{2-9} ... & \vspace{-2em}
%\begin{lstlisting}
%var
%    B : variant;
%
%begin
%    B:=VarArrayCreate([1,10,1,10],varDouble);
%    Identity(B,10);
%    Print(B,10);
%    VarClear(B);
%end.
%\end{lstlisting} \\
%& \\
%\excn{2-10} & \vspace{-2em}
%\begin{lstlisting}
%program exercise2_10;
%{$APPTYPE CONSOLE}
%uses
%    SysUtils, Math;
%
%function f(x : double) : double;
%begin
%  Result:=Exp(x)*Power(sin(x),2);
%end;
%
%begin
%    writeln(f(1.0));
%end.
%\end{lstlisting} \\
%& \\
%\excn{2-11} & \vspace{-2em}
%\begin{lstlisting}
%program exercise2_11;
%{$APPTYPE CONSOLE}
%uses
%  SysUtils,
%  Math;
%
%function f(x : double) : double;
%begin
%    Result:=sin(x);
%end;
%
%var
%    x : double;
%    dx : double;
%
%begin
%    writeln('    x              f(x)');
%\end{lstlisting} \\
%& \\
%\end{tabular}
%
%%---------------------------------------------------------------------
%
%\begin{tabular}[t]{lp{0.8\textwidth}}
%\excn{2-11} ... & \vspace{-2em}
%\begin{lstlisting}
%    DecimalSeparator:='.';
%    x:=-1.0;
%    dx:=0.1;
%    while x<1.05 do
%    begin
%        writeln(format('    %6.3f %6.3f',
%            [x, sin(x)]));
%        x:=x+dx;
%    end;
%    readln;
%end.
%\end{lstlisting} \\
%& \\
%\excn{2-12} & \vspace{-2em}
%\begin{lstlisting}[escapechar=\%]
%program exercise2_12;
%{$APPTYPE CONSOLE}
%uses
%    SysUtils, Math;
%var
%    f : TextFile;
%    nPoints, i : integer;
%    x, y, LastX, LastY : double;
%    SummedLength : double;
%begin
%
%    AssignFile(f, 'linje.dat');
%    Reset(f);
%    readln(f,nPoints);
%    SummedLength := 0.0;
%
%    for i:=1 to nPoints do
%    begin
%        readln(f, x, y);
%        if i>1 then
%        begin
%            SummedLength:=SummedLength+
%                Sqrt(Power(x-LastX,2)+Power(y-LastY,2));
%        end;
%
%        LastX:=x;
%        LastY:=y;
%    end;
%
%    CloseFile(f);
%
%    writeln('Total l%�%ngd', SummedLength);
%    readln;
%end.
%\end{lstlisting} \\
%\end{tabular}
%
%%---------------------------------------------------------------------
%
%\begin{tabular}[t]{lp{0.8\textwidth}}
%\excn{2-13} & \vspace{-2em}
%\begin{lstlisting}
%program exercise2_13;
%{$APPTYPE CONSOLE}
%uses
%  SysUtils;
%
%var
%    c1, c2, c3, c4 : string;
%
%begin
%    c1:='Object Pascal';
%    c2:='�r';
%    c3:='kul';
%    c4:=c1+' '+c2+' '+c3+'.';
%    writeln(c4);
%    readln;
%end.
%\end{lstlisting} \\
%& \\
%\excn{2-13} & \vspace{-2em}
%\begin{lstlisting}
%unit Matrix;
%
%interface
%
%procedure Identity(var A : variant; n : integer);
%procedure Zero(var A : variant; n : integer);
%procedure Transpose(var A : variant; n : integer);
%
%implementation
%
%procedure Identity(var A : variant; n : integer);
%var
%    i, j : integer;
%begin
%    for i:=1 to n do
%        for j:=1 to n do
%            if i=j then
%                A[i,j]:=1.0
%            else
%                A[i,j]:=0.0;
%end;
%
%procedure Zero(var A : variant; n : integer);
%var
%    i, j : integer;
%begin
%    for i:=1 to n do
%        for j:=1 to n do
%            A[i,j]:=0.0;
%end;
%\end{lstlisting} \\
%\end{tabular}
%
%%---------------------------------------------------------------------
%
%\begin{tabular}[t]{lp{0.8\textwidth}}
%\excn{2-13} ... & \vspace{-2em}
%\begin{lstlisting}
%procedure Transpose(var A : variant; n : integer);
%var
%    i, j : integer;
%    B : variant;
%begin
%
%    B:=VarArrayCreate([1,n,1,n], varDouble);
%
%    for i:=1 to n do
%        for j:=1 to n do
%            B[i,j]:=A[j,i];
%
%    for i:=1 to n do
%        for j:=1 to n do
%            A[i,j]:=B[i,j];
%end;
%
%end.
%\end{lstlisting} \\
%
%\excn{2-13} ... & \vspace{-2em}
%\begin{lstlisting}[escapechar=\%]
%%Huvudprogram%
%
%program exercise2_14;
%{$APPTYPE CONSOLE}
%uses
%    SysUtils,
%    Matrix in 'Matrix.pas';
%
%var
%    A : variant;
%
%begin
%    A:=VarArrayCreate([1,20,1,20], varDouble);
%    Matrix.Zero(A,20);
%    Matrix.Identity(A,20);
%    Matrix.Transpose(A,20);
%    VarClear(A);
%end.
%\end{lstlisting} \\
%& \\
%\end{tabular}


%%---------------------------------------------------------------------
%---------------------------------------------------------------------
\chapter{K�llkod} \label{app:application_source}
%---------------------------------------------------------------------
%---------------------------------------------------------------------

%---------------------------------------------------------------------
\section{FIW - atan2 example}
%---------------------------------------------------------------------

\subsection*{vatan2.f90}

\fmodesrc \lstinputlisting[texcl]{source/fiw_atan2/vatan2.f90}

\subsection*{fmath.pas}

\pmodesrc \lstinputlisting[texcl]{source/fiw_atan2/fmath.pas}

\subsection*{fmath\_sample.dpr}

\pmodesrc \lstinputlisting[texcl]{source/fiw_atan2/fmath_sample.dpr}

%---------------------------------------------------------------------
\section{Kontinuerlig balk}
%---------------------------------------------------------------------

\subsection{beam.f90}

\fmodesrc \lstinputlisting[texcl]{source/beam/beam.f90}

\subsection{calc.f90}

\fmodesrc \lstinputlisting[texcl]{source/beam/calc.f90}

\subsection{BeamModel.pas}

\pmodesrc \lstinputlisting[texcl]{source/beam/beammodel.pas}

\subsection{BeamDraw.pas}

\pmodesrc \lstinputlisting[texcl]{source/beam/beamdraw.pas}

\subsection{BeamCalc.pas}

\pmodesrc \lstinputlisting[texcl]{source/beam/beamdraw.pas}

\subsection{Huvudformul�r main.pas}

\pmodesrc \lstinputlisting[texcl]{source/beam/main.pas}

\subsection{Balkegenskapsformul�r beamprop.pas}

\pmodesrc \lstinputlisting[texcl]{source/beam/beamprop.pas}

\subsection{Materialformul�r beamprop.pas}

\pmodesrc \lstinputlisting[texcl]{source/beam/beammat.pas}

\subsection{Randvillkorsformul�r beambc.pas}

\pmodesrc \lstinputlisting[texcl]{source/beam/beambc.pas}

\subsection{Resultatformul�r beamresult.pas}

\pmodesrc \lstinputlisting[texcl]{source/beam/beamresult.pas}


\backmatter

\end{document}
